\chapter{Introduction}
\section{Motivation}
%What is text? How is it motivating?

%What is machine learning? What are its advantages? How is it motivating?

%What are the problems with machine learning? How is it motivating?


%What is domain-specific? What is domain knowledge? %On the web there is a large volume of raw text data, e.g. Reviews of products, movies, anime, books, music, social posts by individuals, self-descriptive text about a website or product, and so on. These can be categorized into domains; each domain has its own quirks, knowledge, and method of being brought about. Although a movie review may sound similar to a book review, they typically differ hugely in the distribution of words used.

%

\section{Interpretability}

%What is interpretability? How is the value of interpretability measured in the real world?

%How can we meet the needs of the real world?  Is it transparancy, the system having easy to understnad components, etc... what  are the different views on what an interpretabile system is?

%What specific interpretability task are we trying to solve? How do we define interpretability? Why is it valuable, where is it used? What was the hypothesis/research question?
%%What are distributional models?
Most successful approaches in recent times, like vector-spaces, word-vectors, and others, rely on the distributional model of semantics. This model relies on encoding unstructured text e.g. of a movie review, as a vector, where each dimension corresponds to how frequent each word is, we are able to calculate how similar the entities are, e.g. we know that if two movies have a similar distribution of words in their reviews, like frequent use of the word 'scary', or 'horror', then they would have a higher similarity value. These models, also known as 'semantic spaces' encode this similarity information spatially.

applications/need for good interpretability:
\begin{itemize}
	\item Safety
	\item Troubleshooting, bug fixing, model improvement
	\item Knowledge learning
	\item EU's "Right to explanation"
	\item Discrimination
\end{itemize}

%What is a conceptual space? What are entities?  What are properties? What is commonsense reasoning?
properties of an interpretable classifier:
\begin{itemize}
	\item Complexity: 'the magic number is seven plus or minus two' \cite{Saaty2003} also has many positive effects for its users, like lower response times \cite{Narayanan2018, Huysmans2011}, better question answering and confidence for logical problem questions \cite{Huysmans2011} and higher satisfaction \cite{Narayanan2018}.
	\item Transparancy: 
	\item Explainability: 
	\item Generalizability:
\end{itemize}


%X%X%What is a symbolic approach?  %One approach to making sense of these domains is to produce rules from expert knowledge. An expert in movies would tell you that if the review talks about it being a "cannibal horror film", we can understand that it is likely a scary movie and is related to the original 'Cannibal Holocaust' movie. Encoding this kind of knowledge is difficult, time-consuming, and hard to automate reliably.

Properties, entities, the benefits and application of a representation formed of these

Basic introduction to directions, explanation of the utility and application of our approach
\section{Thesis Overview / Contributions}

%What were our objectives starting out? 
%What are our intentions with how the work in the thesis will be used?
%What are our contributions?
%%What are our aims for this chapter? What do we overall want to do? (Already kind-of said in Chapter 1, but worth repeating I guess in some form)
%In \ref{Chapter3}, we introduce a pipeline that starts with unstructured text, and ends with an interpretable representation of entities represented by properties labelled by clusters of words. Further, we demonstrate the applicability of these representations in a simple Decision Tree that uses just a few of these properties to classify entities. In Figure \ref{ExamplesWithTree}, we show some example movie entities, their associated properties, and a Decision Tree classifying whether or not they are a Horror movie. 

In \ref{Chapter3}, we focus on further experimenting with one relationship that was formalized in \cite{Derrac2015}: a ranking of entities on properties. In particular, we use this method of building a representation of entities as a way to convert a vector space into an interpretable representation, for use in an interpretable classifier. The reason that we chose this representation to expand on is because by representing each entity $e$ with a vector $v$ that corresponds to a ranking $r$, the meaning of each dimension is distinct, and we are able to find labels composed of clusters of words for these dimensions. Here, we make the distinction between a property and a word, a property is a natural property of the space that exists in terms of a ranking of entities, and words are the labels we use to describe this property.