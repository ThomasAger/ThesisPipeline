% This thesis template has been put together and edited by Diego Pizzocaro
% It's a collection/edit of files produced by Frank C Langbein for his PhD thesis
% and then edited by Ahmed Alazzawi, Hmood Al Dossari, Manar I. Hosny, H.Z. Al-Dossari 
% and kindly shared by them.
%
% Following, the original GNU license chosen by Frank C Langbein, 
% which is the same license for distributing this package.
%
% beautification.tex -- Beautification of Reverse Engineered Geometric Models
%% PhD Thesis by Frank C. Langbein%
% Maintainer: Frank C. Langbein <frank@langbein.org>
% Version: 1.0
% Copyright (c) 2003 Frank C Langbein.
% Permission is granted to copy, distribute and/or modify this document
% under the terms of the GNU Free Documentation License, Version 1.2 or
% any later version published by the Free Software Foundation; with no
% Invariant Sections, no Front-Cover Texts, and no Back-Cover Texts. A
% copy of the license is included in the section entitled "GNU Free
% Documentation License".
%


\documentclass[a4paper,oneside,onecolumn,openright,12pt]{book}


\makeatletter
% Copy pasted junk

\newcommand{\green}[1]{\textcolor{green}{#1}}


%Graphics
%\usepackage{graphicx}
% Langbein added these - Tom
\usepackage{caption}
\usepackage{footmisc}
\usepackage{paralist}
\usepackage{array}
\usepackage{epsf}
\usepackage{amsmath}
\usepackage{rotating}
\usepackage{listings}
\usepackage{varwidth}
\usepackage{adjustbox}
\usepackage{tabularx}
\usepackage{listings}
\usepackage{amsthm}
\usepackage{longtable}
\usepackage{afterpage}
\usepackage{subfig}
%\usepackage{algorithm}
%\usepackage{rotating}
%\usepackage{amssymb}
\usepackage{multibib}
\usepackage{booktabs}
% Fonts, encoding, etc.
\usepackage{type1cm}
\usepackage[latin1]{inputenc}
\usepackage[british]{babel}
\usepackage[T1]{fontenc}
\usepackage{times}
\usepackage{hyperref}
\usepackage[all]{hypcap}
\usepackage{longtable}
%\usepackage{algorithm}
\usepackage[noend]{algorithmic}
%\usepackage{rotating}
\usepackage{multirow}
\usepackage{lscape}
\usepackage{float}
%\usepackage{subfigure}
% Line spacing
\def\baselinestretch{1.5}
\parindent0cm
\parskip1.5ex\@plus.7ex\@minus.1ex\relax
\lstset{
	basicstyle=\ttfamily,
	columns=fullflexible,
	keepspaces=true,
}

%Acronyms
 \usepackage{acronym}
 %Inline lists
 \usepackage{paralist}

% Page dimensions
\usepackage{vmargin}
\setpapersize{A4}
\setmarginsrb{25mm}{20mm}{25mm}{10mm}{14.5pt}{8mm}{0pt}{11mm}

% Footer and header
\usepackage{afterpage}
\usepackage{fancyhdr}
\pagestyle{fancy}
\fancyhead{}
\fancyhead[LE,RO]{\thepage}
\fancyhead[LO,RE]{\slshape \leftmark}
\fancyfoot{}
\renewcommand{\chaptermark}[1]{}
\renewcommand{\sectionmark}[1]%
             {\markboth{\thesection\ #1}{\thesection\ #1}}
\renewcommand{\subsectionmark}[1]{}
\fancypagestyle{plain}{%
  \fancyhead{}
  \fancyhead[LE,RO]{\thepage}
  \fancyfoot{}
  \renewcommand{\headrulewidth}{.6pt}
}

% Chapter
\def\@makechapterhead#1{%
  \ \\[-35.5pt]\hbox to \textwidth {%
    \hfill {\vbox{\hbox{\rule[5pt]{140pt}{4pt}}%
        \hbox to 140pt {\hfill\huge\bfseries\slshape \@chapapp\space\thechapter\/}}}}%
  \vskip55\p@%
  {\parindent \z@ \raggedright \normalfont%
    \interlinepenalty\@M%
    \Huge \bfseries #1\par\nobreak%
    \vskip 45\p@%
  }}
\def\@chapter[#1]#2{\ifnum \c@secnumdepth >\m@ne
                       \if@mainmatter
                         \refstepcounter{chapter}%
                         \typeout{\@chapapp\space\thechapter.}%
                         \addcontentsline{toc}{chapter}%
                                   {\protect\numberline{\thechapter}#1}%
                       \else
                         \addcontentsline{toc}{chapter}{#1}%
                       \fi
                    \else
                      \addcontentsline{toc}{chapter}{#1}%
                    \fi
                    \chaptermark{#1}%
                    \addtocontents{lof}{\protect\addvspace{10\p@}}%
                    \addtocontents{lot}{\protect\addvspace{10\p@}}%
                    \addtocontents{loa}{\protect\addvspace{10\p@}}%
                    \if@twocolumn
                      \@topnewpage[\@makechapterhead{#2}]%
                    \else
                      \@makechapterhead{#2}%
                      \@afterheading
                    \fi}
\def\@schapter#1{\addcontentsline{toc}{chapter}{#1}%
                 \markboth{#1}{#1}%
                 \addtocontents{lof}{\protect\addvspace{10\p@}}%
                 \addtocontents{lot}{\protect\addvspace{10\p@}}%
                 \addtocontents{loa}{\protect\addvspace{10\p@}}%
                 \if@twocolumn%
                    \@topnewpage[\@makeschapterhead{#1}]%
                 \else%
                    \@makeschapterhead{#1}%
                    \@afterheading%
                 \fi}
\def\@makeschapterhead#1{%
  \ \\[-35.5pt]\hbox to \textwidth {%
    \hfill {\vbox{\hbox{\rule[5pt]{140pt}{0pt}\rule[5pt]{0pt}{4pt}}%
        \hbox to 140pt {\hfill\huge\bfseries\slshape \ \/}}}}%
  \vskip55\p@%
  {\parindent \z@ \raggedright \normalfont%
    \interlinepenalty\@M%
    \Huge \bfseries  #1\par\nobreak%
    \vskip 45\p@%
  }}

% Table of contents
\def\contentsname{Contents}
\renewcommand\tableofcontents{%
    \if@twocolumn%
      \@restonecoltrue\onecolumn%
    \else%
      \@restonecolfalse%
    \fi%
    \chapter*{\contentsname}%
    \@starttoc{toc}%
    \if@restonecol\twocolumn\fi%
    }

% Bibliography
\renewenvironment{thebibliography}[1]
     {\chapter*{\bibname}%
      \list{\@biblabel{\@arabic\c@enumiv}}%
           {\settowidth\labelwidth{\@biblabel{#1}}%
            \leftmargin\labelwidth
            \advance\leftmargin\labelsep
            \@openbib@code
            \usecounter{enumiv}%
            \let\p@enumiv\@empty
            \renewcommand\theenumiv{\@arabic\c@enumiv}}%
      \sloppy
      \clubpenalty4000
      \@clubpenalty \clubpenalty
      \widowpenalty4000%
      \sfcode`\.\@m}
     {\def\@noitemerr
       {\@latex@warning{Empty `thebibliography' environment}}%
      \endlist}

% Floats
\long\def\@makecaption#1#2{%
  \vskip\abovecaptionskip
  \sbox\@tempboxa{\textbf{#1: #2}}%
  \ifdim \wd\@tempboxa >\hsize
    \textbf{#1: #2.}\par
  \else
    \global \@minipagefalse
    \hb@xt@\hsize{\hfil\box\@tempboxa\hfil}%
  \fi
  \vskip\belowcaptionskip}
\renewcommand{\topfraction}{0.9}
\renewcommand{\textfraction}{0.1}
\renewcommand{\floatpagefraction}{0.9}

% Tables
\usepackage{dcolumn}
\usepackage{hhline}

% Graphics
\usepackage[dvips]{graphicx}
\usepackage{rotating}
\usepackage{psfrag}
\usepackage{epic}
\usepackage{tabularx}
\usepackage{eepic}

% Algorithm environment
\newcounter{algorithm}[chapter]
\renewcommand{\thealgorithm}{\thechapter.\@arabic\c@algorithm}
\def\fps@algorithm{t}
\def\ftype@algorithm{1}
\def\ext@algorithm{loa}
\def\fnum@algorithm{Algorithm~\thealgorithm}
\newenvironment{algorithm}{\@float{algorithm}}{\end@float}
\newenvironment{algorithm*}{\@dblfloat{algorithm}}{\end@dblfloat}
\newenvironment{alevel}%
   {\begin{list}{}{%
      \setlength{\topsep}{0pt}%
      \setlength{\parskip}{0pt}%
      \setlength{\partopsep}{0pt}%
      \setlength{\parsep}{0pt}%
      \setlength{\itemsep}{0pt}}}%
   {\end{list}}
\newcommand\listalgorithmsname{List of Algorithms}
\newcommand\listofalgorithms{%
    \if@twocolumn
      \@restonecoltrue\onecolumn
    \else
      \@restonecolfalse
    \fi
    \chapter*{\listalgorithmsname}%
      \@mkboth{\listalgorithmsname}{\listalgorithmsname}%
    \@starttoc{loa}%
    \if@restonecol\twocolumn\fi
    }
\newcommand*\l@algorithm{\@dottedtocline{1}{1.5em}{2.3em}}

% List of figures and tables
\renewcommand\listoffigures{%
    \if@twocolumn
      \@restonecoltrue\onecolumn
    \else
      \@restonecolfalse
    \fi
    \chapter*{\listfigurename}%
      \@mkboth{\listfigurename}{\listfigurename}%
    \@starttoc{lof}%
    \if@restonecol\twocolumn\fi
    }
\renewcommand\listoftables{%
    \if@twocolumn
      \@restonecoltrue\onecolumn
    \else
      \@restonecolfalse
    \fi
    \chapter*{\listtablename}%
      \@mkboth{\listtablename}{\listtablename}%
    \@starttoc{lot}%
    \if@restonecol\twocolumn\fi
    }

% Math symbols, fonts, etc.
\usepackage{amsmath}
\usepackage{amsfonts}
\usepackage{amssymb}

\newcommand{\N}{\mathbb{N}}
\newcommand{\Z}{\mathbb{Z}}
\newcommand{\Q}{\mathbb{Q}}
\newcommand{\R}{\mathbb{R}}
\newcommand{\C}{\mathbb{C}}
\renewcommand{\S}{\mathbb{S}}
\renewcommand{\P}{\mathbb{P}}
\newcommand{\E}{\mathbb{E}}

\newcommand{\Cf}{\mathfrak{C}}
\newcommand{\Pf}{\mathfrak{P}}

\DeclareMathOperator{\sign}{sign}
\DeclareMathOperator{\avg}{avg}
\DeclareMathOperator{\floor}{floor}
\DeclareMathOperator{\ceil}{ceil}
\DeclareMathOperator{\round}{round}

\providecommand{\abs}[1]{\lvert#1\rvert}
\providecommand{\absd}[1]{\left\lvert#1\right\rvert}
\providecommand{\card}[1]{\lvert#1\rvert}
\providecommand{\norm}[1]{\lVert#1\rVert}

% URLs
\usepackage{url}
%% Define a new 'leo' style for the package that will use a smaller font.
\makeatletter
\def\url@leostyle{%
  \@ifundefined{selectfont}{\def\UrlFont{\sf}}{\def\UrlFont{\footnotesize\ttfamily}}}
\makeatother
%% Now actually use the newly defined style.
\urlstyle{leo}



\makeatother


\begin{document}


\frontmatter


\begin{titlepage}

\begin{center}
\vspace*{3ex}
\textbf{\Huge Disentangling \new{L}ow-\new{D}imensional \new{V}ector \new{S}pace \new{R}epresentations of \new{T}ext \new{D}ocuments}\\[2ex]
%\textbf{\Huge Title line 2}\\[12ex]
\textbf{\large A thesis submitted in partial fulfilment}\\[1ex]
\textbf{\large of the requirement for the degree of Doctor of
  Philosophy}\\[16ex]
\textbf{\LARGE Thomas J. Ager}\\
\vfill
\textbf{\LARGE January 2020}\\
\vfill
\textbf{\LARGE Cardiff University}\\[1ex]
\textbf{\LARGE School of Computer Science \& Informatics}\\[4ex]
\end{center}

\end{titlepage}
\newpage\thispagestyle{empty}\cleardoublepage


\thispagestyle{plain}

\textbf{\large Declaration}

This thesis is the result of my own independent work, except where otherwise stated,
and the views expressed are my own. Other sources are acknowledged by explicit references.
The thesis has not been edited by a third party beyond what is permitted by
Cardiff University's Use of Third Party Editors by Research Degree Students Procedure.\\[2ex]
Signed \dotfill \ (candidate) \hspace*{10em}\\[1ex]
Date\ \ \ \ \ \dotfill \hspace*{18em}

\vfill

\vfill

\textbf{\large Statement 1}

This thesis is being submitted in partial fulfilment of the requirements for the degree of PhD.\\[2ex]
Signed \dotfill \ (candidate) \hspace*{10em}\\[1ex]
Date\ \ \ \ \ \dotfill \hspace*{18em}

\textbf{\large Statement 2}

This thesis is the result of my own independent work/investigation,
except where otherwise stated. Other sources are acknowledged by
explicit references.\\[2ex]
Signed \dotfill \ (candidate) \hspace*{10em}\\[1ex]
Date\ \ \ \ \ \dotfill \hspace*{18em}

\vfill

\textbf{\large Statement 3}

I hereby give consent for my thesis, if accepted, to be available for photocopying and for inter-library loan,
 and for the title and summary to be made available to outside organisations.\\[2ex]
Signed \dotfill \ (candidate) \hspace*{10em}\\[1ex]
Date\ \ \ \ \ \dotfill \hspace*{18em}

\vfill

\vspace*{3ex}

\textbf{\large Word Count}  \ \ \ \ \dotfill \hspace*{18em}

(Excluding Abstract, acknowledgements, declarations, contents pages, appendices, tables,
diagrams and figures, references, bibliography, footnotes and endnotes)



\ \vspace*{1.11cm}
\markboth{Dedication}{}
\begin{flushright}
\textbf{\large To my parents.}\\
\textbf{\large for their invaluable and unwavering support.}
\end{flushright}
\newpage
\markboth{}{}
\cleardoublepage


\chapter*{Abstract}

%%%%%%%%%%%%%%%%% OLD WORK

%NESY 
%Vector space embeddings can be used as a tool for learning semantic relationships from unstructured text documents. Among others, earlier work has shown how in a vector space of entities (e.g. different movies) fine-grained semantic relationships can be identified with directions (e.g. more violent than).

%IJCAI 
%In many applications, it is important for classification models to be interpretable. Unfortunately, however, prominent While dimensionality reduction methods could be used in principle, the resulting dimensions would typically not be interpretable. To address this shortcoming, we propose a method to learn interpretable features from a given low-dimensional vector space representation.

%CONLL
%In this paper we consider semantic spaces consisting of objects from some particular domain (e.g.\ IMDB movie reviews). and 

%%%%%%%%%%%%%%%%% OLD WORK


% General problem introduction, why it matters, what past work did, what the results will achieve

% Most vector spaces are entangled


% By the way, if you use the word disentangled in the title of your thesis, you should explain this term in the abstract.

% Broader context

%This thesis is concerned with vector space representations of  documents in a given domain (e.g. movies in a domain of movie reviews). Such representations are often learned from text documents that describe these entities. 
In contrast to traditional document representations such as bags-of-words, vector space representations that are currently most popular tend to be lower-dimensional. This has important advantages, e.g. making the representation of a given document less dependent on the exact words that are used in its description. However, this comes at an important cost, namely that the representation is  "entangled", meaning that the features of the representation are not individually meaningful or interpretable. The main aim of this thesis is to address this problem  by disentangling   vector spaces into representations that are composed of individual, important and meaningful features in the given domain. In a domain of IMDB movie reviews, where each document is a review, the disentangled feature representation would be separated into features that describe how "Scary", "Romantic", ..., "Comedic",  the  movie is. This thesis builds on an initial approach introduced by  Derrac \cite{Derrac2015}, where it was  observed that low-dimensional vector space representations of documents  often model important features of the domain as directions. The method begins by obtaining direction vectors for all terms  using a linear classifier to find a hyper-plane that separates entities that have a term and do not have a term. Then, from each hyper-plane the orthogonal vector is taken as a direction that goes from documents that least have the word (as they are furthest from the hyper-plane on the negative side) to documents that most have it (as they are furthest from the hyper-plane on the positive side). The importance of these terms is determined by how well the linear classifier performs on a standard classification metric, e.g. accuracy, which approximates how linearly separable the entities are that contain the term in the vector space. The most important terms are taken, and documents are ranked on these directions by calculating the dot product between the orthogonal vector to the hyper-plane and each document vectors. This results in  a value is obtained that corresponds to how  far up documents are on the direction, enabling    documents to be ranked on how much they 'have' a particular meaningful feature,  e.g.\ movies can be ranked on how "Scary" they are. %In terms of application, a representation like this can be used in entity oriented search engines, recommendation systems, among others, or perhaps even be used as input to a simple classifier as an alternative to a hard to decipher  neural networks. 

% Contribution 1

The work by Derrac \cite{Derrac2015} obtained semantic features from Multi-Dimensional Scaling (MDS) document embeddings and validated their work by classifying entities using a rule-based classifier (FOIL), resulting in rules of the form "\textbf{IF} \textit{x} is more scary than most horror films \textbf{THEN} \textit{x} is a horror film.". The first main contribution of this thesis is a deeper investigation and improvement of this method, in particular by  experimentally validating that the  features derived from the initial embeddings are clear and semantic across a variety of unsupervised document embedding models and domains. Variants of the method are introduced that are shown to improve  performance, and the results are  quantitatively tested using low-depth decision trees.

% Contribution 2

Neural network architectures have advanced to the state-of-the-art  in many tasks, following the idea that their hidden layers can be viewed as document embeddings, the second main contribution of the work is in deriving semantic features from the hidden layers of neural networks. In particular, a qualitative analysis of two kinds of neural networks is conducted, feed-forward networks and stacked auto-encoders. Auto-encoders are  stacked by using their hidden-layer as the input  to another auto-encoder. Characteristics of  learned feed-forward networks are identified, in particular that they can learn higher-quality representations of  features that are relevant to the considered classification problem. In the case of auto-encoders, it is found  that as  auto-encoders are  stacked, increasingly abstract features can be derived from them. For example, in the initial auto-encoder layers, features like "Horror" and "Comedy" can be separated well by the linear classifier, meanwhile in the latter layers features like "society" and "relationships" are easy to separate. After identifying directions that model important features of documents in each stacked auto-encoder, symbolic rules are induced that relate specific features to more general ones. These rules are interesting, as they can be used to produce a domain theory of the domain where abstract terms are explained in terms of specific ones, for example:

\begin{align}\label{rule1}\texttt{IF Emotions  AND  Journey THEN Adventure}\end{align} 

% Contribution 3

 In addition to introducing variants of the method in \cite{Derrac2015}, this thesis also introduces an additional post-processing step for improving disentangled feature representations by fine-tuning the original embedding to prioritize faithfully modelled directions. Methods for learning document embeddings  are mostly aimed at modelling similarity. It is found  that there is an inherent trade-off between capturing similarity and faithfully modelling features as directions. Following this observation, a simple method to fine-tune  document embeddings is proposed, with the aim of improving the quality of the feature directions obtained from them. Crucially, this  method is also fully unsupervised, requiring only a bag-of-words representation of the objects as input.  In particular,  clusters of terms are identified that refer to salient properties of the considered domain, and a simple neural network model is used to learn a new representation in which each of these properties is more faithfully modelled as a direction. It is found that in most cases this method improves the ranking of entities, and results in increased performance when disentangled feature representations are used as input to classifiers. %As in Chapter \ref {ch3}, the usefulness of this approach is validated using low-depth decision trees, where it is found that in most cases performance improves.


%The objective of this thesis is to improve how such semantically meaningful feature directions can be learned from a given  document embedding.  This thesis concerns itself with application of this methodology to a variety of document embeddings. These semantic features are beneficial for a variety of purposes,  One important examples is the fact that they provide a useful inductive bias.  they may provide insight into what the neural network is doing correctly or incorrectly in its learning process. 

 %The how semantic these features are is evaluated using a document classification task that classifies decision trees with a limited depth. This limitation verifies that the feature representation has captured features that correspond to domain knowledge, as if the classifier can perform well with a low amount of features those features must be semantic. 



% Paragraph focusing on how methods can be useful for your applications

Overall, this thesis quantitatively and qualitatively validates  that disentangled feature representations of meaningful features can be derived from low-dimensional vector spaces of documents, across a variety of domains and document embedding models.  New variants of the method introduced by Derrac \cite{Derrac2015} are evaluated and found to be effective. Neural networks are investigated, and in the case of  stacked auto-encoders,  features become more abstract as auto-encoders are stacked, and a method to obtain a domain theory that links specific properties to more general ones is introduced. In the case of feed-forward networks, a variety of interesting characteristics are observed, in particular that feed-forward networks make features that are more relevant to the task more separable. Finally, a new method is introduced to fine-tune existing document embeddings such that their features are more meaningful. 



%This won't be clear, as you haven't actually explained that the semantic features allow you to derive a more interpretable, feature-based representation of the entities.




%This sentence is not clear. I think you are talking about existing approaches to learn disentangled neural network representations (like InfoGAN), but that won't be clear to the reader + probably this is too much detail for the abstract anyway. 

% features like these are usually obtained by using a specialized learning procedure, e.g.\ by enforcing sparsity constraints or orthogonality between features, and learned from the original text data.

 

%Can this method be applied to a variety of representations?
%Can this method be applied to more domains?
%Does this method result in a representation of clear semantic features?
%Can these semantic features be used to form a useful alternative representation?
%Are the features meaningful qualitatively in a variety of domains and for a variety of representations?

%The work in Chapter \ref{ch3} addresses these questions, and introduces variants to their method that result in  better performance on tasks.

%There are a variety of approaches to obtaining vector space representations where features are clear and semantic. 

%simple models such as (small) decision trees cannot readily be applied to text documents due to the size of the underlying feature space. 
 

  





% Disentanglement and why it matters

% Existing disentanglement methods and why they dont fit the gap

% What we actually did to solve this problem 1st chapter

% What the results were 1st chapter

% What we did 2nd

% Results 2nd

% What we did 3rd

% Results 3rd

% Overall conclusions

%Vector space representations of documents can be re-organized  such that their dimensions correspond to clear semantic features. Furthermore, these disentangled representations can be used to provide a useful inductive bias to classifiers, and as the features are disentangled, they can be used to qualitatively investigate the hidden layers of neural networks to gain valuable insights into how they represent documents. 
%\def\baselinestretch{1}\normalfont

\def\baselinestretch{1.5}\normalfont


\chapter*{Acknowledgements}

It has been a long journey from the start of this PhD to the end of it, and I owe almost all of my success to my supervisor Steven Schockaert, and my unofficial co-supervisor Ondrej Kuzelka. I would like to thank them especially for all their guidance and their many ideas about the work undertaken in the thesis.

I would also like to thank my colleagues at Cardiff University, who provided a community to engage with when I needed an audience for my presentations, or to discuss my thoughts about the work. It is certainly in large part thanks to them that I was able to be where I am today.

My deepest thanks go to my parents, who were able to support me through the final years of my PhD. Finally I thank God, who guided me when I needed it most.

\tableofcontents

\newpage
\chapter*{List of Publications}

The work introduced in this thesis is based on the following publications.

\begin{itemize} 

\item Ager, T., Kuzelka, O., & Schockaert, S. (2018). Modelling salient features as directions in fine-tuned semantic spaces. CoNLL 2018 - 22nd Conference on Computational Natural Language Learning, Proceedings, 530-540. https://doi.org/10.18653/v1/k18-1051

\item Ager, T., Kuzelka, O., & Schockaert, S. (2016). Inducing symbolic rules from entity embeddings using auto-encoders. CEUR Workshop Proceedings, 1768.

\end{itemize}

\listoffigures

\listoftables

%\listofalgorithms

%

\chapter*{List of Publications} 
\begin{acronym}
\acro{ML}{Machine Learning}
\acro{NLP}{Natural Language Processing}
\acro{NDCG}{Normalized Discounted Cumulative Gain}
  \begin{enumerate}
	\item \bibentry{Ager2012b}
	\item \bibentry{Ager2016}
\end{enumerate}
% to use in the thesis, use: \ac{GPS}
% more goes here.
\end{acronym}



\mainmatter
%begin each chapter chapter1.tex by \Chapter{chapter1}, for example, then the sections and so on...
\chapter{Introduction}

% Introduction to the internet, data, basic text representations
% Introduction to machine learning, benefits of data for machine learning, machine learning representations
% Problems with machine learning, interpretability, lack of interpretability in machine learningg
% We introduce a series of methods for transforming uninterpretable machine learning representations into interpretable ones
% Finding directions in vector spaces and using those to produce interpretable representations
% Fine-tuning these directions to get a better result
% Interpreting and investigating neural networks with these directions



\section{Motivation}
%What is text? How is it motivating?

%What are the desiradata of a good representation?
% Unsupervised
With the rise of services on the web that enable large-scale user-generation of text data, e.g. Social Media sites (Facebook, Twitter), Review sites (IMDB, Rotten Tomatoes, Amazon) and content-aggregation sites (Reddit, Tumblr), the internet has become largely populated by text posts that are related to some specific, niche topic within a domain. For example, a review on Amazon for a product is specially tailored text for that product within the domain of Amazon reviews. Taken from a closer lens, we could even argue that each review-type has its own domain, e.g. Product reviews, Food reviews, Movie reviews. However, the text posts themselves are largely unstructured semantically. Humans can have an intuitive understanding of the semantics that are present in unstructured text, but machines do not. 

One task of Natural Language Processing is to obtain this semantic understanding from text by obtaining a machine-readable representation that contains domain knowledge. A basic approach to obtain a representation of this text is to represent entities (e.g. reviews, text-posts) by the frequency of their words, see \ref{Bag-of-words-example}.

\begin{figure}[t]
	\includegraphics[width=\textwidth]{images/bowbowbow.png}
	\centering
	\caption{Bag-of-words  }\label{Bag-of-words-example}
\end{figure}


 Below, we show a review with its associated properties labelled.

\begin{figure}[t]
	\includegraphics[width=\textwidth]{images/leg_length.png}
	\centering
	\caption{Example properties  }\label{IntroDecisionTree}
\end{figure}

% Granular
We can understand these properties to have a degree to which they apply, for example the size of the clothing might be "XXL", "XL", "L", "M" or "S", or the quality may be "Very good", "Good", "Ok", "Bad" or "Very bad". For the former, we may rely on the metadata available from the site itself, but for the latter the way to obtain this information is less clear. Although we may infer that the rating has some indication of these properties, it does not describe the properties or the degree to which the review refers to them. This kind of information is valuable for making sense of the world of unstructured text, and has broad applications, e.g. The most immediate example is perhaps that they allow for a natural way to implement critique-based recommendation systems, where users can specify how their desired result should relate to a given set of suggestions \cite{viappiani2006preference}. For instance, \cite{Vig:2012:TGE:2362394.2362395} propose a movie recommendation system in which the user can specify that they want to see suggestions for movies that are ``similar to this one, but scarier''. If the property of being scary is adequately modelled as a direction in a semantic space of movies, such critiques can be addressed in a straightforward way. Similarly, in \cite{kovashka2012whittlesearch} a system was developed that can find ``shoes like these but shinier'', based on a semantic space representation that was derived from visual features. Semantic search systems can use such directions to interpret queries involving gradual and possibly ill-defined features, such as ``\emph{popular} holiday destinations in Europe'' \cite{DBLP:conf/sigir/JameelBS17}. While features such as popularity are typically not encoded in traditional knowledge bases, they can often be represented as semantic space directions.  %Copied from CONLL

\subsection{Directions}\label{intro:directions}


However, manually labelling these properties and the degrees to which entities (e.g. reviews, text-posts) have them is extremely time-consuming. 

A potentially ideal system would be as follows: We collect large amounts of unstructured text data, separated into domains, and obtain the properties of each domain from this data, and rank entities on the degree to which they have these properties. In this way, properties would be understood on a scale built from the domain directly, so that each domain has its own meanings for words according to their own idiosyncrasies. As the process does not require any manual labelling the quality of these properties could be improved simply by obtaining more data. Further, as we are learning from unstructured data, not only would this allow us to understand the data in terms of what we know, but it would also introduce us to new ideas that we may not have previously understood. This kind of representation also has value in application to Machine Learning tasks. If we can separate the semantics of the space linearly into properties, we are able to learn simple linear classifiers that perform well. 

Simple linear classifiers built from a representation composed of rankings on properties have an additional benefit of being more understandable.


% Natural clustering
% Semantically distinct
% Interpretable
% Curse of dimensionality
% Generalizability ("shared factors across many tasks" \cite{Bengio2012})

%What is machine learning? What are its advantages? How is it motivating?

%What are the problems with machine learning? How is it motivating?


%What is domain-specific? What is domain knowledge? %On the web there is a large volume of raw text data, e.g. Reviews of products, movies, anime, books, music, social posts by individuals, self-descriptive text about a website or product, and so on. These can be categorized into domains; each domain has its own quirks, knowledge, and method of being brought about. Although a movie review may sound similar to a book review, they typically differ hugely in the distribution of words used.

%

\section{Interpretability}

%What is interpretability? How is the value of interpretability measured in the real world?

%How can we meet the needs of the real world?  Is it transparancy, the system having easy to understnad components, etc... what  are the different views on what an interpretabile system is?

%What specific interpretability task are we trying to solve? How do we define interpretability? Why is it valuable, where is it used? What was the hypothesis/research question?
%%What are distributional models?
Most successful approaches in recent times, like vector-spaces, word-vectors, and others, rely on the distributional model of semantics. This model relies on encoding unstructured text e.g. of a movie review, as a vector, where each dimension corresponds to how frequent each word is, we are able to calculate how similar the entities are, e.g. we know that if two movies have a similar distribution of words in their reviews, like frequent use of the word 'scary', or 'horror', then they would have a higher similarity value. These models, also known as 'semantic spaces' encode this similarity information spatially.

Semantic relationships can be obtained from semantic spaces. 

applications/need for good interpretability:
\begin{itemize}
	\item Safety
	\item Troubleshooting, bug fixing, model improvement
	\item Knowledge learning
	\item EU's "Right to explanation"
	\item Discrimination
\end{itemize}

%What is a conceptual space? What are entities?  What are properties? What is commonsense reasoning?
properties of an interpretable classifier:
\begin{itemize}
	\item Complexity: 'the magic number is seven plus or minus two' \cite{Saaty2003} also has many positive effects for its users, like lower response times \cite{Narayanan2018, Huysmans2011}, better question answering and confidence for logical problem questions \cite{Huysmans2011} and higher satisfaction \cite{Narayanan2018}.
	\item Transparancy: 
	\item Explainability: 
	\item Generalizability:
\end{itemize}


%X%X%What is a symbolic approach?  %One approach to making sense of these domains is to produce rules from expert knowledge. An expert in movies would tell you that if the review talks about it being a "cannibal horror film", we can understand that it is likely a scary movie and is related to the original 'Cannibal Holocaust' movie. Encoding this kind of knowledge is difficult, time-consuming, and hard to automate reliably.

Properties, entities, the benefits and application of a representation formed of these

Basic introduction to directions, explanation of the utility and application of our approach
\section{Thesis Overview / Contributions}

%What were our objectives starting out? 
%What are our intentions with how the work in the thesis will be used?
%What are our contributions?
%%What are our aims for this chapter? What do we overall want to do? (Already kind-of said in Chapter 1, but worth repeating I guess in some form)
%In \ref{Chapter3}, we introduce a pipeline that starts with unstructured text, and ends with an interpretable representation of entities represented by properties labelled by clusters of words. Further, we demonstrate the applicability of these representations in a simple Decision Tree that uses just a few of these properties to classify entities. In Figure \ref{ExamplesWithTree}, we show some example movie entities, their associated properties, and a Decision Tree classifying whether or not they are a Horror movie. 

In \ref{Chapter3}, we focus on further experimenting with one relationship that was formalized in \cite{Derrac2015}: a ranking of entities on properties. In particular, we use this method of building a representation of entities as a way to convert a vector space into an interpretable representation, for use in an interpretable classifier. The reason that we chose this representation to expand on is because by representing each entity $e$ with a vector $v$ that corresponds to a ranking $r$, the meaning of each dimension is distinct, and we are able to find labels composed of clusters of words for these dimensions. Here, we make the distinction between a property and a word, a property is a natural property of the space that exists in terms of a ranking of entities, and words are the labels we use to describe this property.
\chapter{Background}
\section{Representation Based Approaches to Interpretability}
\subsection{Methods to represent text}
\subsection{Interpretable Methods to represent text}
\subsection{Conceptual spaces}
\subsection{Word Vectors}
\section{Directions and Semantic Relations}
\subsection{Obtaining directions/semantic relations and formalizing them from representations}

1. Methods to represent text
a. bow
b. tfidf
c. ppmi
2. Vector spaces, word vectors
a. MDS
b. PCA
c. AWV (Depending
3. Interpretability in vector spaces
a. NNSE
b. compositional
c. 2007 paper as wikipedia similarities
d. Topic models
e. Infogan, etc

\chapter{Datasets and Semantic Spaces}\label{Chapter3}


\section{Introduction}\label{chapter3:Introduction}

Domain-specific semantic spaces like the ones in this work are used, for instance, to represent items in recommender systems \cite{Vasile:2016:MPE:2959100.2959160,liang2016factorization,van2016learning}, to represent entities in semantic search engines \cite{DBLP:conf/sigir/JameelBS17,van2017structural}, or to represent examples in classification tasks \cite{DBLP:conf/iccv/DemirelCI17}. Methods like Multi-Dimensional Scaling and Principal Component Analysis that produce these semantic spaces are in widespread use for document representation and data analysis, and are typically built from word frequency statistics (See \ref{ch2:reps}). However, a wide variety of training methods have been used to obtain semantic spaces. Distributional word-vectors that rely on learning via word-context have had great success as a component of neural learning systems achieving state-of-the-art results on key natural language processing tasks like Language Modelling \cite{Gong2018}, Constituency Parsing \cite{Edunov2018}, and Part-Of-Speech Tagging \cite{Edunov2018}, and have also been applied for document representations \cite{Le2014, Lau2014}. Ideally, we would like to retain the benefits of these learning methods while also making them interpretable. The methodology in this work is a post-processing step that can be applied to representations regardless of how they have been learned, by leveraging the spatial relationships in the representation.

The interpretable representation that is obtained by this method is composed of in terms of salient features, where each of these features is described using a cluster of natural language terms. This is somewhat similar to Latent Dirichlet Allocation (LDA), which learns a representation of text documents as multinomial distributions over latent topics, where each of these topics corresponds to a multinomial distribution over words \cite{Blei03latentdirichlet}.  Topics tend to correspond to salient features, and are typically labelled with the most probable words according to the corresponding distribution. On the other hand, our work leverages clustering methods to obtain the feature labels. Broadly speaking, in the context of document classification, the main advantage of topic models is that their topics tend to be easily interpretable, while Vector Space Models tend to be more flexible in the kind of meta-data that can be exploited e.g.\ they allow us to use neural representation learning methods to obtain these spaces. The approach proposed in this Chapter aims to combine the best of both worlds, by providing a way to derive interpretable representations from Vector Space Models.  Many extensions of LDA have been proposed to incorporate additional information as well, e.g.\ aiming to avoid the need to manually specify the number of topics \cite{teh2005sharing}, modelling correlations between topics \cite{Blei2006}, or by incorporating meta-data such as authors or time stamps \cite{rosen2004author,wang2006topics}. Nonetheless, such techniques for extending LDA offer less flexibility than neural network models, e.g.\ for exploiting numerical attributes or visual features. For comparison, in our experiments the standard topic model algorithm Latent Dirchlet Allocation (LDA) is used as a baseline to  compare to the new methodology that transforms standard Vector Space Model representations. %Topic Models are also entirely probabilistic, while our method relies entirely on the spatial relationships present in the vector space model.




%How is the chapter going to play out? Whats going to happen?
%As our work performs well even at lower-depth trees, this gives potential users more flexibility in how they want to present the information, e.g. to a potential client. Compared to Bag-Of-Words, which loses its representation capabilities the lower the depth.

This chapter continues as follows: First the method is described, making explicit the variations from to the original method in \cite {Derrac2015}. This is followed by a qualitative  and quantitative analysis, finishing with a conclusion on the  benefits and limitations of this approach.



%\section{Related Work}
%Sparse word vectors
%Adapted to composition \cite{Fyshe2015}
%\subsection{Semantic Relations \& Their Applications}

%Our method uses the relationships inherent in a Vector Space Model. Other work has formalized the relationships found in Vector Space Models, for example in word-vectors, linear analogies (see Section \ref{WordVectors, Ethayarajh2018}, were found where the vector between  %Copy pasted from this 
%[ENTIRE SECTION COPY PASTED FROM PREVIOUS PAPER]
%\textbf{Linear Classifiers}
%Decision Trees, linear SVM's, logistic regression, decision tables, IF Then rules.

%What are the available options for interpretable linear classification?

%How have each of these methods been measured or validated in the literature in regards to interpretability? How about application to real world situations?

%\textbf{Non linear classifiers}
%What non linear classifiers networks are interpretable? How have they done it? How have they measured it? How does it compare to a linear method?

%\textit {Neural networks}Approximating w/linear model, Interpretable nodes/weights

%\textit {Other Stuff}

%\subsection{Interpretable Representations}


%There are two ways in which topic models can be used for document classification. First, a supervised topic model can be used, in which the underlying graphical model is explicitly extended with a variable that represents the class label \cite{Blei2010}. Second, the parameters of the multinomial distribution corresponding to a given document can be used as a feature vector for a standard classifier, such as a Support Vector Machine (SVM) or Decision Tree. .




% One of the more popular models for text representation that labels features in a similar way to our method are Topic Models.


\section{Datasets}\label{ch3:datasets}

The experiments are using five different domains. To begin, the properties of these domains are explained to try to give an insight into the kind of text stored within them. This is to better inform analysis of our qualitative results. Examples are shown in three domains in Table \ref{ch3:TextExamples}.

\begin{table}[] 
	\scriptsize
	\begin{tabular}{lp{6.75cm}p{6.75cm}}
		Data Type  & Unprocessed                                                                                                                                                                                                                                                                                                                                                                               & Processed       \\
		\midrule[\heavyrulewidth]
		Newsgroups & morgan and guzman will have era's 1 run higher than last year, and  the cubs will be idiots and not pitch harkey as much as hibbard.  castillo won't be good (i think he's a stud pitcher)                                                                                                                                                                                                & morgan guzman eras run higher last year cubs idiots pitch harkey much hibbard castillo wont good think hes stud pitcher                            \\
		Sentiment  & All the world's a stage and its people actors in it--or something like that. Who the hell said that theatre stopped at the orchestra pit--or even at the theatre door? 
		Why is not the audience participants in the theatrical experience, including the story itself?<br /><br />This film was a grand experiment that said: "Hey! the story is you and it 
		needs more than your attention, it needs your active participation"". ""Sometimes we bring the story to you, sometimes you have to go to the story.""<br /><br />Alas no one listened, 
		but that does not mean it should not have been said." & worlds stage people actors something like hell said theatre stopped orchestra pit even theatre door audience participants
		theatrical experience including  story film grand experiment said hey story needs attention needs active participation sometimes bring story sometimes go story alas one listened mean
		said \\
		Reuters    & U.K. MONEY MARKET SHORTAGE FORECAST REVISED DOWN The Bank of England said it had revised its forecast of the shortage in the money market down to 450 mln stg before taking account of its morning operations. At noon the bank had estimated the shortfall at 500 mln stg.                                                                                                               & uk money market shortage forecast revised bank england said revised forecast shortage money market 450 mln stg taking account morning operations noon bank estimated shortfall 500 mln stg     \\
		
	\end{tabular}
	\caption{Text examples from the first three domains}\label{ch3:TextExamples}
\end{table}

\textbf{20 Newsgroups\footnote{http://qwone.com/~jason/20Newsgroups/}} Obtained from scikit-learn. \footnote{https://scikit-learn.org/0.19/modules/generated/sklearn.datasets.fetch\textunderscore20newsgroups.html\#sklearn.datasets.fetch\textunderscore20newsgroups} Where documents are discussions from one of twenty different groups, specifically Atheism, Computer Graphics, Microsoft Windows, IBM PC Hardware, Mac Hardware, X-Window (GUI Software), Automobiles, Motorcycles, Baseball, Hockey, Cryptography, Electronics, Medicine, Space, Christianity, Guns, The Middle East, General Politics and General Religion. These also act as the classes for the dataset. Originally containing 18,846 documents, in this work it is preprocessed using sklearn to remove headers, footers and quotes. Then, empty and duplicate documents are removed, resulting in 18302 documents. The vocabulary size (unique words) is 141,321. The data is not shuffled. After filtering out terms that did not occur in at least two documents, ending up with a vocabulary of size 51,064. The number of positive instances averaged across all classes is 942, around 5\%.

\textbf{IMDB Sentiment} Obtained from Keras \footnote{https://keras.io/datasets/} Where documents are IMDB movie reviews, containing 50,000 documents with a vocabulary size of 78588. After removing terms that did not occur in at least two documents, ending up with a vocabulary of size 55384. This is a smaller change than the newsgroups, which began with a larger vocabulary than sentiment, but ended vocabularies about the same. This means that newsgroups contained many terms that were not relevant to a majority of the documents, likely because the 20 different newsgroups spread across so many topics. The corpus is split half and half between positive and negative reviews, with the task being to identify the sentiment of the review, so the number of positive instances in the classes is 25,000.

\textbf{Reuters-21578, Distribution 1.0} Obtained from NLTK\footnote{https://www.nltk.org/book/ch02.html}. Documents from the Reuters financial newswire service in 1987,  originally containing 10788 documents. After removing empty and duplicate documents, ending up with 10655 documents. It originally contained 90 classes, but as they were extremely unbalanced all classes that did not have at least 100 positive instances were removed, resulting in 21 classes. These classes are Trade, Grain, Natural Gas (nat-gas), Crude Oil (crude), Sugar, Corn, Vegetable Oil (veg-oil), Ship, Coffee, Wheat, Gold, Acquisitions (acq), Interest, Money/Foreign Exchange (money-fx), Soybean, Oilseed, Earnings and Earnings Forecasts (earn), BOP, Gross National Product (gnp), Dollar (dlr) and Money-Supply.   The original vocabulary size is 51,0001, and after removing all words that do not occur in at least two documents, the vocabulary size is 22542. The number of positive instances averaged across all classes is 541, around 5\%. 

\textbf{Placetypes} Taken from work by Derrac \cite{Derrac2015}. Documents are composed of concatenated flickr tags, where each document, named after a flickr tag, is composed of all flickr tags where that tag occurred. A minimum of 1,000 photos for each tag was a requirement, and the tags selected were taken from three different taxonomies (Geonames, Foursquare and the site category for the common-sense knowledge base OpenCYC).  It originally has a vocabulary size of 746,527 and 1383 documents. This is a very large vocabulary size to document ratio. The end vocabulary for this space was 100,000, which is used as a hard limit. This is roughly equivalent to removing all documents that would not be in at least 6 documents. As most classes in this domain are extremely sparse (less than 100 positive instances) no classes are deleted. There are three tasks, generated from three different place type taxonomies. The Foursquare taxonomy, classifying the 9 top-level categories from Foursquare in September 2013, Arts and Entertainment, College and University, Food, Professional and Other Places, Nightlife Spot, Parks And Outdoors, Shops and Service, Travel and Transport and Residence. the GeoNames taxonomy where 7 of 9 categories were used, Stream/Lake, Parks/Area, Road/Railroad, Spot/Building/Farm, Mountain/Hill/Rock, Undersea, and Forest/Heath. The OpenCYC Taxonomy, where 93 categories were used by Derrac, but it was only possible to match 25 of those classes to the representations. As 8 of these remaining classes had a low number of positive occurrences, OpenCYC classes are removed that do not have positive instances for at least 30 documents, leaving us with 17, Aqueduct, Border, Building, Dam, Facility, Foreground, Historical Site, Holy Site, Landmark, Medical Facility, Medical School, Military Place, Monsoon Forest, National Monument, Outdoor Location, Rock Formation, and Room. Naturally as these tasks were derived from taxonomies they are multi-label.

\textbf{Movies} Taken from work by Derrac \cite{Derrac2015}. A dataset where each document is a movie represented by all of its reviews concatenated across a number of sources (Rotten Tomatoes, IMDB, Amazon Reviews). It starts off with a vocabulary size of 551,080 and a document size of 15,000. However, after investigating the data made available by the authors, it was found that there were a number of duplicate documents. After removing these duplicate documents, there are 13978 documents. In the same way as the place-types, the vocabulary is limited at size 100,000. Three tasks are used to evaluate, 23 movie genres, specifically Action, Adventure, Animation, Biography, Comedy, Crime, Documentary, Drama, Family, Fantasy, Film-Noir, History, Horror, Music, Musical, Mystery, Romance, Sci-Fi, Short, Sport, Thriller, War, Western. 100 of the most common IMDB plot keywords (See Appendix \ref{app:ClassNames}) and Age Ratings from the UK and US, USA-G, UK-12-12A, UK-15, UK-18, UK-PG, USA-PG-PG13, USA-R.


 For each domain, we filter out terms that do not occur in at least two documents, and additionally limit the maximum number of words in a vocabulary to 100,000. For all of these datasets, we split them into a 2/3 training data, 1/3 test data split. We additionally remove the end 20\% of the training data and use that as development data for our hyper-parameters, which is then not used for the final models verified using test data.  For the movies and place-type domains, the original text was not available.



\section{Space Types}

Below the choices for the Vector Space Models that are formally described in Section \ref{bg:SemanticSpaces} are explained:

\textbf{Multi-Dimensional Scaling (MDS)}: Following \cite{derracAIJ}, we use Multi-Dimensional Scaling (MDS) to learn semantic spaces from the angular differences between the PPMI weighted BoW vectors. %A non-linear transformation that is used to evaluate the quality of representations when built from a standard BOW-PPMI. Chosen as it performed well in the work introducing this method.

\textbf{Principal Component Analysis (PCA)}: directly uses the PPMI weighted BoW vectors as input, and which avoids the quadratic complexity of the MDS method. A standard dimensionality reduction technique, used as a baseline reference.

\textbf{Doc2Vec (D2V)}: Inspired by the Skipgram model \cite{DBLP:conf/icml/LeM14}.  A distributional document representation used as a representative of a higher performing method of learning in terms of document classification. For the Doc2Vec space, the hyper-parameters are additionally tuned for the $window size (5, 10, 15)$ referring to the context window, the $min count (1, 5, 10)$ referring to the minimum frequency of words and the $epochs (50, 100, 200)$ of the network for each size space. The process with our two-part hyperparameter optimization as in this case is as follows: Grid search is used to select the parameters for the representation, then find the most suitable model (e.g. Decision Tree, SVM) for that representation. 

\textbf{Average Word Vectors (AWV)}: Finally, we also learn semantic spaces by averaging word vectors, using a pre-trained GloVe word embeddings trained on the Wikipedia 2014 + Gigaword 5 corpus\footnote{\url{https://nlp.stanford.edu/projects/glove/}}. While simply averaging word vectors may seem naive, this was found to be a competitive approach for unsupervised representations in several applications \cite{DBLP:conf/naacl/HillCK16}. We simply average the vector representations of the words that appear at least twice in the BoW representation.



\subsection{Conclusion}

In conclusion, we introduce a methodology to go from a Vector Space Model of Semantics and an associated bag-of-words to an interpretable representation and interpretable classifiers. We define an interpretable representation in this work as having two properties: disentanglement and labels, and an interpretable classifier as a simple linear classifier that has components corresponding to the interpretable representation that has these properties, e.g. nodes in a decision tree. In general, we give a simple methodology that can be used to achieve interpretable features and classifiers as an alternative to methods like Topic Models, and give insight into the parameters required and qualitative results that can be obtained. We extensively test the qualitative and quantitative results, finding that the highest-performing quantitative results also make good intuitive qualitative sense. We find that our method greatly outperforms the original representations on low-depth Decision Trees, giving good evidence that we have disentangled the representation. Additionally, we find that we are also competitive with standard interpretable representation baselines in most cases. We introduce variations to the original work that produced these kind of interpretable representations, in particular finding that scoring directions using NDCG performed better than Kappa in most cases, and that we could achieve much stronger results than the original clustering method using K-means. Further, we experimented using a variety of space-types and domains, verifying that the methodology can be applied more generally than shown in \cite{Derrac2015}. The main experiments that would be interesting to expand on for this chapter would be more state-of-the-art representations, specific investigations of how those representations are able to achieve such strong results, and interpretability experiments to see how our cluster labels fare in real-world situations.

\chapter{Re-structuring Vector Spaces into Interpretable Representations}\label{Chapter3}


\section{Introduction}\label{chapter3:Introduction}
%Representing the knowledge present in natural language data is a key task in Natural Language Processing. The most common representation is one based on frequency statistics: the Bag-Of-Words (BOW). This representation has the benefits of being easy to understand on a granular level, as each feature is a distinctly labelled word. However, it does not capture more complex relationships between words at a conceptual level. For example, it may be the case that in a text domain of movie reviews a horror movie frequently mentions the words "scary" "blood" and "gore", and through these terms occurring together we may infer that it is a Horror movie, or that other documents with similar words like "terrifying" or "blood" are similar documents. This is similarity information, which can be induced by taking the frequency vectors of a Bag-Of-Words and interpreting them in terms of relative cosine similarity. However, simply finding the similarity between each document would result in an \textit {N} by \textit {N} representation. Which is obtrusively large. In reality we do not need a dimension per document - it is more likely that a group of documenta e.g. Horror movie reviews from a domain of movies could be used as a "Horror" feature by averaging their dimensions together, the same with romantic movies or particularly cinematographally good ones. This is why typically these similarity representations are obtained with dimensionality reduction methods - with the general rule of thumb being that you want as many salient features as dimensions, where smaller dimensions represent more general concepts (with the most specific amount of dimensions being equivalent to the amount of words). 

%\noindent \textbf{Semantic spaces.} Within the field of cognitive science, feature representations and semantic spaces both have a long tradition as alternative, and often competing representations of semantic relatedness \cite{tversky1977features}. Conceptual spaces \cite{gardenfors2004conceptual} to some extent unify these two opposing views, by representing objects as points in vector spaces, one for each facet (e.g.\ color, shape, taste in a conceptual space of fruit), such that the dimensions of each of these vector spaces correspond to primitive features. %Different from other vector space models, however, a conceptual space is typically composed of several vector spaces, each of which intuitively models a single facet from the given domain. For example, a conceptual space of fruit could be composed of vector spaces modelling color, shape, taste, price, size and weight. 
%The main appeal of conceptual spaces stems from the fact that they allow a wide range of cognitive and linguistic phenomena to be modelled in an elegant way. The idea of learning semantic spaces with accurate feature directions can be seen as a first step towards methods for learning conceptual space representations from data, and thus towards the use of more cognitively plausible representations of meaning in computer science. Our method also somewhat relates to the debates in cognitive science on the relationship between similarity and rule based processes  \cite{HAHN1998197}, in the sense that it allows us to explicitly link similarity based categorization methods (e.g.\ an SVM classifier trained on semantic space representations) with rule based categorization methods (e.g.\ the decision trees that we will learn from the feature directions).


This chapter introduces a methodology to go from a Vector Space Model (VSM) of text-documents and associated Bag-Of-Words (BOW) representation to an interpretable document representation. Interpretability as an idea has a variety of interpretations and meanings (See \ref{ch2:Interpretability}). In this thesis, an ideal interpretable representation's features have three important properties: First that each feature models an important aspect of the domain, second that these aspects are distinct from each other, such that each feature represents a unique aspect of the domain, and thirdly that each feature is labelled with a cluster of words describing the aspect of the domain the feature is modelling. To give an insight into what kind-of domain knowledge these features represent, examples of word clusters that describe features are shown In Table \ref{ch3:ExampleRep}. 










% These interpretable representation are demonstrated to be effective in simple linear classifiers on document classification tasks. 


The method is entirely unsupervised and 'disentangles' an existing vector space model into salient features. The idea of disentanglement is present in representation learning \cite{Bengio2012} e.g. when given a raw video file of a person jumping, ideally a representation would spatially separate the notions of 'jumping', from the 'person', and the 'background'. In this work disentanglement is used to refer to separating an existing vector space into salient features, such that each dimension of the space is a labelled feature. 

\begin{table}[] 
	\scriptsize
	\begin{tabular}{lll}                                                                   
		\textbf{IMDB Movie Reviews}                                 & \textbf{Flickr-Placetypes}           & \textbf{20-Newsgroups}                           \\
		\toprule
		courtroom legal trial court                                 & broadway news money hollywood        & switzerland austria sweden swiss     \\
		disturbing disgusting gross                                 & fir bark activism avian              & ham amp reactor watts                \\
		tear cried tissues tears                                    & palace statues ornate decoration     & karabag armenian karabakh azerbaijan \\
		war soldiers vietnam combat                                 & drummer produce musicians performers & 4800 parity 9600 bps                 \\
		message social society issues                               & ubahn railways electrical bahn       & xfree86 linux                        \\
		events accuracy accurate facts                              & winery pots manor winecountry        & umpires umpire 3b viola              \\
		santa christmas season holiday                              & steeple religion monastery cathedral & atm hq ink paradox                   \\
		martial arts kung                                           & blanket whiskers fur adorable        & lpt1 irq chipset mfm                 \\
		bizarre weird awkward                                       & desolate eerie mental loneliness     & manhattan beauchaine bronx queens    \\
		drug drugs dealers dealer                                   & carro shelby 1965 automobiles        & photoshop adobe                      \\
		inspirational inspiring fiction narrative                   & relax dunes tranquil relaxing        & reboost fusion astronomers galactic 
	\end{tabular}  
\caption{Example features of our interpretable representation from three different domains. Each row is a label for a feature from our representation for that domain}\label{ch3:ExampleRep}    
\end{table}   

Domain-specific semantic spaces like the ones in this work are used, for instance, to represent items in recommender systems \cite{Vasile:2016:MPE:2959100.2959160,liang2016factorization,van2016learning}, to represent entities in semantic search engines \cite{DBLP:conf/sigir/JameelBS17,van2017structural}, or to represent examples in classification tasks \cite{DBLP:conf/iccv/DemirelCI17}. Methods like Multi-Dimensional Scaling and Principal Component Analysis that produce these semantic spaces are in widespread use for document representation and data analysis, and are typically built from word frequency statistics (See \ref{ch2:reps}). However, a wide variety of training methods have been used to obtain semantic spaces. Distributional word-vectors that rely on learning via word-context have had great success as a component of neural learning systems achieving state-of-the-art results on key natural language processing tasks like Language Modelling \cite{Gong2018}, Constituency Parsing \cite{Edunov2018}, and Part-Of-Speech Tagging \cite{Edunov2018}, and have also been applied for document representations \cite{Le2014, Lau2014}. Ideally, we would like to retain the benefits of these learning methods while also making them interpretable. The methodology in this work is a post-processing step that can be applied to representations regardless of how they have been learned, by leveraging the spatial relationships in the representation. %The intention of this work is to produce interpretable representations and simple interpretable classifiers that use the domain knowledge spatially encoded by any learning method. 
%Interestingly, the method can even be used to obtain an interpretable representation from neural networks, as they learn vector space models as hidden layers, which is investigated in Chapter \ref{ch5}. %as hidden layers have achieved state-of-the-art results in a majority of NLP tasks e.g. learning using large amounts of data, transferring that data between tasks, or integrating additional constraints.
%in particular those based on the distributional hypothesis have become commonly used in many state-of-the-art natural language processing models, and have been augmented and improved in a variety of ways. For example, Distributional word vectors have achieved state-of-the-art in Language Modelling by ensuring frequent and rare words are given similar semantic importance \cite{Gong2018}, 



%%%% Build up the variance in space types and how they are produced 
%There are many ways to produce a Vector Space Models, from frequency statistics using dimensionality reduction methods like PCA or MDS, to applying the distributional hypothesis and learning through word context, to learning neurally with structural or supervised constraints, e.g. enforcing hierarchical grammar, or integrating different kinds of data, e.g. image data. This has resulted in a wide variety of spatial structures in these representations, each one suited to certain tasks or domains. These representations, in particular distributional and neural representations have recently found widespread success in many domains, e.g. X, Y, Z. 

% (e.g. natural clusters form like in a domain of movie reviews, movies of a positive sentiment would be in a cluster distinct from those of negative sentiment)
%These concepts are instead embedded in the spatial relationships of the space. This has lead to work in understanding exactly how these spatial relationships correspond to domain knowledge. There are many approaches to understanding Vector Space Models, from visualization and exploration tools enforcing sparsity constraints on existing methods, e.g. on distributional vectors using NNSE or with sparse-PCA, to neural network models that produce an interpretable representation e.g. InfoGan is able to produce a representation where each dimension corresponds to an aspect of a handwritten digit, with one dimension for style and so on, or other networks or finding clusters or labelling existing dimensions. However, in this case we are interested in a linear transformation of any existing Vector Space Model. This is a representation agnostic approach, which is valuable as it can make use of the structures already developed for whatever representation is suitable for our task. The benefit of this can even extend to interpreting the 'hidden layers' of supervised networks, as seen in Chapter 5. The second benefit  our method has is that it is entirely unsupervised, and by ensuring that salient semantics of the space are used as features for the new representation we are able to produce simple interpretable linear classifiers, in this case the interpretability of these classifiers being defined as using our interpretable features as components. We show an example of a tree using such features below.

%example nne, nnse, or for frequency statistics sparse pca. In terma of functionality, we coiod compare our method to obtaining a specialized topic model, as both contain labelled features - with the main difference being topic model representations are induced probabilistically rather than inducing a Vector Space Model. We show a comparison between the best performing topic model topics (LDA, see \ref {ch2:LDA}) and our method below.

%In this chapter, we experiment with a method to produce an interpretable representation from an existing Vector Space Model and its associated Bag-Of-Words by leveraging its spatial relationships. This approach gives us a number of benefits. First, we are able to leverage a variety of different Vector Space Models according to the problem that needs to be interpretably solved. Second, we can transform the space in an entirely unsupervised way without needing additional data or supervision signals. 

%The simple linear interpretable classifiers we can obtain from our method are low-depth Decision Trees. if those dimensions correspond closely to the classes we intend to solve (e.g. classifying if a movie is a horror wouls only require a single dimension tp be used in the classifier, and would resolve many desirable properries e.g. generalization.

%in parare useful for our method because unlike a Bag-Of-Words, they encode how these words relate to each other, form the greater concept of a 'Horror' movie, and how the movies which contain these words relate to each other. This kind of knowledge is beneficial to complete knowledge bases, and to capture more complex relationships that can correspond to fuzzy and commonsense knowledge about documents,

%The origin of our method is in commonsense reasoning. The structure of Vector Space Models have been shown to contain properties of commonsense reasoning,  e.g. where movie reviews are documents analogical relations (scary movie 1 is to scary movie 2 as harry potter and the philosophers stone is to harry potter and the chamber of secrets), commonsense betweeness (movies age rated 15 would be spatially between those rated 18 and those rated 12), and so on. These are the kind of semantic relationships we leverage to obtain our representation.

%The method transforms the space into an interpretable representation built on the spatial relationships of the original space. 
The spatial structures of semantic spaces have been used in a variety of ways. In \cite{Dai} it was shown that it is possible to perform vector operations on Paragraph Vectors,  e.g. subtracting word-vectors from paragraph vectors, like in the case of a corpus of arxiv papers, a paper titled "Spectral Clustering", could have the word-vector for "Spectal"  subtracted from it to get papers about general clustering. In the case of distributional representations of words \cite{TomasMikolovWen-tauYih2013} found that "equivalent relations tended to correspond to parallel vector differences" \cite{Mitchell2015}, found that by decomposing representations into orthogonal semantic and syntactic subspaces they were able to produce substantial improvements on various tasks. In \cite{kim2013deriving} directional vectors in word embeddings were found that correspond to adjectival scales (e.g.\ bad $<$ okay $<$ good $<$ excellent) while \cite{Rothe2016} found directions indicating lexical features such as the frequency of occurrence and polarity of words.

The spatial structures we leverage in this work are found in document representations. In particular, directional vectors that describe a particular feature of a domain. A toy example is shown in Figure \ref{ch3:ToyDirectionsGraphic}. These directions have been applied in a variety of domains. For instance,  \cite{gupta2015distributional} found that features of countries, such as their GDP, fertility rate or even level of CO$_2$ emissions, can be predicted from word embeddings using a linear regression model.  Derrac \cite{Derrac2015}  found directions corresponding to properties such as `Scary', `Romantic' or `Hilarious' in a semantic space of movies, for example a direction which goes from a movie that is the least 'Scary' to the most 'Scary'. This chapter builds on their work, with the main contributions of the chapter being application of this method to producing an interpretable representation, deeper and more extensive experimentation, qualitative analysis and application to interpretable classifiers.  %The basic idea of this chapter is that we use these feature-directions to produce a representation where each dimension is a ranking of entities on those feature-direction, and evaluate its effectiveness using simple linear classifiers.

Such feature directions are useful in a wide variety of applications. The most immediate example is perhaps that they allow for a natural way to implement critique-based recommendation systems, where users can specify how their desired result should relate to a given set of suggestions \cite{Viappiani2006}. For instance, \cite{Vig2014} propose a movie recommendation system in which the user can specify that they want to see suggestions for movies that are ``similar to this one, but scarier''. If the property of being scary is adequately modelled as a direction in a semantic space of movies, such critiques can be addressed in a straightforward way. Similarly, in \cite{Kovashka} a system was developed that can find ``shoes like these but shinier'', based on a semantic space representation that was derived from visual features. Semantic search systems can use such directions to interpret queries involving gradual and possibly ill-defined features, such as ``\emph{popular} holiday destinations in Europe'' \cite{Jameel}. While features such as popularity are typically not encoded in traditional knowledge bases, they can often be represented as semantic space directions.  As another application, feature directions can also be used in interpretable classifiers. For example, \cite{Derrac2015} learned rule based classifiers from rankings induced by the feature directions.


%e give an example of the dimensions of a 'disentangled' representation in \ref{ch3:disentangled}. and our experiments in this chapter focus on the performance of a disentangled representation in this context. Further, the methods we investigate are entirely unsupervised, needing only a Bag-Of-Words and an existing Vector Space Model. 

%Interpretable models are generally in three categories, 
%But these topic models do not contain the rich spatial relationships found in Vector Space Models, and their representations cannot be augmented by additional data or grammatical information in exactly the same way.  %Frequency statistics and distributional neural embeddings like word-vectors have achieved strong performance on X by doing B, Y by doing N, Z by doing K. 
%For example, a good vector space in the domain of movies constructed from IMDB movie reviews should contain a natural separation of entities into genres, where Horror movies are spatially distant from Romance movies, and movies that are Romantic Horrors would be somewhere inbetween. We can see an example in Figure \ref{figure:genres_separated}. For a Bag-Of-Words, we can expect similar entities to have similarly scoring terms \ref{PPMI table:PPMI_example}.

%\begin{figure}[t]
%	\includegraphics[width=\textwidth]{images/genres_separated.png}
%	\centering
%	\caption{A conceptual space of movies, where regions correspond to properties and entities are points.}\label{figure:genres_separated}
%\end{figure}

%The relationships in Vector Space Models can be extracted to obtain domain knowledge. %king and queen was parallel to the vector between prince and princess. % NEED TO MAKE THIS REAL



\begin{figure}[t]
	\includegraphics[width=250px]{images/toy_domain.png}
		\centering
	\caption{An example in a toy domain of shapes.}\label{ch3:ToyDirectionsGraphic}
\end{figure}

The simple linear classifiers that are used to evaluate the method's feature directions are low-depth Decision Trees. In Figure \ref{ch3:DecisionTree} an example is shown of a shallow Decision Tree using the method's interpretable representation. Shallow Decision Trees were chosen because they are effective at evaluating the disentanglement of the representations features. If the features are disentangled, then a low-depth Decision Tree will suffice to classify natural domain tasks. Shallow trees also evaluate the semantic generalizability of the features, as if they are able to classify complex classes using only a single feature then that feature must be semantically coherent and generalizable. In terms of interpretability, shallow trees have many positive effects for users, like lower response times \cite{Narayanan2018, Huysmans2011}, better question answering and confidence for logical problem questions \cite{Huysmans2011} and higher satisfaction \cite{Narayanan2018}. Although in this work  the superiority of  low-depth Decision Trees in real-world interpretability applications is not in the scope of the evaluation, as the interpretable representations  could be applied to a variety of classifiers.

The quantitative results for the method results show that the method can successfully disentangle a variety of representations even with trees as limited as depth one, and these shallow trees outperform the original representation greatly when compared to deeper trees on the uninterpretable original features. Additionally, the results in most cases are also competitive with Latent Dirchlet Allocation, a baseline interpretable topic model. The method is shown to be an effective way to obtain a disentangled representation that can effectively produce simple interpretable classifiers. The method is verified to work on five different representation types for five different domains, using natural domain tasks for those domains. 

\begin{figure}[t]
	\includegraphics[width=450px]{images/decision_tree_ex.png}
	\centering
	\caption{An example of a Decision Tree classifying if a movie is in the "Sports" genre. Each Decision Tree Node corresponds to a feature, and the threshold T is equal to the ranking of a document on that feature. The most important direction is used twice, referring to sports and resulting in a majority of negative samples. The nodes at depth three are more specific, sometimes overfitting (e.g. in the case of the "Virus" node) }\label{ch3:DecisionTree}
\end{figure}


%Why interpretability for document classification matters:

%What Vector Space Models are there that perform well at tasks:

%What document classification tasks are important:

%What methods are there for interpretable document classification:

%Derrac \cite{Derrac2015} introduced an unsupervised method to go from a Vector Space Model and its associated Bag-Of-Words to a representation where each dimension is a ranking of documents on a feature of the domain. For example, in the domain of movie reviews genres would be a feature, and the dimension would have a numeric value for each document corresponding to the degree it is a particular genre. The contribution of this Chapter is an analysis and experimentation on the quality of these features applied to document classification. The main insight from our work is that these interpretable features do not suffer a performance drop in a non-linear classifier compared to the original representation, and can outperform the original representation and a baseline interpretable representation in a linear classifier. In addition, we find that if a dimension ranks documents on a feature relevant to the task, it can be competitive with more complex models using a single Decision Tree node. We show an example of the representation from a domain of IMDB movie reviews in \ref{ch3:TreeAndRep}. 

%\begin{figure}[t]
%	\includegraphics[width=\textwidth]{images/tree and rep.png}
%	\centering
%	\caption{Example movies and selected associated dimensions, chosen according to their relevance to the genre task.}\label{ch3:TreeAndRep}
%\end{figure}

%Vector Space Models encode semantic relationships

%What are the applications of semantic relationships? What is the power of a semantic relationship? Why are semantic relationships important in a space?
%prolly need more about

%What are directions? Why are directions important? What have they been used for? What is a feature direction?


% How can I apply directions to produce an interpretable space? Why is this valuable? What makes this interesting? What are feature rankings?


%%What are the advantages, disadvantages of the previous work?



%Why do this work? What is the value of this work?  How is this relevant to the readers interests?


%\subsection{Using a Property Representation in a Linear Classifier}


%Our method can use any vector space that linearly separates entities, and so it has potential longevity. This means that our method is relying on structure in the space that does not directly correspond to our desired representation -  we can view our approach as a linear transformation of the space. We address this problem in Chapter \ref{chapter:finetuning}. However, we have the capability to leverage many different methods to construct a vector space for our representation, so as long as dense representations of entities exist it will be possible to use our method, and as they are improved the results that our method can achieve will be improved too. This kind of flexibility also gives us the potential to combine the resultant representations from different vector spaces for classification, e.g. concatenating the vectors from different spaces.

%Topic models such as Latent Dirichlet Allocation (LDA) represent documents as multinomial distributions over latent topics, where each of these topics corresponds to a multinomial distribution over words \cite{Blei2003}. These topics tend to correspond to semantically meaningful concepts, hence topic models tend to be rather interpretable \cite{Chang2009}. To characterize the semantic concepts associated with the learned topics, topics are typically labelled with the most probable words according to the corresponding distribution. 

% What is a Vector Space Model? Why is a Vector Space Model valuable?




% 




%%What is the previous work?
 %What is a direction




%Finally, \cite{derracAIJ} found salient properties as direction vectors in a Vector Space Model of entities (e.g. Movies in a domain of IMDB movie reviews), and labelled them with clusters of words (e.g. $p_1 = {Scary, Horror, Gore}, p_2 = {Funny, Laughter, Hilarious}$).  We show a toy example in Figure \ref{ToyDirection}. %Copied from CONLL 




%%What is our contribution towards those disadvantages in this chapter?
%"Decision Trees, Decision Tables and Textual descriptions of rules are logically equivalent in the sense that one type of representation can be automatically translated to another (albeit in a simpler or more complex form), while preserving the predictive behaviour of the original model"


%%What are some real world scenarios that you could see our work taking effect in?
%In a case study by , giving the business users the option between a model with higher classification score but more input variables and a lower classification score but less input variables resulted in more buy-in for system designers. By accurately representing salient concepts in the domain, we are also able to offer a similar option; less nodes in the Decision Tree in exchange for more accuracy. % Potentially this citation sucks


%How is the work evaluated? How can you justify the evaluation? 
%%What are some alternative interpretable classifiers? What are some approaches to interpretable classification?

%%How does our work fit into the niche of interpretable classifiers?

%%What kind of tasks are these interpretable classifiers usually on? How do we compare in terms of evaluation?

%%How does our evaluation of interpretability play into our idea of interpretability outlined in Chapter 1?


The interpretable representation that is obtained by this method is composed of in terms of salient features, where each of these features is described using a cluster of natural language terms. This is somewhat similar to Latent Dirichlet Allocation (LDA), which learns a representation of text documents as multinomial distributions over latent topics, where each of these topics corresponds to a multinomial distribution over words \cite{Blei03latentdirichlet}.  Topics tend to correspond to salient features, and are typically labelled with the most probable words according to the corresponding distribution. On the other hand, our work leverages clustering methods to obtain the feature labels. Broadly speaking, in the context of document classification, the main advantage of topic models is that their topics tend to be easily interpretable, while Vector Space Models tend to be more flexible in the kind of meta-data that can be exploited e.g.\ they allow us to use neural representation learning methods to obtain these spaces. The approach proposed in this Chapter aims to combine the best of both worlds, by providing a way to derive interpretable representations from Vector Space Models.  Many extensions of LDA have been proposed to incorporate additional information as well, e.g.\ aiming to avoid the need to manually specify the number of topics \cite{teh2005sharing}, modelling correlations between topics \cite{Blei2006}, or by incorporating meta-data such as authors or time stamps \cite{rosen2004author,wang2006topics}. Nonetheless, such techniques for extending LDA offer less flexibility than neural network models, e.g.\ for exploiting numerical attributes or visual features. For comparison, in our experiments the standard topic model algorithm Latent Dirchlet Allocation (LDA) is used as a baseline to  compare to the new methodology that transforms standard Vector Space Model representations. %Topic Models are also entirely probabilistic, while our method relies entirely on the spatial relationships present in the vector space model.


There is much work on learning interpretable representations, with one popular way being to introduce sparsity or non-negativity constraints while learning, for example, sparse PCA learned using the l1-norm, \cite{H.Zou2006} \cite{Zhang2012},  or Non-Negative Sparse Embeddings (NNSE)  \cite{Murphy} which are sparse interpretable word-vectors obtained using sparse-matrix factorization and non-negativity constraints. A similar technique can also be applied to distributional word-embeddings by integrating this method with the Skip-Gram model \cite{Luo2015}. However, our approach is not intended to transform the learning processes, but rather be a post-processing step on an existing representation.

Similar to the approach in this chapter, \cite{Faruqui2015} introduce a post-processing method to convert any distributional word-vector into sparse word vectors, which additionally satisfy our idea of disentangled interpretability. However, the representation produced by the method in this work differs from sparse representations in that it is dense, where each feature is salient and interpretable. Another method is to describe a representation, e.g. sense word-embeddings that are linked to synsets \cite{Panchenko2016} in-order to make them interpretable. Although this is a post-processing step similar to our method, this is a linking rather than a transformation of the representation.  

Another method is to integrate grammatical structure into the learning of the representation, for example \cite{Liu2017} obtained a representation learned with attention mechanisms on the dependency structures of sentences, but this differs from the intention of our work, which is not to introduce new structures to the representation to make it more interpretable but instead use the already existing structure to obtain an interpretable representation. For short interpretable documents, \cite{Martinc} introduced tax2vec, which produced interpretable features from word taxonomies, useful for low data models. In \cite{Code} word-vectors were clustered and then used as a bag-of-clusters, where if a word occurs in those word-vector clusters it contributes to the Bag-Of-Words frequency. Although clustering is used in the method, it is not used to create a Bag-Of-Words, instead relying on the spatial relationships in the space as our representation. 

%How is the chapter going to play out? Whats going to happen?
%As our work performs well even at lower-depth trees, this gives potential users more flexibility in how they want to present the information, e.g. to a potential client. Compared to Bag-Of-Words, which loses its representation capabilities the lower the depth.

This chapter continues as follows: First the method is described, making explicit the variations from to the original method in \cite {Derrac2015}. This is followed by a qualitative  and quantitative analysis, finishing with a conclusion on the  benefits and limitations of this approach.



%\section{Related Work}
%Sparse word vectors
%Adapted to composition \cite{Fyshe2015}
%\subsection{Semantic Relations \& Their Applications}

%Our method uses the relationships inherent in a Vector Space Model. Other work has formalized the relationships found in Vector Space Models, for example in word-vectors, linear analogies (see Section \ref{WordVectors, Ethayarajh2018}, were found where the vector between  %Copy pasted from this 
%[ENTIRE SECTION COPY PASTED FROM PREVIOUS PAPER]
%\textbf{Linear Classifiers}
%Decision Trees, linear SVM's, logistic regression, decision tables, IF Then rules.

%What are the available options for interpretable linear classification?

%How have each of these methods been measured or validated in the literature in regards to interpretability? How about application to real world situations?

%\textbf{Non linear classifiers}
%What non linear classifiers networks are interpretable? How have they done it? How have they measured it? How does it compare to a linear method?

%\textit {Neural networks}Approximating w/linear model, Interpretable nodes/weights

%\textit {Other Stuff}

%\subsection{Interpretable Representations}


%There are two ways in which topic models can be used for document classification. First, a supervised topic model can be used, in which the underlying graphical model is explicitly extended with a variable that represents the class label \cite{Blei2010}. Second, the parameters of the multinomial distribution corresponding to a given document can be used as a feature vector for a standard classifier, such as a Support Vector Machine (SVM) or Decision Tree. .




% One of the more popular models for text representation that labels features in a similar way to our method are Topic Models.


\section{Method}\label{ch3:method}
% Assumed understanding at this point
% Chapter 1: Why interpretability is important. The value of a good interpretable representation. What directions are and what they mean.
% Chapter 2: The background and literature review for interpretable representations.
% Chapter 3: An introduction on the value of a good interpretable classifier, and the position of our work in that domain and how it can be applied there. What entities, properties are. What a conceptual space is. A literature review on relevant interpretable classifiers.

This section details the methodology to obtain an interpretable representation from only a Vector Space Model and its associated Bag-Of-Words \ref{bg:SemanticSpaces}. %which whose features rank documents on salient semantics, e.g. In a domain of IMDB movie reviews, a movie would be ranked on semantics like  $1: \{Scary, Horror, Bloody\}$ and $2: \{Romantic, Love, Cute\}$, ideally with as many rankings as salient semantics of the domain. 
%Make sure features ae explained well earlier 

%\subsection{The Structure of a Vector Space Model}

%Salient features of the domain are encoded in the structure of a Vector Space Model. 
% What are directions? How are they encoded? Why are they encoded that way?



 

%\begin{figure}[t]
%%	\includegraphics[width=\textwidth]{images/DirectionsGraphic.png}
%	\centering
%	\caption{Original And Converted.}\label{ch3:DirectionsGraphic}
%\end{figure}
%The method to obtain interpretable feature-vectors is an extension of the work by \cite{derracAIJ}. This previous work showed how to, filter out words, cluster words to get features, and obtain rankings of documents on those features. In this section, we further analyse and extend this work, in particular by testing a variety of additional filtering methods and clustering methods, and demonstrating how these feature-vectors can produce simple linear interpretable classifiers. 
%To begin, we filter out terms that will not be well represented in the space, and in-turn will not result in good feature rankings. We do so by first removing all terms below a frequency threshold. Then, for each word we obtain a direction in the space that can induce a ranking of documents on that word, and evaluate how well represented that word is in the space and remove those below a threshold. We can consider the terms remaining to be candidate feature-directions, as they are all well represented in the space and are unlikely to be highly scored as representative due to being infrequent. However, it is possible that they may not represent different information from each other, e.g. "blood" and "bloody" are likely quite similar. Additionally, we can understand that a good feature will not correspond directly to a term, but instead to a more abstract concept, e.g. "Blood, Gore, Horror". To obtain these more abstract feature-directions, and ensure the feature-directions are distinct rather than representing the same information, we cluster together similar feature-directions and obtain their feature-rankings to obtain our final representation. Each of these feature-rankings can be labelled with the cluster of words they are composed of. 



\subsection{Obtaining Directions and Rankings From Words}


We explain the method in terms of document classification.  Assuming a Bag-Of-Words $B_w $ has an associated vocabulary $W_w $, 
in this section we introduce the first step: how to obtain feature-directions $D_w $ for each word $W_w $, and rankings of documents  on these directions $r_w $, where each word is ranked on every document. For this step, not all words are expected to be salient in the domain. Instead, the first step shows how to obtain an interpretable representation where every document is ranked on every word, and the next step shows how to filter these rankings to only salient features.

%%%%%%%%%%%%Features that are important will be spatially organized in a way that reflects the similarity between the Bag-Of-Words representations they were composed of. In particular, we expect that documents will be arranged in a direction, where generally the higher the PPMI score for a group of words that correspond to a feature (e.g. $Horror, Scary, Gore$) the further away they will be from those that have low PPMI scores for those words. We give examples of this in Figure \ref{ch3:DirectionsGraphic}, by projecting documents into a 2D space of salient features we are able to show that these documents are structured according to directions for these features. Salient features will typically be a more abstract representation which will be natural in the domain, e.g. in a domain of IMDB movie reviews, genres

\noindent \textbf{Obtaining directions for each word} Each document is referred to as a point $d_p $ in the vector space model $S_{d}$. For each word $w$, a hyper-plane is obtained $h_w $ from a Linear Support Vector Machine (See Section \ref{bg:SVM}\footnote{This was also tested using a logistic regression classifier, and achieved similar results}) that is trained on the Bag-Of-Words representation so that document points $d_p $ in the space $V_d$  where the word $w $ occurred more than once  for that document $d_{wf} >= 1$ are separated from those where the word did not occur $d_{wf} == 0$. This process is repeated such that a hyper-plane is found for all words in the vocabulary above a frequency threshold $w_f > T$ where $T$ is chosen such that words which are infrequent enough to cause the classifier to overfit are not included. As this task is unbalanced, i.e. there are typically less documents that contain the word compared to those that do not contain it, the weights of the classifier are balanced such that positive instances are weighted in proportion to how rare they are. 

As previously mentioned, not all words will be influential on the structure of the representation. Only words that are salient will be well separated. Although the hyperplane $h_w$ learned is binary (either classifying documents $d_p$ as negative or positive), it can  be expected that the distance of the document points $d_p$ from the hyperplane boundary varies, as the space's $V_w$ similarity structure is in degrees rather than hard boundaries. For example in a space constructed from frequency vectors $W_{wf}$, it can be expected that the documents which contain the word more frequently would be further away from the hyper-plane on the positive side. Similarly, in the case of our experiments, the documents with a higher PPMI value will be more distant from the hyper plane on the positive side. This is the insight that informs the method to obtain the direction. 

The vector $v_w$ perpendicular to the hyperplane $h_w $ is taken as a direction $D_w $ that models documents $d_p$ from the lowest document ranked on the word $w $ (at the distance furthest from the hyperplane on the side where documents $d_p $ are classified) to the highest ranked on the word $w$ at the distance furthest from the hyperplane at the positive side. An example of such a direction $D_w$ is shown in the toy domain in Figure \ref{ch3:ToyHyperPlane}. To  apply this idea to a real domain, we can give an example from movie reviews, where the word is 'Scary' and the most 'Scary' movies are at the tip of the direction and those that are least 'Scary' are at the base of the direction.  %Although these directions do formally correspond to vectors, we refer to them as directions to emphasize their intended ordinal meaning: feature directions are aimed at ranking entities rather than e.g.\ measuring degrees of similarity. %Graphical representation of 


\begin{figure}[t]
	\includegraphics[width=250px]{images/ToyHyperplane.png}
	\centering
	\caption{An example of a hyper-plane and its orthogonal direction in a toy domain of shapes. Green shapes are positive examples and red shapes are negative examples, but despite the problem being binary those closest to the hyper-plane are less defined than those further away, resulting in the orthogonal vector being a direction}\label{ch3:ToyHyperPlane}
\end{figure}




\noindent \textbf{Ranking documents on directions} Although we do refer to the direction $D_w$ ranking documents on a word $w$, we do not yet have a specific value to represent this ranking. Once a feature-direction vector is obtained for each word $D_w$ the next step is to quantify the degree to which each document $d_p$ has that word, by obtaining a value that corresponds to how far-up it is on the direction vector $D_w$.  If $d_p$ is the representation of a document in the given vector space as a point then the dot product is used between the direction vector for the word $D_w $ and the document vector $D_w \cdot p_d$ as the ranking $r_{dw}$ of the document $d$ for the word $w$, and in particular, we take $r_{d1} < r_{d2}$ to mean that the document $d_2$ 'has' the feature  to a greater extent than $d_1$ (e.g. in a domain of movie reviews if the word is 'cinematography', the movie will likely have notable cinematography). Once the dot product value is obtained in this way for each document on a word, this forms a ranking feature for the word.  By obtaining a ranking of all documents on all words,  a rankings matrix of size $d_n * w_n $ can be obtained. This representation forms the basis of the method, with future steps removing those directions that are not salient, and then clustering similar directions together. %To give an understanding of these rankings and their saliency, we show some examples of features and documents ranked on them for different domains with varying degrees of saliency. In the next section, we show how we measure the saliency of directions and filter those that are not salient out of the representation. %Graphical representation of entities being ranked on a direction vector






%\begin{figure}[t]
%	\includegraphics[width=\textwidth]{images/genres_separated.png}
%	\centering
%	\caption{Original And Converted.}\label{ch3:OrigAndConverted}
%\end{figure}



\subsection{Filtering Words}

%With the rankings $R_r$, we could create a representation of each document $d$, composed of $w_n$ dimensions, where each dimension is a ranking of the document $d$ on that word $r_dw$. However, many of the words are not spatially important enough in the representation to result in a quality ranking - they are not salient features.

In this section, words are filtered that do not influence the structure of the domain. This is done by evaluating them using a scoring metric, and removing the words that are not sufficiently well scored. Originally, only binary features were considered. These binary features were measured  in terms of the performance of the SVM classifier.  If the hyperplane correctly separates the entities well, it must mean that whether word $w$ relates to document $d$ (i.e.\ whether it is used in the description of $d$) is important enough to affect the Vector Space Model representation of $d$.  However, this approach does not consider the quality of the ranking. To consider this, the new metric Normalized Discounted Cumulative Gain was introduced, using the bag-of-words as its target ranking under the assumption that if a ranking matches the ordered score of a PPMI BOW, then it is a good ranking. If this is the case, it can also be assumed that this means the word was strongly influential in the space, as it retains the detail of the Bag-Of-Words information in the space's structure. 

\noindent

\noindent \textbf{Cohen's Kappa}. This is the metric used in the work that originally introduced this method \cite{Derrac2015}. This is a binary feature evaluation metric that deals with the problem that these words are often very imbalanced. In particular, for very rare words, a high accuracy might not necessarily imply that the corresponding direction is accurate. For this reason, they proposed to use Cohen's Kappa score instead. In our experiments, however, it was found that this can be too restrictive, allowing us to sometimes obtain better results with the more simple accuracy metric.\smallskip %This is basically copy pasted.


 \textbf{Classification accuracy}. If a model has high accuracy for a word $w$, it seems reasonable to assume that $w$ describes a salient property for the given domain. However, despite balancing the weights of the original SVM used to obtain the hyper-plane, the value this metric places on correctly predicting negative classification often results in noise particular to this metric being identified, e.g. metadata like a reviewers name that only occurs in a few reviews being given a high accuracy score as the method, as it overfit to only predict negative instances.%This is basically copy pasted.
\smallskip


\noindent \textbf{Normalized Discounted Cumulative Gain} % MATHEMATICS NEEDS TO BE REWRITTEN
This is a standard metric in information retrieval which evaluates the quality of a ranking w.r.t.\ some given relevance scores \cite{jarvelin2002cumulated}. It favours initial documents over later ones.  Some alternative metrics were tried that did not prioritize the top rankings being correct more, but this came with two problems. First, PPMI has a large number of zero scores. This makes the lower dot product documents have an uneven comparison, disrupting the score based on them being given a non-zero ranking score by the method. The second is that the documents without many occurrences of the word are less prioritized in the space, and largely influenced by other words, making their ranking less reliable.  In our case, the rankings $r_d$ of the document $d$ are those induced by the dot products $v_w \cdot d$ and the relevance scores are determined by the Pointwise Positive Mutual Information (PPMI) score $\textit{PPMI}(w,d)$, of the word $w$ in the BoW representation of entity $d$ where
$\textit{PPMI}(w,d) = \max \big(0, \log\big(\frac{p_{wd}}{p_{w*} \cdotp p_{*d}}\big)\big)$, and
\begin{align*}
p_{wd} &= \frac{n(w, d)}{\sum_{w'} \sum_{d'} n(w', d')}
\end{align*}
where $n(w,d)$ is the number of occurrences of $w$ in the BoW representation of object $d$, $p_{w*} = \sum_{e'} p_{wd'}$ and $p_{*d} = \sum_{w'} p_{w'd}$. %This is basically copy pasted.
\smallskip

By scoring the words on these features, a simple cut-off is applied (e.g. the top 2000 scored words) to obtain the most salient words. Ideally, this cut-off would be at the point where the words stop corresponding to salient features. However, it is difficult to determine this, so in practice this value is taken as a hyper-parameter.

In principal, NDCG should be better suited for gradual features. In practice, however, there was not such a clear pattern in the differences between the words chosen by these metrics despite often finding different words. Put another way, it is difficult to say if the words highly scored by NDCG are more gradual than other scoring metrics.%We show the top scoring terms found by each scoring type below in Table \ref {X}.

\subsection{Labelling Words}\label{ch3:LabellingWords}
Although the rankings of single words are informative for models, it is difficult for a human to grasp the meaning of a word without context. This can be resolved simply by finding the $n$ most similar directions to each word's direction.

Another approach is to use a clustering method like k-means. For these clustering method, the aim is to go from single word directions $D_w$ to clusters of these single word directions $C_d$ labelled by the words clustered together $C_w$.
If we consider two directions, "Blood" and "Gore", both of these are approximating a similar feature of movies, they both relate to how much blood a movie contains. Because of this, their directions will be very similar to each other. This is the first idea behind using a clustering method on these directions. It  resolves the issue of repetition in the directions, and if the clustered directions are averaged then that clustered direction will balance between documents that used the word 'Bloody' to describe the bloodiness of the movie and the word 'Gore'. Some films may be 'Bloody', but may not necessarily have the term 'Gore' in their reviews, and vice versa. Or, a review may favour one term over the other. By using a clustering method, a direction could be obtained that more accurately represents the semantics of a bloody film more than either of the terms individually. 

It is not always the case that this new clustered direction will perform better than a single relevant direction for a class. In fact, its possible that when clustering many terms together, the ranking can be more disrupted than helped. For example given a cluster  $\{Romance, Love\}$ and a cluster $\{Blood, Gore\}$ the direction for $\{Cute\}$ is clearly more relevant to the former rather than the latter, and likely has been used in the reviews for romance movies. But it has also likely been used in reviews for movies containing cute animals. This would make the new clustered direction $\{Romance, Love, Cute\}$ perform worse at classifying the movie genre "Romance", but a bit better at classifying animal movies. Ideally, this feature would form a new cluster - but a balance must be held between retaining the precision of the rankings and introducing new rankings that are appropiately disentangled from the existing ones, without repeating existing concepts. In the quantitative results, sometimes clustering performed worse than single directions, and not being able to find this balance for the specific classes in question can be attributed as to why.
%What are we doing with clustering
 %Similarly, obtaining a hyper plane using a Logistic Regression classifier that uses occurences of both and either of these terms as positive would be similar to this averaged direction.

% When is clustering useful? When does clustering outperform single directions? When are single directions useful?

%What is the value of a cluster label?


%The final benefit to clustering the words is that linear classifiers are generally suited better to 'disentangled' representations \cite{Bengio2012}. In this case, we refer to disentanglement in the sense of obtaining a feature vector where each dimension is distinct, rather than the Vector Space Model being naturally clustered. Additionally, if our representation is dense and disentangled into the natural features of the domain, it is unlikely to overfit and will be able to generalize more easily. 

The previous work's clustering method is used, and additionally k-means is experimented with:

\noindent \textbf{Derrac's K-Means Variation} This is the clustering method used in the work this method was introduced in \cite{derrac2015}. As input to the clustering algorithm, it considers the $N$ best-scoring candidate feature directions $v_w$, where $N$ is a hyperparameter. The main idea underlying their approach is to select the cluster centers such that (i) they are among the top-scoring candidate feature directions, and (ii) are as close to being orthogonal to each other as possible. 
 
The output of this step is a set of clusters $C_1,...,C_K$, where each cluster $C_j$ is identified with a set of words.
Furthermore  $v_{C_j}$ will be written to denote the centroid of the directions corresponding to the words in the cluster $C_j$, which can be computed as $v_{C_j}= \frac{1}{|C_j|} \sum_{w_l\in C_j} v_l$ provided that the vectors $v_w$ are all normalized. These centroids $v_{C_1},...,v_{C_k}$ are the feature directions that are identified by the method. 

The first cluster centroid is chosen by taking the top-scoring direction for its associated metric. Then, centroids are selected until the desired amount is reached by taking the maximum of the summed absolute cosine similarity of all current centroids, in other words taking the most dissimilar direction to all of the current directions. Once the centroids are selected, for each remaining direction the centroid is found it is most similar to, and the centroid is updated once the direction has been added. 

Cluster centroids are taken as cluster directions, and the representation is obtained by ranking documents on this cluster direction. It is also possible to rank documents on the initial direction only, and use the cluster names as descriptions if the clusters are too noisy.

\noindent \textbf{K-Means} Although the previous method does have a method for selecting cluster centres, typically it was found that it relies too much on its initial directions, meaning if a noisy direction is chosen as the first cluster centre, then key directions may be missed. Avoiding this is difficult without extensive and sometimes arbitrary hyper-parameter optimization. For this reason, it was decided to try K-Means as a clustering algorithm. K-means traditionally begins with $K$ centroids $c$ randomly placed into the space. In our case, these centers are weighted according to the squared distance from the closest center already chosen. \cite{Arthur} %This intialization is improved by XXX method.  
Then, the distance between each point $d_p$ and centroid $c$ is calculated. In-order for euclidian distance to be meaningful, directions are normalized making euclidian distance the same as cosine similarity. Each point $p$ is then assigned to its closest centroid $c$. Then, the centroids are recomputed to be the mean of their assigned points. This process starting with the distance calculation is repeated until the points assigned to the centroids do not change. 

%We show some examples of clusters from each of these methods in Table \ref {XY}.

%Although we are able to find the words that are most salient, the best features in the domain may not correspond directly to words. Further, the features may not be well described by their associated word. In-order to find better representations of properties, we cluster together similar vectors $v_w$, following the assumption that those vectors which are similar are representing some property more general than their individual words, and we can find it between them.

%As the final step, we cluster the best-scoring candidate feature directions $v_w$. Each of these clusters will then define one of the feature directions to be used in applications. The purpose of this clustering step is three-fold: it will ensure that the feature directions are sufficiently different (e.g.\ in a space of movies there is little point in having \emph{funny} and \emph{hilarious} as separate features), it will make the features easier to interpret (as a cluster of terms is more descriptive than an individual term), and it will alleviate sparsity issues when we want to relate features with the BoW representation, which will play an important role for the fine-tuning method described in the next section.


%Table \ref{tabKappaNDCG} displays some examples of clusters that have been obtained for three of the datasets that will be used in the experiments, modelling respectively movies, place-types and newsgroup postings. For each dataset, we used the scoring function that led to the best performance on development data(see Section \ref{secExperiments}). Only the first four words whose direction is closest to the centroid $v_C$ are shown.
%\noindent \textbf{K-Means}
%\noindent \textbf{Derrac's K-Means Variation}
%\noindent \textbf{Mean-shift}
%\noindent \textbf{Hdbscan}

%Our overall aim is to find directions in the Vector Space Model that model salient features of the considered domain. For example, given a Vector Space Model of movies, we would like to find a direction that models the extent to which each movie is scary, among others. Such a direction would then allow us to rank movies from the least scary to the most scary. We will refer to such directions as \emph{feature directions}. Formally, each feature direction will be modelled as a vector $v_f$. However, we refer to \emph{directions} rather than \emph{vectors} to emphasize their intended ordinal meaning: feature directions are aimed at ranking objects rather than e.g.\ measuring degrees of similarity. 

\section{Qualitative Results}

\subsection{Datasets}\label{ch3:datasets}

The experiments are using five different domains. To begin, the properties of these domains are explained to try to give an insight into the kind of text stored within them. This is to better inform analysis of our qualitative results. Examples are shown in three domains in Table \ref{ch3:TextExamples}.

\begin{table}[] 
	\scriptsize
	\begin{tabular}{lp{6.75cm}p{6.75cm}}
		Data Type  & Unprocessed                                                                                                                                                                                                                                                                                                                                                                               & Processed       \\
		\midrule[\heavyrulewidth]
		Newsgroups & morgan and guzman will have era's 1 run higher than last year, and  the cubs will be idiots and not pitch harkey as much as hibbard.  castillo won't be good (i think he's a stud pitcher)                                                                                                                                                                                                & morgan guzman eras run higher last year cubs idiots pitch harkey much hibbard castillo wont good think hes stud pitcher                            \\
		Sentiment  & All the world's a stage and its people actors in it--or something like that. Who the hell said that theatre stopped at the orchestra pit--or even at the theatre door? 
		Why is not the audience participants in the theatrical experience, including the story itself?<br /><br />This film was a grand experiment that said: "Hey! the story is you and it 
		needs more than your attention, it needs your active participation"". ""Sometimes we bring the story to you, sometimes you have to go to the story.""<br /><br />Alas no one listened, 
		but that does not mean it should not have been said." & worlds stage people actors something like hell said theatre stopped orchestra pit even theatre door audience participants
		theatrical experience including  story film grand experiment said hey story needs attention needs active participation sometimes bring story sometimes go story alas one listened mean
		said \\
		Reuters    & U.K. MONEY MARKET SHORTAGE FORECAST REVISED DOWN The Bank of England said it had revised its forecast of the shortage in the money market down to 450 mln stg before taking account of its morning operations. At noon the bank had estimated the shortfall at 500 mln stg.                                                                                                               & uk money market shortage forecast revised bank england said revised forecast shortage money market 450 mln stg taking account morning operations noon bank estimated shortfall 500 mln stg     \\
		
	\end{tabular}
	\caption{Text examples from the first three domains}\label{ch3:TextExamples}
\end{table}

\textbf{20 Newsgroups\footnote{http://qwone.com/~jason/20Newsgroups/}} Obtained from scikit-learn. \footnote{https://scikit-learn.org/0.19/modules/generated/sklearn.datasets.fetch\textunderscore20newsgroups.html\#sklearn.datasets.fetch\textunderscore20newsgroups} Where documents are discussions from one of twenty different groups, specifically Atheism, Computer Graphics, Microsoft Windows, IBM PC Hardware, Mac Hardware, X-Window (GUI Software), Automobiles, Motorcycles, Baseball, Hockey, Cryptography, Electronics, Medicine, Space, Christianity, Guns, The Middle East, General Politics and General Religion. These also act as the classes for the dataset. Originally containing 18,846 documents, in this work it is preprocessed using sklearn to remove headers, footers and quotes. Then, empty and duplicate documents are removed, resulting in 18302 documents. The vocabulary size (unique words) is 141,321. The data is not shuffled. After filtering out terms that did not occur in at least two documents, ending up with a vocabulary of size 51,064. The number of positive instances averaged across all classes is 942, around 5\%.

\textbf{IMDB Sentiment} Obtained from Keras \footnote{https://keras.io/datasets/} Where documents are IMDB movie reviews, containing 50,000 documents with a vocabulary size of 78588. After removing terms that did not occur in at least two documents, ending up with a vocabulary of size 55384. This is a smaller change than the newsgroups, which began with a larger vocabulary than sentiment, but ended vocabularies about the same. This means that newsgroups contained many terms that were not relevant to a majority of the documents, likely because the 20 different newsgroups spread across so many topics. The corpus is split half and half between positive and negative reviews, with the task being to identify the sentiment of the review, so the number of positive instances in the classes is 25,000.

\textbf{Reuters-21578, Distribution 1.0} Obtained from NLTK\footnote{https://www.nltk.org/book/ch02.html}. Documents from the Reuters financial newswire service in 1987,  originally containing 10788 documents. After removing empty and duplicate documents, ending up with 10655 documents. It originally contained 90 classes, but as they were extremely unbalanced all classes that did not have at least 100 positive instances were removed, resulting in 21 classes. These classes are Trade, Grain, Natural Gas (nat-gas), Crude Oil (crude), Sugar, Corn, Vegetable Oil (veg-oil), Ship, Coffee, Wheat, Gold, Acquisitions (acq), Interest, Money/Foreign Exchange (money-fx), Soybean, Oilseed, Earnings and Earnings Forecasts (earn), BOP, Gross National Product (gnp), Dollar (dlr) and Money-Supply.   The original vocabulary size is 51,0001, and after removing all words that do not occur in at least two documents, the vocabulary size is 22542. The number of positive instances averaged across all classes is 541, around 5\%. 

\textbf{Placetypes} Taken from work by Derrac \cite{Derrac2015}. Documents are composed of concatenated flickr tags, where each document, named after a flickr tag, is composed of all flickr tags where that tag occurred. A minimum of 1,000 photos for each tag was a requirement, and the tags selected were taken from three different taxonomies (Geonames, Foursquare and the site category for the common-sense knowledge base OpenCYC).  It originally has a vocabulary size of 746,527 and 1383 documents. This is a very large vocabulary size to document ratio. The end vocabulary for this space was 100,000, which is used as a hard limit. This is roughly equivalent to removing all documents that would not be in at least 6 documents. As most classes in this domain are extremely sparse (less than 100 positive instances) no classes are deleted. There are three tasks, generated from three different place type taxonomies. The Foursquare taxonomy, classifying the 9 top-level categories from Foursquare in September 2013, Arts and Entertainment, College and University, Food, Professional and Other Places, Nightlife Spot, Parks And Outdoors, Shops and Service, Travel and Transport and Residence. the GeoNames taxonomy where 7 of 9 categories were used, Stream/Lake, Parks/Area, Road/Railroad, Spot/Building/Farm, Mountain/Hill/Rock, Undersea, and Forest/Heath. The OpenCYC Taxonomy, where 93 categories were used by Derrac, but it was only possible to match 25 of those classes to the representations. As 8 of these remaining classes had a low number of positive occurrences, OpenCYC classes are removed that do not have positive instances for at least 30 documents, leaving us with 17, Aqueduct, Border, Building, Dam, Facility, Foreground, Historical Site, Holy Site, Landmark, Medical Facility, Medical School, Military Place, Monsoon Forest, National Monument, Outdoor Location, Rock Formation, and Room. Naturally as these tasks were derived from taxonomies they are multi-label.

\textbf{Movies} Taken from work by Derrac \cite{Derrac2015}. A dataset where each document is a movie represented by all of its reviews concatenated across a number of sources (Rotten Tomatoes, IMDB, Amazon Reviews). It starts off with a vocabulary size of 551,080 and a document size of 15,000. However, after investigating the data made available by the authors, it was found that there were a number of duplicate documents. After removing these duplicate documents, there are 13978 documents. In the same way as the place-types, the vocabulary is limited at size 100,000. Three tasks are used to evaluate, 23 movie genres, specifically Action, Adventure, Animation, Biography, Comedy, Crime, Documentary, Drama, Family, Fantasy, Film-Noir, History, Horror, Music, Musical, Mystery, Romance, Sci-Fi, Short, Sport, Thriller, War, Western. 100 of the most common IMDB plot keywords (See Appendix \ref{app:ClassNames}) and Age Ratings from the UK and US, USA-G, UK-12-12A, UK-15, UK-18, UK-PG, USA-PG-PG13, USA-R.


 For each domain, we filter out terms that do not occur in at least two documents, and additionally limit the maximum number of words in a vocabulary to 100,000. For all of these datasets, we split them into a 2/3 training data, 1/3 test data split. We additionally remove the end 20\% of the training data and use that as development data for our hyper-parameters, which is then not used for the final models verified using test data.  For the movies and place-type domains, the original text was not available.



\subsection{Space Types}

Below the choices for the Vector Space Models that are formally described in Section \ref{bg:SemanticSpaces} are explained:

\textbf{Multi-Dimensional Scaling (MDS)}: Following \cite{derracAIJ}, we use Multi-Dimensional Scaling (MDS) to learn semantic spaces from the angular differences between the PPMI weighted BoW vectors. %A non-linear transformation that is used to evaluate the quality of representations when built from a standard BOW-PPMI. Chosen as it performed well in the work introducing this method.

\textbf{Principal Component Analysis (PCA)}: directly uses the PPMI weighted BoW vectors as input, and which avoids the quadratic complexity of the MDS method. A standard dimensionality reduction technique, used as a baseline reference.

\textbf{Doc2Vec (D2V)}: Inspired by the Skipgram model \cite{DBLP:conf/icml/LeM14}.  A distributional document representation used as a representative of a higher performing method of learning in terms of document classification. For the Doc2Vec space, the hyper-parameters are additionally tuned for the $window size (5, 10, 15)$ referring to the context window, the $min count (1, 5, 10)$ referring to the minimum frequency of words and the $epochs (50, 100, 200)$ of the network for each size space. The process with our two-part hyperparameter optimization as in this case is as follows: Grid search is used to select the parameters for the representation, then find the most suitable model (e.g. Decision Tree, SVM) for that representation. 

\textbf{Average Word Vectors (AWV)}: Finally, we also learn semantic spaces by averaging word vectors, using a pre-trained GloVe word embeddings trained on the Wikipedia 2014 + Gigaword 5 corpus\footnote{\url{https://nlp.stanford.edu/projects/glove/}}. While simply averaging word vectors may seem naive, this was found to be a competitive approach for unsupervised representations in several applications \cite{DBLP:conf/naacl/HillCK16}. We simply average the vector representations of the words that appear at least twice in the BoW representation.


\subsection{The best-performing directions for each domain}

To give an understanding of the kind-of directions found for each domain, the top-scoring ones are presented in Table \ref{ch3:TopScoringQua}. These are arranged from highest scoring to least scoring, with the score-type and space-type chosen by performance. These are not clusters, but rather single directions with the two most similar directions in brackets beside them for context. This is the alternative way of presenting these directions as mentioned at the start of Section \ref{ch3:LabellingWords}. 

There is an  interesting difference between the sentiment directions and the movies directions in the examples below. Both of these domains are composed of movie reviews, but the documents in the former are a concatenation of a number of reviews across different sources, while the latter are individual reviews. This has resulted in the more general concepts that apply to many movies being salient in the movies domain, but are less important than the names of actors and actresses in the sentiment domain. This is likely because the PPMI scores for actor names would be high as they are both rare and definitive for movies. For the newsgroups domain, a number of directions  are seen that are likely to only belong to a certain newsgroups, e.g. you would find the word 'celestial' more often in the religious sections than the others, and the word 'diesel' more often in the automobile section but not others. This is an expected natural clustering of the domain into its 20 newsgroups. The place-types section generally describes either aspects of the camera (e.g. canon60d), aspects of the photo (greyscale) or features found in the photo (gardening). The former likely relates to the degree to which filters or editing has been applied to the photo, while the latter makes more sense for our classification task. For the reuters dataset, the highest scored semantics seem to generally be related to dates (1st, may, june), however there is also some business jargon (quarterly, avg, dlr). 

\begin{landscape}
	\begin{table}[]
		\scriptsize
		\begin{tabular}{lllll}
			\textbf{Movies} \textit{(50 MDS NDCG)}        & \textbf{Sentiment} \textit{(100 D2V NDCG)}   & \textbf{Newsgroups} \textit{(50 D2V NDCG)} 			  & \textbf{Place-types} \textit{(50 PCA Kappa)}	 						 & \textbf{Reuters} \textit{(200 MDS NDCG)}       \\
			horror (scares, scary)               & glenda (glen, matthau)         & karabag (iranian, turkiye)                 & blackcountry (listed, westmidlands)     & franklin (fund, mthly)            \\
			hilarious (funniest, hilarity)       & scarlett (gable, dalton)       & leftover (flaming, vancouver)              & ears (stare, adorable)                  & quarterly (shearson, basis)       \\
			bollywood (hindi, india)             & giallo (argento, fulci)        & wk (5173552178, 18084tmibmclmsuedu)        & spagna (espanha, colores)               & feb (28, splits)                  \\
			laughs (funnier, funniest)           & bourne (damon, cusack)         & 1069 (mlud, wibbled)                       & oldfashioned (winery, antiques)         & 22 (booked, hong)                 \\
			jokes (gags, laughs)                 & piper (omen, knightley)        & providence (norris, ahl)                   & gardening (greenhouse, petals)          & april (monthly, average)          \\
			comedies (comedic, laughs)           & casper (dolph, damme)          & celestial (interplanetary, bible)          & pagoda (hindu, carved)                  & sets (principally, precious)      \\
			hindi (bollywood, india)             & norris (chuck, rangers)        & mlud (wibbled, 1069)                       & artificial (saturation, cs4)            & 16 (creditor, trillion)           \\
			war (military, army)                 & holmes (sherlock, rathbone)    & endif (olwm, ciphertext)                   & inner (curved, rooftops)                & 1st (qtr, pennsylvania)           \\
			western (outlaw, unforgiven)         & rourke (mickey, walken)        & gd3004 (35894, intergraph)                 & celebrate (festive, celebrity)          & 26 (approve, inadequate)          \\
			romantic (romance, chemistry)        & ustinov (warden, cassavetes)   & rtfmmitedu (newsanswers, ieee)             & vietnamese (ethnic, hindu)              & 23 (offsetting, weekly)           \\
			songs (song, tunes)                  & scooby (doo, garfield)         & eng (padres, makefile)                     & cn (elevated, amtrak)                   & prior (recapitalization, payment) \\
			sci (science, outer)                 & doo (scooby, garfield)         & pizza (bait, wiretap)                      & mannequin (bags, jewelry)               & avg (shrs, shr)                   \\
			funniest (hilarious, funnier)        & heston (charlton, palance)     & porsche (nanao, mercedes)                  & falcon (r, 22)                          & june (july, venice)               \\
			noir (noirs, bogart)                 & homer (pacino, macy)           & gebcadredslpittedu (n3jxp, skepticism)     & jewish (monuments, cobblestone)         & march (31, day)                   \\
			documentary (documentaries, footage) & welles (orson, kane)           & scsi2 (scsi, cooling)                      & canon60d (kitlens, 600d)                & regular (diesel, petrol)          \\
			animation (animated, animators)      & frost (snowman, damme)         & playback (quicktime, xmotif)               & reflective (curved, cropped)            & 4th (qtr, fourth)                 \\
			adults (adult, children)             & streisand (bridget, salman)    & 35894 (gd3004, medin)                      & mason (edward, will)                    & 27 (chemlawn, theyre)             \\
			creepy (spooky, scary)               & davies (rhys, marion)          & diesel (volvo, shotguns)                   & aerialview (manmade, largest)           & 14 (borrowing, borrowings)        \\
			gay (gays, homosexuality)            & cinderella (fairy, stepmother) & evolutionary (shifting, hulk)              & shelf (rack, boxes)                     & 11 (chapter, ranged)              \\
			workout (intermediate, instruction)  & boll (uwe, belushi)            & techniciandr (obp, 144k)                   & monroe (raleigh, jefferson)             & may (probably, however)           \\
			thriller (thrillers, suspense)       & rochester (eyre, dalton)       & 8177 (obp, 144k)                           & litter (fujichrome, e6)                 & 38 (33, strong)                   \\
			funnier (laughs, funniest)           & edie (soprano, vertigo)        & shaw (medicine, ottoman)                   & streetlights (streetlamp, headlights)   & m1 (m2, m3)                       \\
			suspense (suspenseful, thrillers)    & scarecrow (zombies, reese)     & scorer (gilmour, lindros)                  & carlzeiss (f2, voigtlander)             & dlr (writedown, debt)             \\
			arts (hong, chan)                    & kramer (streep, meryl)         & xwd (xloadimage, openwindows)              & manmade (aerialview, below)             & five (years, jones)               \\
			christianity (religious, religion)   & marty (amitabh, goldie)        & ee (275, xloadimage)                       & demolished (neglected, rundown)         & bushels (soybeans, ccc)           \\
			musical (singing, sing)              & columbo (falk, garfield)       & com2 (com1, v32bis)                        & wald (berge, wildflower)                & revs (net, 3for2)                 \\
			gore (gory, blood)                   & kidman (nicole, jude)          & examiner (corpses, brass)                  & arquitetura (exposition, cidade)        & 29 (175, include)                 \\
			animated (animation, cartoon)        & juliet (romeo, troma)          & migraine (ama, placebo)                    & greyscale (highcontrast, monochromatic) & acquisition (make, usairs)        \\
			gags (jokes, slapstick)              & garland (judy, lily)           & parliament (parliamentary, armored)        & alameda (monday, marin)                 & payable (div, close)              \\          
			
		\end{tabular}
		\caption{The top-scoring words for each domain, scoring metric and space type determined by the highest F1-score}\label{ch3:TopScoringQua}
	\end{table}
\end{landscape}





\subsection{Comparing Space Types}

To select these quantitative examples for  comparing score types, it was first demonstrated on the movies domain to be consistent with previous examples. However, as this does not contain the doc2vec space, additional results are provided in the next section for the newsgroups. The space that performed well on the genres task for the movies is used, with the understanding that genres as a key natural classification task will likely give good example directions that correspond to domain knowledge. After selecting this space, the same sized spaces are chosen from the other space-types (size 200). The same score-type and frequency cut-off  as the best performing space-type are also used. In this case, the best performing type for the PCA space was 20,000 frequency cutoff and NDCG. So even though sometimes a different frequency cut-off performed better for the other space-types, this is equalized so that the words are the same. This means that sometimes the space-type is  a slightly worse performing one than chosen as the final results, and that the original space has a performance advantage, but this makes the results more consistent. These qualitative experiments are approached with the following idea: spaces that perform better on natural domain tasks using Decision Trees contain unique natural directions that other spaces do not have. 

The commonalities between spaces are much more prevalent than the differences, with natural concepts of the domain being represented in all of the different space types. However, different spaces do perform better than others on natural domain tasks. For this reason, the directions which are unique to each space-type are shown.

When examining the table of results, it can be observed that the common terms are mostly salient concepts relevant to the domain. However, MDS has the most unique general concepts relevant to the domain that others do not have. AWV contains names, and concepts which are interesting but more related to specific aspects than genre (train, slaves). Meanwhile PCA seems to prioritize words in the reviews that are not concepts but rather parts of sentences (surprisingly, admit, talents, tired, anymore). However, both PCA and MDS contain unique noisy terms as well. The term 'berardinelli' and 'rhodes' for MDS as well as 'compuserve' for PCA are artifacts of the data being obtained from the web. Despite this, it seems that MDS does contain more interesting unique directions than PCA, and as it performed best on the genres task, this makes sense.

\begin{landscape}
	\begin{table}[]
		\scriptsize
		\begin{tabular}{llll}
			MDS                                   & AWV                      & PCA                                     & Common                                  \\
			berardinelli \textit{(employers, distributor)} & billy \textit{(thrown, dirty)}    & amount \textit{(leaving, pick)}                  & noir \textit{(fatale, femme)}                    \\
			crawford \textit{(joan, davis)}                & brother \textit{(brothers, boys)} & fails \textit{(fit, pick)}                       & gay \textit{(homosexual, homosexuality)}         \\
			hitchcocks \textit{(hitchcock, alfred)}        & fonda \textit{(henry, jane)}      & pick \textit{(fails, fit)}                       & prison \textit{(jail, prisoners)}                \\
			warners \textit{(warner, bros)}                & building \textit{(built, climax)} & stands \textit{(fails, cover)}                   & arts \textit{(rec, robomod)}                     \\
			nuclear \textit{(weapons, soviet)}             & train \textit{(tracks, thrown)}   & surprisingly \textit{(offer, fit)}               & allens \textit{(woody, allen)}                   \\
			joan \textit{(crawford, barbara)}              & slaves \textit{(slavery, excuse)} & copyright \textit{(email, compuserve)}           & jokes \textit{(laughs, joke)}                    \\
			kidnapped \textit{(kidnapping, torture)}       &                          & length \textit{(reflect, expressed)}             & animation \textit{(animated, cartoon)}           \\
			hop \textit{(hip, rap)}                        &                          & profanity \textit{(reflect, producers)}          & sherlock \textit{(holmes, detective)}            \\
			kung \textit{(martial, jackie)}                &                          & compuserve \textit{(copyright, internetreviews)} & western \textit{(westerns, wayne)}               \\
			ballet \textit{(dancers, dancer)}              &                          & talents \textit{(admit, agree)}                  & songs \textit{(song, lyrics)}                    \\
			gambling \textit{(vegas, las)}                 &                          & admit \textit{(agree, talents)}                  & comedies \textit{(comedic, laughs)}              \\
			alcoholic \textit{(drunk, alcoholism)}         &                          & developed \textit{(introduced, sounds)}          & workout \textit{(exercise, challenging)}         \\
			waves \textit{(surfing, wave)}                 &                          & intended \textit{(bother, werent)}               & laughs \textit{(funnier, hilarious)}             \\
			jaws \textit{(jurassic, godfather)}            &                          & constantly \textit{(putting, sounds)}            & drug \textit{(drugs, addict)}                    \\
			jungle \textit{(natives, island)}              &                          & tired \textit{(anymore, mediocre)}               & sci \textit{(science, fiction)}                  \\
			employers \textit{(berardinelli, distributor)} &                          & produced \textit{(spoiler, surprising)}          & documentary \textit{(documentaries, interviews)} \\
			pot \textit{(weed, stoned)}                    &                          & involving \textit{(believes, belief)}            & students \textit{(student, schools)}             \\
			canadian \textit{(invasion, cheap)}            &                          & anymore \textit{(continue, tired)}               & thriller \textit{(thrillers, suspense)}          \\
			murphy \textit{(eddie, comedian)}              &                          & leaving \textit{(fit, pick)}                     & allen \textit{(woody, allens)}                   \\
			comics \textit{(comedian, comedians)}          &                          & makers \textit{(producers, aspects)}             & funniest \textit{(hilarious, laughing)}          \\
			kidnapping \textit{(kidnapped, torture)}       &                          & introduced \textit{(developed, considered)}      & gags \textit{(jokes, slapstick)}                 \\
			subscribe \textit{(email, internetreviews)}    &                          & loses \textit{(climax, suffers)}                 & adults \textit{(children, adult)}                \\
			vegas \textit{(las, gambling)}                 &                          & negative \textit{(positive, bother)}             & animated \textit{(animation, cartoon)}           \\
			distributor \textit{(berardinelli, employers)} &                          & expressed \textit{(reflect, opinions)}           & dancing \textit{(dance, dances)}                 \\
			wave \textit{(waves, surfing)}                 &                          & mildly \textit{(mediocre, forgettable)}          & teen \textit{(teenage, teens)}                   \\
			rhodes \textit{(internetreviews, email)}       &                          & helped \textit{(putting, allowed)}               & soldiers \textit{(soldier, army)}                \\
			hippie \textit{(pot, sixties)}                 &                          & reflect \textit{(expressed, opinions)}           & indie \textit{(independent, festival)}           \\
			weed \textit{(pot, stoned)}                    &                          & opinions \textit{(reflect, expressed)}           & suspense \textit{(suspenseful, thriller)}        \\
			caribbean \textit{(pirates, island)}           &                          & frequently \textit{(occasionally, consistently)} & creepy \textit{(scary, eerie)}                   \\
			eddie \textit{(murphy, comedian)}              &                          & content \textit{(agree, proves)}                 & italian \textit{(italy, spaghetti)}              \\
			sixties \textit{(beatles, hippie)}             &                          & allowed \textit{(helped, werent)}                & jews \textit{(jewish, nazis)}                    \\
			... 8 More                   		  &                          & suffers \textit{(lacks, loses)}                  & ... 1480 more               
		\end{tabular}
		\caption{Unique terms between space-types}\label{ch3:ComparingSpaceTypes}
	\end{table}
\end{landscape}

\subsubsection{Score Types}

There are unique directions for each different space type from the movies domain, each suitable to different tasks. Obtained in the same way as before, this time the 200 MDS space is used that performed the best on the genres task and found those unique to it. Once again, the most understandable and general concepts are those that are common to all score-types. NDCG performed the best on most tasks, and it can be seen that a lot of new concepts are introduced in NDCG compared to the other scoring types. F1 by and large seems is difficult to understand, referring to names or specific aspects of the scene, and accuracy is similar. Kappa has some unique sentiment related terms, as as well as some aspects of the presentation of the film (featurette, critic, technical), but it does not contain unique general concepts the way NDCG does. It can be surmised that as NDCG contains these unique conceptual directions, it is able to perform better than other score-types.
\begin{landscape}
	\begin{table}[]
		\scriptsize
		\begin{tabular}{lllll}
			\textbf{NDCG}                     & \textbf{F1}            		     & \textbf{Accuracy}    				   & \textbf{Kappa}                     & \textbf{Common}                               \\
			gay \textit{(homosexuality, sexuality)}    & company \textit{(sell, pay)}              & kennedy \textit{(republic, elected)}             & definately \textit{(alot, awesome)}         & horror \textit{(scares, scares)}              \\
			arts \textit{(hong, chan)}                 & street \textit{(city, york)}              & bags \textit{(listened, salvation)}              & guns \textit{(gun, shoot)}                  & laughs \textit{(funnier, funnier)}            \\
			sports \textit{(win, players)}             & red \textit{(numerous, fashion)}          & summers \textit{(verge, medieval)}               & flawless \textit{(perfection, brilliantly)} & jokes \textit{(gags, gags)}                   \\
			apes \textit{(remembered, planet)}         & project \textit{(creating, spent)}        & revolve \textit{(sincerely, historian)}          & mail \textit{(reviewed, rated)}             & comedies \textit{(comedic, comedic)}          \\
			german \textit{(germans, europe)}          & mark \textit{(favor, pull)}               & locale \textit{(foster, sharply)}                & garbage \textit{(crap, horrible)}           & sci \textit{(scifi, alien)}                   \\
			satire \textit{(parody, parodies)}         & lady \textit{(actress, lovely)}           & cooler \textit{(downward, reports)}              & featurette \textit{(featurettes, extras)}   & funniest \textit{(hilarious, hilarious)}      \\
			band \textit{(rock, vocals)}               & fire \textit{(ground, force)}             & spades \textit{(ralph, medieval)}                & complaint \textit{(extra, added)}           & creepy \textit{(spooky, spooky)}              \\
			crude \textit{(offensive, offended)}       & post \textit{(essentially, purpose)}      & filmography \textit{(ralph, experiments)}        & mission \textit{(enemy, saving)}            & thriller \textit{(thrillers, thrillers)}      \\
			dancing \textit{(dance, dances)}           & heads \textit{(large, throw)}             & quentin \textit{(downward, anime)}               & ruin \textit{(wondering, heck)}             & funnier \textit{(laughs, laughs)}             \\
			restored \textit{(print, remastered)}      & water \textit{(land, large)}              & employers \textit{(finishes, downward)}          & wars \textit{(forces, enemy)}               & suspense \textit{(suspenseful, suspenseful)}  \\
			drugs \textit{(drug, abuse)}               & road \textit{(drive, trip)}               & formal \textit{(victory, kennedy)}               & prefer \textit{(compare, added)}            & gore \textit{(gory, gory)}                    \\
			church \textit{(religious, jesus)}         & brother \textit{(son, dad)}               & tube \textit{(esta, muscle)}                     & heroes \textit{(packed, hero)}              & gags \textit{(jokes, jokes)}                  \\
			sexuality \textit{(sexual, sexually)}      & party \textit{(decide, hot)}              & woefully \textit{(restless, knockout)}           & necessarily \textit{(offer, draw)}          & science \textit{(sci, sci)}                   \\
			sexually \textit{(sexual, sexuality)}      & badly \textit{(awful, poorly)}            & scientists \textit{(hilarity, locale)}           & portray \textit{(portrayed, portraying)}    & gory \textit{(gore, gore)}                    \\
			england \textit{(british, english)}        & limited \textit{(aspect, unlike)}         & overboard \textit{(civilized, cinderella)}       & critic \textit{(reviewed, net)}             & government \textit{(political, political)}    \\
			ocean \textit{(sea, boat)}                 & impression \textit{(instance, reasons)}   & rumors \textit{(homosexuality, characteristics)} & reviewed \textit{(rated, mail)}             & suspenseful \textit{(suspense, suspense)}     \\
			marry \textit{(married, marriage)}         & trip \textit{(journey, road)}             & salvation \textit{(bags, cooler)}                & saving \textit{(carry, forced)}             & frightening \textit{(terrifying, terrifying)} \\
			campy \textit{(cult, cheesy)}              & michael \textit{(producers, david)}       & actively \textit{(assassination, overcoming)}    & technical \textit{(digital, presentation)}  & military \textit{(army, army)}                \\
			christian \textit{(religious, jesus)}      & memory \textit{(forgotten, memories)}     & stretching \textit{(victory, hideous)}           & statement \textit{(exist, critical)}        & slapstick \textit{(gags, gags)}               \\
			melodrama \textit{(dramatic, tragedy)}     & james \textit{(robert, michael)}          & downward \textit{(cooler, crawling)}             & shocked \textit{(hate, warning)}            & scary \textit{(scare, scare)}                 \\
			sing \textit{(singing, sings)}             & thin \textit{(barely, flat)}              & rocked \textit{(staple, demented)}               & flying \textit{(air, force)}                & blu \textit{(unanswered, ray)}                \\
			sentimental \textit{(touching, sappy)}     & pre \textit{(popular, include)}           & affectionate \textit{(esta, muscle)}             & danger \textit{(dangerous, edge)}           & internetreviews \textit{(rhodes, rhodes)}     \\
			depressing \textit{(bleak, suffering)}     & faces \textit{(constant, unlike)}         & protest \textit{(protective, assassination)}     &                                    & cgi \textit{(computer, computer)}             \\
			evidence \textit{(investigation, accused)} & values \textit{(exception, wise)}         & confined \textit{(cooler, downward)}             &                                    & email \textit{(web, web)}                     \\
			adorable \textit{(cute, sweet)}            & unusual \textit{(odd, seemingly)}         & inhabit \textit{(quentin, drawback)}             &                                    & thrilling \textit{(thrill, exciting)}         \\
			episodes \textit{(episode, television)}    & lovers \textit{(lover, lovely)}           & latin \textit{(communities, mount)}              &                                    & web \textit{(email, email)}                   \\
			teenager \textit{(teen, teenage)}          & frame \textit{(image, effect)}            & reception \textit{(como, finishes)}              &                                    & horror \textit{(scares, scares)}              \\
			magical \textit{(fantasy, lovely)}         & mans \textit{(ultimate, sees)}            & uptight \textit{(suspensful, stalked)}           &                                    & laughs \textit{(funnier, funnier)}            \\
			health \textit{(medical, suffering)}       & efforts \textit{(generally, nonetheless)} & brink \textit{(inexplicable, freddy)}            &                                    & suspense \textit{(suspenseful, suspenseful)} 
		\end{tabular}
		\caption{Different score types}
	\end{table}
\end{landscape}

%We show terms unique to each domain type above. The terms are contextualized by finding the two closest term directions to the term. Here, we show the terms that are not duplicated in meaning for the other spaces. We can understand that the term 'gay (gays, homosexuality) has a similar meaning to the term 'homosexuality (gays, gay)' despite the term 'homosexuality' being high scored for one space, and the term 'gay' being highly scored for the other space. Almost all of the unique terms were of this variety. To demonstrate instead the unique meanings found in the individual spaces, we filter the results as follows: from the term and its associated descriptive terms found by getting the most similar terms, if those terms or any of the more similar terms occur in any of the terms or more similar terms in the space's top 1000 terms that we are comparing to, do not include those terms.

%We find that the 50-dimensional MDS space, which performs 0.04 F1 score higher on the genres task, finds many interesting and unique terms, which can potentially enable more nuanced decisions in the Decision Trees for classification. On the other hand, in the PCA space, we find terms that relate to metadata, and spanish language words. We argue that this means the MDS is better at finding unique interesting meanings than PCA, in the case of using a frequency-based BOW to create the representation.

%We perform the same process as above but comparing MDS and AWV, and find that the PPMI-based representations have their own metadata in the representation that is elevated. We can assume that the AWV space does not contain these metadata as there are simply not word-vectors available for these terms. Although, we also find unique terms that are likely to help performance on natural domain tasks for the MDS space, e.g. goodfellas, disgusting, swashbuckling. Interestingly, for the AWV space we find similar spanish-language terms, but also find some new concepts, in particular the 'republican' and 'yoga' directions. AWV was also originally 20,000 frequency and NDCG.



\subsubsection{Comparing PPMI representations to doc2vec}

Now in Table \label{ch3:compared2vmds} a comparison is shown between a time when doc2vec was the highest performing representation, in this case on the newsgroups domain. Doc2vec is compared to MDS in this case as MDS also performed well. This is to see if doc2vec, by making use of word-vectors and word-context can find interesting unique directions compared to MDS, which was obtained from a PPMI BOW. In general, it is found that MDS contains a lot more irrelevant words than D2V, specifically related to parts-of-words. It seems that doc2vec was better at recognizing these words as noise and uninteresting compared to PPMI, which must have prioritized these words. Doc2Vec also brings up some interesting concepts, e.g. cryptology, which is very relevant to the 20 newsgroup subtype of cryptography. It can be expected that by using word vectors, doc2vec is able to more easily identify interesting words and de-prioritize words which are common to the english language despite potentially being more rare in a smaller dataset.

\begin{table}[]
	\scriptsize
	\begin{tabular}{lll}
		\textbf{D2V}                                    & \textbf{MDS}                              & \textbf{Common}                                     \\
		leftover \textit({pizza, brake)}                & hi \textit({folks, everyone)}             & chastity \textit({shameful, soon)}                  \\
		wk \textit({5173552178, 18084tmibmclmsuedu)}    & looking \textit({spend, rather)}          & n3jxp \textit({gordon, gebcadredslpittedu)}         \\
		eng \textit({padres, makefile)}                 & need \textit({needs, means)}              & skepticism \textit({gebcadredslpittedu, n3jxp)}     \\
		porsche \textit({nanao, 1280x1024)}             & post \textit({summary, net)}              & anyone \textit({knows, else)}                       \\
		diesel \textit({cylinders, steam)}              & find \textit({couldnt, look)}             & gebcadredslpittedu \textit({soon, gordon)}          \\
		scorer \textit({gilmour, lindros)}              & hello \textit({kind, thank)}              & intellect \textit({soon, gordon)}                   \\
		parliament \textit({caucasus, semifinals)}      & david \textit({yet, man)}                 & please \textit({respond, reply)}                    \\
		atm \textit({padres, inflatable)}               & got \textit({mine, youve)}                & thanks \textit({responses, advance)}                \\
		cryptology \textit({attendees, bait)}           & go \textit({take, lets)}                  & email \textit({via, address)}                       \\
		intake \textit({calcium, mellon)}               & question \textit({answer, answered)}      & know \textit({let, far)}                            \\
		433 \textit({366, 313)}                         & interested \textit({including, products)} & get \textit({wait, trying)}                         \\
		ghetto \textit({warsaw, gaza)}                  & list \textit({mailing, send)}             & think \textit({important, level)}                   \\
		lens \textit({lenses, ankara)}                  & sorry \textit({guess, hear)}              & good \textit({luck, bad)}                           \\
		rushdie \textit({sinless, wiretaps)}            & heard \textit({ever, anything)}           & shafer \textit({dryden, nasa)}                      \\
		immaculate \textit({porsche, alice)}            & cheers \textit({kent, instead)}           & bobbeviceicotekcom \textit({manhattan, beauchaine)} \\
		keenan \textit({lindros, bosnian)}              & say \textit({nothing, anything)}          & dryden \textit({shafer, nasa)}                      \\
		boxer \textit({jets, hawks)}                    & number \textit({call, numbers)}           & im \textit({sure, working)}                         \\
		linden \textit({mogilny, 176)}                  & mailing \textit({list, send)}             & sank \textit({bronx, away)}                         \\
		candida \textit({yeast, noring)}                & call \textit({number, phone)}             & banks \textit({soon, gordon)}                       \\
		octopus \textit({web, 347)}                     & thank \textit({thanx, better)}            & like \textit({sounds, looks)}                       \\
		czech \textit({detectors, kuwait)}              & read \textit({reading, group)}            & shameful \textit({soon, gordon)}                    \\
		survivor \textit({warsaw, croats)}              & phone \textit({company, number)}          & could \textit({away, bobbeviceicotekcom)}           \\
		5173552178 \textit({circumference, wk)}         & mail \textit({send, list)}                & would \textit({appreciate, wouldnt)}                \\
		18084tmibmclmsuedu \textit({circumference, wk)} & doesnt \textit({isnt, mean)}              & beauchaine \textit({bobbeviceicotekcom, away)}      \\
		3369591 \textit({circumference, wk)}            & lot \textit({big, little)}                & ive \textit({seen, never)}                          \\
		mcwilliams \textit({circumference, wk)}         & thats \textit({unless, youre)}            & surrender \textit({soon, gebcadredslpittedu)}       \\
		coldblooded \textit({dictatorship, czech)}      & believe \textit({actually, truth)}        & problem \textit({problems, fix)}                    \\
		militia \textit({federalist, occupying)}        & youre \textit({unless, theyre)}           & windows \textit({31, dos)}                          \\
		cbc \textit({ahl, somalia)}                     & send \textit({mail, mailing)}             & gordon \textit({soon, gebcadredslpittedu)}         
	\end{tabular}
	\caption{Comparing an MDS sapce to a D2V space for Newsgroups, where a D2V space performed best.}\ref{ch3:compared2vmds}
\end{table}

%\subsubsection{The best performing clusters from each domain}



%\subsubsection{Comparing clustering methods}

%The previous two experiments were conducted with the IMDB movies dataset, but the doc2vec space is not available for that dataset, as the original text corpus was not made available by the authors. Because of this, we choose to compare the PCA representation for the sentiment space and the Doc2Vec representation for the sentiment space, as these are still IMDB movie reviews, but instead with reviews as documents rather than compilations of reviews as documents, we can expect different directions to be important but still use our knowledge of movies and reviews to inform us. We take the 100 dimensional D2V space and the 100 dimensional PCA space, both of which perform best using the ndcg scoring type which we do not change.

%The assumption going into this analysis was that, in the same way that the AWV space does not contain metadata that is not present in the word-vectors, as will the Doc2Vec space not be informed especially by that metadata. Further, as in this case the D2V space outperforms the PCA space on the sentiment task, we also expect that by using word-context to produce the representation, we are able to find better sentiment information directions, for example by better understanding sarcasm. Additionally, the benefit of word-vectors is likely to play into this space in a positive way, informing the representation based on global context. 

%When looking at these directions, we can split the terms into two types. The names of things, and conceptual properties about movies. For the doc2vec space, the names largely dominate the unique terms list, which we can understand to mean that these terms used contextually and in the broader global context from the word-vectors is informative enough in the doc2vec space to be usefully represented. For the PCA space, there are still these unique names found, but there are less overall terms and it's a bit more balanced between concepts and names. 

%For some additional examples, we compare the MDS representation for the newsgroups and the Doc2Vec representation. The MDS performed best with F1, but we compare it to NDCG, as that is what the doc2vec space performs best with.

\section{Quantitative Results}


\subsection{Evaluation Method}

Primarily the effectiveness of a representation is evaluated on its ability to perform in low-depth Decision Trees, specifically CART Decision Trees (See Background Section \ref{bg:trees}) with a limited depth of one, two and three. This evaluation has a few assumptions: A good interpretable representation disentangles salient domain knowledge into its dimensions, and natural domain tasks (e.g. classifying genres of movies using their reviews) can be evaluated effectively using that salient domain knowledge. Put another way, if the space is representing domain knowledge well it can be expected that the space is linearly separable for key semantics of the domain. In spatial terms, a representation will be capable of being linearly transformed by our method into these distinct relevant concepts if semantically distinct entities are spatially separated, and semantically similar entities are close together.

If only the the quality of the representation was being evaluated, only Linear SVM's could be used to find the hyper-planes that effectively separate these spatial representations for the class. However, the representations that encode this spatial information are not interpretable, so a linear classifier although able to separate the documents that contain the class and do not contain them will not be interpretable either. It is our main interest to evaluate how well a representation encodes these key semantics while also being restricted by the requirement to be disentangled into words or clusters, in other words how well it represents the information while also being interpretable.

Given these assumptions, low-depth Decision Trees can give an estimation of how good an interpretable representation is. If the representation cannot perform for a class at a one-depth tree, then it is not disentangled such that it contains a single salient dimension that effectively evaluates a class. If a representation cannot perform well on two-depth trees, then the representation is not disentangled into three concepts that can sufficiently determine that class, and if a representation cannot perform well on three-depth trees, it has not disentangled the representation such that there are nine relevant concepts that are relevant to that class. To see what these different trees look like see Figure \ref{DecisionTree}. A comparison to put this in better perspective is to an unbounded tree. Unbounded trees select a large amount of dimensions in order to achieve a performance difference on development data, but when applied to test data the models do not generalize well. This is because they overfit, rather than using the key semantics of the space to classify.



Primarily F1-score  is used to determine if a classifier is good or not. This is because many of the classes are unbalanced so accuracy is not a good metric, as high accuracy could be achieved by predicting only zeros.  All of the results shown in this section are the end-product of a two-part hyper-parameter optimization. Each Decision Tree has its own set of hyper-parameters that are optimized as does each representation-type. These are the models trained on the training data and scored on the test data, with the highest performing in terms of F1-score parameters from hyper-parameter optimization on the development data. For ease of comparison, some results are provided with SVM's and unbounded Decision Trees, as well as a baseline Topic Model, which is used as a reference for a standard interpretable representation. Below, the parameters are listed that are optimized for each of these model types:

\textbf{Linear Support Vector Machines (SVM's)}\ref{bg:SVM}: C parameters and gamma parameters.
C {1.0, 0.01, 0.001, 0.0001}, Gamma {1.0, 0.01, 0.001, 0.0001}.

\textbf{Topic Models}\ref{bg:TopicModel}: Two priors: The doc topic prior {0.001, 0.01, 0.1} and the topic word prior {0.001, 0.01, 0.1}

\textbf{CART Decision Trees }\ref{bg:trees}: The number of features to consider when looking for the best split. $None, auto, log2$ and the criterion for a node split $criterion: gini, entropy$.
	
For the baselines, four different Vector Space Models are used, a Bag-Of-Words of PPMI (BOW-PPMI) scores and a standard Latent Dirchlet Allocation (LDA) Topic Model. As well as the original filtering done to the representations, for the BOW-PPMI additionally all terms are filtered out that do not occur in at least $(d_N / 1000)$ documents. Otherwise, there would be too many irrelevant terms to be a fair comparison. The dimension amounts that are compared are of size $(50, 100, 200)$.
The MDS space is not available for sentiment, as the memory cost was too prohibitive with 50,000 documents, and there are no doc2vec spaces for placetypes/movies, as it was only possible access to the Bag-Of-Words representation.

When obtaining the single word directions, starting with all of the baseline representations and vocabularies, the infrequent terms are filtered from these vocabularies according to a hyper-parameter that is tuned. As the doc2vec has already been hyper-parameter optimized, the optimal doc2vec space that scored the highest for its class on a Linear SVM is used, rather than tuning the entire process around the doc2vecs vectors. So for example, when evaluating the Keywords task for the movies, directions are obtained from the doc2vec space that performed best for a linear SVM on the Keywords task following the previous experiments. 


Results are obtained for the rankings induced from these word directions on Decision Tree's limited to a depth-three in-order to select the best parameters when using directions for each class. The parameters that are desirable to determine are the type of Vector Space Model, the size of the space, the frequency threshold and the score threshold, which determines the top scoring directions. To do so, for each space-type of each size, a grid search is used to find the best frequency and score cut-offs for that sized space-type. Then, from these space-types and sizes the best performing one is selected. There is a balance between finding words which are useful for creating salient features in our clustering step without including too many words which do not. As our clustering methods are unsupervised, it is important that to try and limit the amount of junk being entered into them, despite the classifiers that use these directions typically being able to filter out those directions which are not suitable to the class. Additionally, as the vocabulary size varies from dataset to dataset, the threshold will naturally be different for each one. 

These results allow us to choose for each class, the best Vector Space Model and Scoring-type for that class.
Next, we test single directions, attempting to find a good amount of directions to cluster and not including words which may hamper the unsupervised classification, as well as the best space-type for each domain. We found that generally,  classifiers performed better with more data, so we use 20000 as our frequency cutoff and 2000 as our score cutoff. Our hyper-parameters for the frequency cut-off were 5000, 10000 and 20000, and our hyper-parameters for the score-cutoff were 1000 and 2000.

We continue with the optimal space and score-type chosen by the single direction experiments, and use the same frequency and score thresholds as before. Two different clustering algorithms are experimented with: Derrac and K-Means. As these algorithms select centroids from the top-scoring directions or randomly, we can expect that some clusters may not be salient features of the space. This is because top-scoring directions, e.g. for accuracy could simply infrequent terms that do not have much meaning, and these infrequent terms could also be randomly selected. We could use grid-search on the frequency and score cutoffs when obtaining these results in order to avoid terms that may disrupt existing clusters or form cluster centers that are not salient features of the space, but we chose a more standardized process that would rely on the parameters of the clustering algorithms and the ability of the classifiers to filter out clusters that are not informative, so as to not make a time-costly grid search a necessary part of the process.


For K-means clustering, we use Mini batch K-means, implemented by scikit-learn \footnote{https://scikit-learn.org/stable/modules/generated/sklearn.cluster.MiniBatchKMeans.html}, introduced by \cite{Sculley2010} and kmeans++ to initialize \cite{Arthur}

\subsection{Summary of all Results}

To begin, the original dimensions of the space are compared to the rankings on single words, the rankings on cluster directions, and the Bag-Of-Words of PPMI scores and topic models on low-depth Decision Trees. Single directions or clusters outperform the baselines in most cases, with the exceptions being in the place-types domain and the keywords task for the movies. For the keywords task, the natural explanation is that in a depth-1 tree, finding words which are directly corresponding to particular keywords is easier with words than if using directions, not only because certain words may have been filtered out, but also because as they are infrequent they may not be well-represented in the space. In this case, the PPMI representation is perfect, as it can find 1-1 matches with the classes without the representations of those words being spatially influenced by other similar words, as it can be expected for them to be in the space. However, this changes when going from depth-one to depth-two and depth-three, which is likely  due to overfitting in the case of the PPMI representation. Sometimes Decision Trees of depth-two outperform those of depth-one, but generally depth-three trees perform best.  In the case of the place-types, although topic models and PPMI representations are indeed the best, it is not by a wide-margin. Meanwhile when the single directions perform the best in these domains for other tree types they perform much better than the other approaches. Additionally, place-types is our most unbalanced domain with the least documents, so it is possible that they overfit. 




\begin{landscape}
	\begin{table}[]
		\begin{tabular}{llll@{\hskip 0.25in}lll@{\hskip 0.25in}lllll}
			& Genres                          &              &               & Keywords                        &                 &                       & Ratings                         &                      &                     &  & \\
			Movies            & D1                              & D2                              & D3                              & D1                              & D2                              & D3                              & D1                              & D2                              & D3                              &             &             \\
			\toprule[\heavyrulewidth]
			Space             & 0.301                           & 0.358                           & 0.354                           & 0.185                           & 0.198                           & 0.201                           & 0.463                           & 0.475                           & 0.486                           &             &             \\
			Single directions & \textbf{0.436} & 0.463                           & 0.492                           & 0.23                            & \textbf{0.233} & \textbf{0.224} & 0.466                           & 0.499                           & 0.498                           &             &             \\
			Clusters          & 0.431                           & \textbf{0.513} & \textbf{0.506} & 0.215                           & 0.22                            & 0.219                           & \textbf{0.504} & \textbf{0.507} & \textbf{0.513} &             &             \\
			PPMI              & 0.429                           & 0.443                           & 0.483                           & \textbf{0.243} & 0.224                           & 0.224                           & 0.47                            & 0.453                           & 0.453                           &             &             \\
			Topic             & 0.415                           & 0.472                           & 0.455                           & 0.189                           & 0.05                            & 0.075                           & 0.473                           & 0.243                           & 0.38                            &             &             \\
			& Newsgroups                      &                                 &                                 & Sentiment                       &                                 &                                 & Reuters                         &                                 &                                 &             &             \\
			& D1                              & D2                              & D3                              & D1                              & D2                              & D3                              & D1                              & D2                              & D3                              &             &             \\
			
			\toprule[\heavyrulewidth]
			Rep               & 0.251                           & 0.366                           & 0.356                           & 0.705                           & 0.77                            & 0.773                           & 0.328                           & 0.413                           & 0.501                           &             &             \\
			Single dir        & 0.418                           & \textbf{0.49}  & \textbf{0.537} & 0.784                           & 0.814                           & \textbf{0.821} & \textbf{0.678} & \textbf{0.706} & 0.72                            &             &             \\
			Cluster           & 0.394                           & 0.433                           & 0.513                           & 0.735                           & \textbf{0.844} & 0.813                           & 0.456                           & 0.569                           & 0.583                           &             &             \\
			PPMI              & 0.33                            & 0.407                           & 0.444                           & 0.7                             & 0.719                           & 0.73                            & 0.616                           & 0.699                           & \textbf{0.723} &             &             \\
			Topic             & \textbf{0.431} & 0.423                           & 0.444                           & \textbf{0.79}  & 0.791                           & 0.811                           & 0.411                           & 0.527                           & 0.536                           &             &             \\
			& Foursquare                      &                                 &                                 & OpenCYC                         &                                 &                                 & Geonames                        &                                 &                                 &             &             \\
			Placetypes        & D1                              & D2                              & D3                              & D1                              & D2                              & D3                              & D1                              & D2                              & D3                              &             &             \\
			\toprule[\heavyrulewidth]
			Rep               & 0.438                           & 0.478                           & 0.454                           & 0.383                           & 0.397                           & 0.396                           & 0.349                           & 0.34                            & 0.367                           &             &             \\
			Single dir        & \textbf{0.541} & 0.498                           & \textbf{0.531} & 0.404                           & \textbf{0.428} & 0.39                            & \textbf{0.444} & \textbf{0.533} & \textbf{0.473} &             &             \\
			Cluster           & 0.462                           & 0.507                           & 0.496                           & \textbf{0.413} & 0.42                            & \textbf{0.429} & 0.444                           & 0.458                           & 0.47                            &             &             \\
			PPMI              & 0.473                           & \textbf{0.512} & 0.491                           & 0.371                           & 0.351                           & 0.352                           & 0.361                           & 0.301                           & 0.242                           &             &             \\
			Topic             & 0.488                           & 0.433                           & 0.526                           & 0.365                           & 0.271                           & 0.313                           & 0.365                           & 0.3                             & 0.219                           &             &             \\   
		\end{tabular}
		\caption{summary of all results}
	\end{table}
\end{landscape}

\subsection{Baseline Representations}

In Table \ref{ch3:represults} all variations of the baseline representations used directly as input to Decision Trees and SVM's are shown. These examples that do not apply our methodology, serve as a reference point for what is possible using standard linear models without the need for interpretability. In the representations, there is a big performance drop when going from depth three trees to depth one trees. These kind of performance drops are expected for these representations, as they do not have dimensions that correspond to key semantics, so it is unlikely that a smaller tree can use the available dimensions to model a class with limited depth. In this full table the precision and recall scores are included for clarity, mainly to explain why the high recall scores occur. This is because the weights are balanced as a hyper-parameters, and when the weight is balanced so that positive instances are weighted more heavily, the model prioritizes recall over precision. When this high recall score doesn't occur, that means that not balancing the weights performed better on the development data.

The size of the space is not as influential as the representation type in these results for the Decision Trees. For this reason only the best performing representation of each type are shown in Table \ref{ch3:represults}. Out of the space-types, PCA performed much better than its counterparts for reuters, newsgroups and sentiment.  The MDS representation performs comparably well using a unrestricted depth tree or an SVM, which shows that with a classifier that can make use of all the dimensions, the performance does not decrease as much. This is likely due to the way that PCA orders its dimensions in importance,  resulting in key semantics in its first dimensions, giving it an advantage in low-depth Decision Trees. However, this does not necessarily mean that it contains better directions. In the single directions results, PCA is outperformed by MDS and other representations in F1 score for low Decision Tree depths in any of these domains, with the exception of the depth-two trees for sentiment. Despite MDS often encoding the key semantics across more dimensions than other representations, our method is still able find meaningful directions from this space. There is little link between performance on the raw dimensions of the space and performance with rankings on directions in low-depth Decision Trees.  This is somewhat counterintuitive, as it would be normal to expect that a representation which performs poorly when used directly as input to a classifier would have similar performance after a linear transformation, but the reason that it works in our case is because low-depth Decision Trees rely on key semantics being disentangled into individual dimensions. Despite the information encoded in the space, if it is not disentangled then the classifier will not perform well.



\begin{landscape}
\begin{table}[]
	\scriptsize
	\begin{tabular}{lllll@{\hskip 0.25in}llll@{\hskip 0.25in}llll@{\hskip 0.25in}llll@{\hskip 0.25in}lllll}
		Newsgroups & D1                              &                                 &                                 &                                 & D2                              &                                 &                                 &                                 & D3                              &                                 &                                 &                                 & DN                              &                                 &                                 &                                 & SVM                             &                                 &                                 &                                 &                                 \\
& ACC                             & F1                              & Prec                            & Rec                             & ACC                             & F1                              & Prec                            & Rec                             & ACC                             & F1                              & Prec                            & Rec                             & ACC                             & F1                              & Prec                            & Rec                             & ACC                             & F1                              & Prec                            & Rec                             &                                 \\
\toprule[\heavyrulewidth]
PCA 200    & 0.701                           & 0.251                           & 0.148                           & 0.811                           & 0.843                           & 0.366                           & 0.245                           & 0.719                           & 0.956                           & 0.355                           & 0.54                            & 0.265                           & 0.946                           & 0.44                            & 0.45                            & 0.432                           & 0.969                           & 0.612                           & 0.746                           & 0.519                           &                                 \\
PCA 100    & 0.698                           & 0.247                           & 0.146                           & 0.813                           & 0.835                           & 0.362                           & 0.241                           & 0.731                           & 0.957                           & 0.356                           & 0.576                           & 0.257                           & 0.948                           & 0.451                           & 0.465                           & 0.438                           & 0.969                           & 0.586                           & 0.768                           & 0.474                           &                                 \\
PCA 50     & 0.68                            & 0.24                            & 0.141                           & 0.829                           & 0.834                           & 0.355                           & 0.234                           & 0.735                           & 0.957                           & 0.329                           & 0.472                           & 0.253                           & 0.947                           & 0.45                            & 0.462                           & 0.438                           & 0.966                           & 0.52                            & 0.745                           & 0.399                           &                                 \\
\midrule[\heavyrulewidth]
AWV 200    & 0.687                           & 0.217                           & 0.126                           & 0.781                           & 0.758                           & 0.256                           & 0.156                           & 0.718                           & 0.764                           & 0.26                            & 0.157                           & 0.751                           & 0.937                           & 0.339                           & 0.352                           & 0.328                           & 0.961                           & 0.468                           & 0.641                           & 0.369                           &                                 \\
AWV 100    & 0.677                           & 0.21                            & 0.122                           & 0.775                           & 0.78                            & 0.275                           & 0.173                           & 0.683                           & 0.746                           & 0.25                            & 0.149                           & 0.769                           & 0.934                           & 0.324                           & 0.332                           & 0.317                           & 0.865                           & 0.4                             & 0.265                           & 0.812                           &                                 \\
AWV 50     & 0.696                           & 0.219                           & 0.127                           & 0.772                           & 0.777                           & 0.272                           & 0.168                           & 0.71                            & 0.743                           & 0.25                            & 0.149                           & 0.786                           & 0.935                           & 0.325                           & 0.335                           & 0.316                           & 0.842                           & 0.362                           & 0.233                           & 0.819                           &                                 \\
\midrule[\heavyrulewidth]
MDS 200    & 0.581                           & 0.184                           & 0.103                           & \textbf{0.837} & 0.742                           & 0.262                           & 0.16                            & 0.729                           & 0.719                           & 0.236                           & 0.139                           & 0.785                           & 0.935                           & 0.327                           & 0.332                           & 0.323                           & 0.965                           & 0.501                           & \textbf{0.802} & 0.364                           &                                 \\
MDS 100    & 0.586                           & 0.187                           & 0.105                           & 0.833                           & 0.754                           & 0.261                           & 0.159                           & 0.727                           & 0.705                           & 0.236                           & 0.138                           & \textbf{0.808} & 0.935                           & 0.33                            & 0.338                           & 0.321                           & 0.878                           & 0.439                           & 0.308                           & 0.765                           &                                 \\
MDS 50     & 0.593                           & 0.153                           & 0.087                           & 0.647                           & 0.716                           & 0.25                            & 0.15                            & \textbf{0.756} & 0.736                           & 0.243                           & 0.144                           & 0.774                           & 0.935                           & 0.324                           & 0.335                           & 0.313                           & 0.854                           & 0.394                           & 0.259                           & 0.821                           &                                 \\
\midrule[\heavyrulewidth]
D2V 200    & 0.682                           & 0.205                           & 0.119                           & 0.746                           & 0.802                           & 0.268                           & 0.169                           & 0.646                           & 0.77                            & 0.269                           & 0.164                           & 0.75                            & 0.94                            & 0.366                           & 0.389                           & 0.346                           & 0.961                           & 0.468                           & 0.641                           & 0.369                           &                                 \\
D2V 100    & 0.682                           & 0.208                           & 0.12                            & 0.762                           & 0.792                           & 0.268                           & 0.168                           & 0.662                           & 0.786                           & 0.268                           & 0.164                           & 0.727                           & 0.94                            & 0.376                           & 0.392                           & 0.361                           & \textbf{0.971} & \textbf{0.628} & 0.761                           & 0.535                           &                                 \\
D2V 50     & 0.683                           & 0.207                           & 0.12                            & 0.764                           & 0.809                           & 0.294                           & 0.187                           & 0.694                           & 0.782                           & 0.28                            & 0.172                           & 0.761                           & 0.943                           & 0.394                           & 0.415                           & 0.376                           & 0.97                            & 0.601                           & 0.758                           & 0.497                           &                                 \\
\midrule[\heavyrulewidth]
PPMI       & \textbf{0.948} & 0.33                            & \textbf{0.532} & 0.239                           & 0.947                           & 0.407                           & 0.511                           & 0.338                           & 0.944                           & \textbf{0.444} & 0.506                           & 0.396                           & \textbf{0.951} & \textbf{0.494} & \textbf{0.496} & \textbf{0.492} & 0.962                           & 0.613                           & 0.627                           & 0.599                           &                                 \\
\midrule[\heavyrulewidth]
Topic      & 0.852                           & \textbf{0.431} & 0.304                           & 0.743                           & \textbf{0.96}  & \textbf{0.423} & \textbf{0.604} & 0.326                           & \textbf{0.961} & 0.444                           & \textbf{0.606} & 0.35                            & 0.944                           & 0.432                           & 0.434                           & 0.429                           & 0.879                           & 0.46                            & 0.318                           & \textbf{0.835} &                                 \\
	\end{tabular}
\caption{Full results for the newsgroups.}\label{ch3:represults}
\end{table}
\end{landscape}




\begin{landscape}
\begin{table}
	\centering
	\scriptsize
	\caption{Results for all other domains for the representations.}
	\begin{tabular}{ll@{\hskip 0.15in}l@{\hskip 0.2in}l@{\hskip 0.15in}l@{\hskip 0.2in}l@{\hskip 0.15in}l@{\hskip 0.2in}l@{\hskip 0.15in}l@{\hskip 0.2in}l@{\hskip 0.15in}l@{\hskip 0.1in}lll@{\hskip 0.15in}l@{\hskip 0.2in}l@{\hskip 0.15in}l@{\hskip 0.2in}l@{\hskip 0.15in}l@{\hskip 0.2in}l@{\hskip 0.15in}l@{\hskip 0.2in}l@{\hskip 0.15in}l@{\hskip 0.2in}l@{\hskip 0.15in}l@{\hskip 0.2in}lll}
		Reuters    & D1              &                 & D2              &                 & D3              &                 & DN              &                 & SVM             &                 &  & Sentiment & D1              &                 & D2              &                 & D3              &                 & DN              &                 & SVM             &                  \\
		& ACC             & F1              & ACC             & F1              & ACC             & F1              & ACC             & F1              & ACC             & F1              &  &           & ACC             & F1              & ACC             & F1              & ACC             & F1              & ACC             & F1              & ACC             & F1               \\ 

\cmidrule[\heavyrulewidth]{1-11}\cmidrule[\heavyrulewidth]{13-23}
		PCA        & 0.847           & 0.328           & 0.917           & 0.413           & 0.978           & 0.501           & 0.978           & 0.565           & 0.989           & 0.761           &  & PCA       & 0.745           & 0.705           & 0.755           & 0.77            & 0.778           & 0.773           & \textbf{0.781}  & \textbf{0.779}  & \textbf{0.891}  & \textbf{0.893}   \\
		AWV        & 0.782           & 0.252           & 0.971           & 0.328           & 0.974           & 0.417           & 0.973           & 0.495           & 0.987           & 0.719           &  & AWV       & 0.642           & 0.652           & 0.643           & 0.694           & 0.695           & 0.717           & 0.66            & 0.663           & 0.827           & 0.829            \\
		MDS        & 0.791           & 0.263           & 0.9             & 0.357           & 0.979           & 0.489           & 0.976           & 0.522           & 0.988           & 0.67            &  & D2V       & 0.642           & 0.664           & 0.66            & 0.707           & 0.702           & 0.7             & 0.711           & 0.708           & 0.878           & 0.878            \\
		D2V        & 0.818           & 0.268           & 0.867           & 0.298           & 0.974           & 0.445           & 0.971           & 0.482           & 0.986           & 0.724           &  & PPMI      & 0.616           & 0.7             & 0.655           & 0.719           & 0.675           & 0.73            & 0.712           & 0.71            & 0.887           & 0.888            \\
		PPMI       & \textbf{0.975}  & \textbf{0.616}  & \textbf{0.978}  & \textbf{0.699}  & \textbf{0.98}   & \textbf{0.723}  & \textbf{0.984}  & \textbf{0.746}  & \textbf{0.99}   & \textbf{0.8}    &  & Topic     & \textbf{0.793}  & \textbf{0.79}   & \textbf{0.794}  & \textbf{0.791}  & \textbf{0.81}   & \textbf{0.811}  & 0.733           & 0.73            & 0.815           & 0.822            \\
		Topic      & 0.92            & 0.411           & 0.977           & 0.527           & 0.977           & 0.536           & 0.977           & 0.56            & 0.95            & 0.513           &  &           &                 &                 &                 &                 &                 &                 &                 &                 &                 &                  \\
		&                 &                 &                 &                 &                 &                 &                 &                 &                 &                 &  &           &                 &                 &                 &                 &                 &                 &                 &                 &                 &                  \\
		Placetypes & D1              &                 & D2              &                 & D3              &                 & DN              &                 & SVM             &                 &  & Movies    & D1              &                 & D2              &                 & D3              &                 & DN              &                 & SVM             &                  \\
		OpenCYC    & ACC             & F1              & ACC             & F1              & ACC             & F1              & ACC             & F1              & ACC             & F1              &  & Genres    & ACC             & F1              & ACC             & F1              & ACC             & F1              & ACC             & F1              & ACC             & F1               \\ 
		\cmidrule[\heavyrulewidth]{1-11}\cmidrule[\heavyrulewidth]{13-23}
		PCA        & 0.586           & 0.346           & 0.708           & 0.343           & 0.695           & 0.342           & 0.832           & 0.309           & 0.847           & 0.474           &  & PCA       & 0.722           & 0.301           & 0.755           & 0.339           & 0.717           & 0.321           & 0.884           & 0.372           & \textbf{0.925}  & 0.518            \\
		AWV        & 0.625           & \textbf{0.383}  & 0.651           & 0.376           & 0.728           & \textbf{0.396}  & \textbf{0.844}  & \textbf{0.362}  & 0.85            & 0.466           &  & AWV       & 0.679           & 0.29            & 0.774           & 0.321           & 0.756           & 0.343           & 0.873           & 0.312           & 0.922           & 0.496            \\
		MDS        & 0.624           & 0.364           & 0.7             & \textbf{0.397}  & 0.731           & 0.374           & 0.843           & 0.305           & 0.861           & \textbf{0.476}  &  & MDS       & 0.679           & 0.298           & 0.79            & 0.358           & 0.773           & 0.354           & 0.887           & 0.385           & 0.875           & \textbf{0.532}   \\
		PPMI       & \textbf{0.728}  & 0.371           & 0.75            & 0.351           & 0.739           & 0.352           & 0.843           & 0.323           & \textbf{0.9}    & 0.366           &  & PPMI      & \textbf{0.852}  & \textbf{0.429}  & \textbf{0.91}   & 0.443           & \textbf{0.912}  & \textbf{0.483}  & 0.882           & \textbf{0.416}  & 0.923           & 0.526            \\
		Topic      & 0.708           & 0.365           & \textbf{0.87}   & 0.271           & \textbf{0.87}   & 0.313           & 0.831           & 0.313           & 0.808           & 0.407           &  & Topic     & 0.767           & 0.415           & 0.905           & \textbf{0.472}  & 0.912           & 0.455           & \textbf{0.889}  & 0.415           & 0.843           & 0.491            \\
		&                 &                 &                 &                 &                 &                 &                 &                 &                 &                 &  &           &                 &                 &                 &                 &                 &                 &                 &                 &                 &                  \\
		Placetypes & D1              &                 & D2              &                 & D3              &                 & DN              &                 & SVM             &                 &  & Movies    & D1              &                 & D2              &                 & D3              &                 & DN              &                 & SVM             &                  \\
		Foursquare & ACC             & F1              & ACC             & F1              & ACC             & F1              & ACC             & F1              & ACC             & F1              &  & Keywords  & ACC             & F1              & ACC             & F1              & ACC             & F1              & ACC             & F1              & ACC             & F1               \\ 
		\cmidrule[\heavyrulewidth]{1-11}\cmidrule[\heavyrulewidth]{13-23}
		PCA        & 0.731           & 0.342           & 0.823           & 0.393           & 0.86            & 0.388           & 0.887           & 0.398           & 0.896           & 0.568           &  & PCA       & 0.647           & 0.185           & 0.644           & 0.193           & 0.677           & 0.199           & 0.846           & 0.161           & 0.787           & 0.272            \\
		AWV        & 0.767           & 0.401           & 0.828           & 0.478           & 0.85            & 0.452           & 0.905           & \textbf{0.505}  & 0.923           & \textbf{0.622}  &  & AWV       & 0.5             & 0.16            & 0.641           & 0.179           & 0.595           & 0.174           & 0.853           & 0.141           & 0.717           & 0.23             \\
		MDS        & \textbf{0.915}  & 0.438           & 0.804           & 0.427           & 0.86            & 0.454           & 0.893           & 0.462           & 0.932           & 0.619           &  & MDS       & 0.633           & 0.179           & 0.69            & 0.198           & 0.674           & 0.201           & 0.84            & 0.163           & 0.788           & \textbf{0.28}    \\
		PPMI       & 0.889           & 0.473           & 0.915           & \textbf{0.512}  & 0.904           & 0.491           & 0.881           & 0.31            & \textbf{0.938}  & 0.567           &  & PPMI      & \textbf{0.818}  & \textbf{0.243}  & 0.745           & \textbf{0.224}  & 0.739           & \textbf{0.224}  & 0.847           & \textbf{0.17}   & \textbf{0.921}  & 0.217            \\
		Topic      & 0.864           & \textbf{0.488}  & \textbf{0.916}  & 0.433           & \textbf{0.917}  & \textbf{0.526}  & \textbf{0.907}  & 0.464           & 0.916           & 0.569           &  & Topic     & 0.629           & 0.189           & \textbf{0.932}  & 0.05            & \textbf{0.93}   & 0.075           & \textbf{0.857}  & 0.152           & 0.678           & 0.21             \\
		&                 &                 &                 &                 &                 &                 &                 &                 &                 &                 &  &           &                 &                 &                 &                 &                 &                 &                 &                 &                 &                  \\
		Placetypes & D1              &                 & D2              &                 & D3              &                 & DN              &                 & SVM             &                 &  & Movies    & D1              &                 & D2              &                 & D3              &                 & DN              &                 & SVM             &                  \\
		Geonames   & ACC             & F1              & ACC             & F1              & ACC             & F1              & ACC             & F1              & ACC             & F1              &  & Ratings   & ACC             & F1              & ACC             & F1              & ACC             & F1              & ACC             & F1              & ACC             & F1               \\ 
		\cmidrule[\heavyrulewidth]{1-11}\cmidrule[\heavyrulewidth]{13-23}
		PCA        & 0.502           & 0.301           & 0.69            & 0.305           & 0.68            & 0.295           & 0.821           & 0.243           & 0.844           & 0.401           &  & PCA       & \textbf{0.65}   & 0.463           & 0.681           & \textbf{0.475}  & 0.684           & \textbf{0.486}  & 0.744           & 0.408           & 0.771           & 0.58             \\
		AWV        & 0.657           & 0.326           & 0.755           & 0.323           & 0.842           & \textbf{0.367}  & 0.813           & 0.332           & 0.865           & \textbf{0.514}  &  & AWV       & 0.601           & 0.423           & 0.618           & 0.433           & 0.596           & 0.448           & 0.736           & 0.372           & 0.73            & 0.532            \\
		MDS        & 0.626           & 0.349           & 0.695           & \textbf{0.34}   & 0.796           & 0.272           & \textbf{0.845}  & 0.295           & 0.638           & 0.397           &  & MDS       & 0.592           & 0.437           & 0.635           & 0.449           & 0.631           & 0.452           & \textbf{0.752}  & \textbf{0.412}  & 0.773           & \textbf{0.589}   \\
		PPMI       & \textbf{0.808}  & 0.361           & 0.732           & 0.301           & 0.76            & 0.242           & 0.83            & 0.283           & \textbf{0.894}  & 0.312           &  & PPMI      & 0.583           & 0.47            & 0.635           & 0.453           & 0.605           & 0.453           & 0.73            & 0.384           & \textbf{0.825}  & 0.536            \\
		Topic      & 0.771           & \textbf{0.365}  & \textbf{0.863}  & 0.3             & \textbf{0.85}   & 0.219           & 0.828           & \textbf{0.348}  & 0.819           & 0.349           &  & Topic     & 0.575           & \textbf{0.473}  & \textbf{0.789}  & 0.243           & \textbf{0.789}  & 0.38            & 0.739           & 0.375           & 0.704           & 0.501           
	\end{tabular}\label{ch3:represults_all}
\end{table}
\end{landscape}





\subsection{Word Directions}
Although Linear SVM's perform the best on these representations without the need for interpretability, other results will be for low-depth Decision Trees in-order to easily distinguish the degree to which key semantics correspond to dimensions in the representations.

 The main takeaway from this section is that in most cases performance greatly increases compared to the original representations used directly as input to the model (For the exact differences, see Appendix \ref{app:repandsingdirdiff}). 
 
 Interestingly, there was also more variance in the difference between space-type sizes, making it an important hyper-parameter for the single directions. The best space type also varied across domains. Loosely, it is possible to attribute the performance increase for a space-type to either modelling the rankings for the same directions better, or containing unique terms that were particularly relevant to the classes. However, when looking at the qualitative results, generally the words common to all space-types are the most salient \ref{ch3:ComparingSpaceTypes}. We can see if this is the case by looking at the Decision Trees for the same task that had the most difference between the space-types and space-sizes. If a Decision Tree contains mostly similar words, but the performance is greater, we can attribute it to a better quality ranking in the space. If the Decision Tree contains different words, especially as the first node, then we know that it was because the words that were modelled well were different between them. 
 
 We see that generally, the best space type is the same across a variety of tasks in the same domain, AWV is the best for the place-types but MDS is best for the movies (despite a marginal difference in the ratings). This could mean that performance on one natural task will generalize well to the others, so the space-type/size of the space that we identify contains the key semantics for that domain rather than a particular task. 
 
 NDCG was selected as the best score-type for Sentiment, Newsgroups, Reuters, Movies Genres, Movies Keywords in depth-3 Decision Trees. Place-types foursquare used F1-score, but the classes are very unbalanced and there are few documents. 
 %To investigate why space-types perform better than others for each domain, we compare the Decision Trees where there is the most difference. For the genres, MDS outperforms PCA and AWV on the genres task by 0.03. 

%What is the importance of the space size?
%What is the importance of the space type?
%How do directions perform compared to spaces? Why?
%What kind of directions do we find for each score type?
%Why does the score type matter? What score type works best?

%These single directions typically overfit.

\begin{landscape}
\begin{table}[]
	\centering
	\scriptsize
	\begin{tabular}{llllllllllllll}
Newsgroups & D1                              &                  &                &                       & D2                              &                     &                    &      & D3                              &                &                &         &                    \\
& ACC                             & F1                              & Prec                            & Rec                             & ACC                             & F1                              & Prec                            & Rec                             & ACC                             & F1                              & Prec                            & Rec                             &                                 \\
PCA 200    & 0.955                           & 0.348                           & 0.521                           & 0.261                           & 0.959                           & 0.424                           & 0.678                           & 0.309                           & 0.96                            & 0.454                           & 0.674                           & 0.343                           &                                 \\
PCA 100    & 0.957                           & 0.382                           & 0.491                           & 0.313                           & 0.961                           & 0.474                           & 0.679                           & 0.364                           & \textbf{0.963} & 0.512                           & 0.694                           & 0.406                           &                                 \\
PCA 50     & 0.957                           & 0.373                           & 0.417                           & 0.337                           & \textbf{0.963} & 0.478                           & 0.621                           & 0.388                           & 0.963                           & 0.506                           & 0.7                             & 0.396                           &                                 \\
AWV 200    & 0.832                           & 0.35                            & 0.226                           & 0.777                           & 0.957                           & 0.383                           & 0.517                           & 0.305                           & 0.958                           & 0.445                           & 0.598                           & 0.354                           &                                 \\
AWV 100    & 0.83                            & 0.343                           & 0.219                           & 0.785                           & 0.823                           & 0.36                            & 0.233                           & \textbf{0.792} & 0.956                           & 0.387                           & 0.563                           & 0.295                           &                                 \\
AWV 50     & 0.807                           & 0.341                           & 0.215                           & 0.816                           & 0.833                           & 0.361                           & 0.236                           & 0.762                           & 0.954                           & 0.392                           & 0.511                           & 0.318                           &                                 \\
MDS 200    & \textbf{0.959} & \textbf{0.418} & \textbf{0.543} & 0.339                           & 0.962                           & 0.465                           & 0.669                           & 0.357                           & 0.962                           & 0.493                           & \textbf{0.707} & 0.379                           &                                 \\
MDS 100    & 0.857                           & 0.365                           & 0.244                           & 0.725                           & 0.959                           & 0.428                           & 0.624                           & 0.326                           & 0.96                            & 0.453                           & 0.644                           & 0.349                           &                                 \\
MDS 50     & 0.821                           & 0.324                           & 0.206                           & 0.762                           & 0.842                           & 0.386                           & 0.258                           & 0.77                            & 0.957                           & 0.398                           & 0.596                           & 0.299                           &                                 \\
D2V 200    & 0.831                           & 0.343                           & 0.22                            & 0.784                           & 0.96                            & 0.47                            & \textbf{0.683} & 0.358                           & 0.962                           & 0.494                           & 0.69                            & 0.385                           &                                 \\
D2V 100    & 0.844                           & 0.374                           & 0.243                           & 0.803                           & 0.961                           & \textbf{0.49}  & 0.642                           & 0.396                           & 0.962                           & 0.517                           & 0.67                            & 0.421                           &                                 \\
D2V 50     & 0.845                           & 0.388                           & 0.252                           & \textbf{0.844} & 0.962                           & 0.488                           & 0.639                           & 0.395                           & 0.963                           & \textbf{0.537} & 0.673                           & \textbf{0.446} &                                 \\
&                                 &                                 &                                 &                                 &                                 &                                 &                                 &                                 &                                 &                                 &                                 &                                 &                                 \\
Reuters    & D1                              &                                 & D2                              &                                 & D3                              &                                 & Sentiment                       & D1                              &                                 & D2                              &                                 & D3                              &                                 \\
& ACC                             & F1                              & ACC                             & F1                              & ACC                             & F1                              &                                 & ACC                             & F1                              & ACC                             & F1                              & ACC                             & F1                              \\
PCA        & 0.976                           & 0.658                           & 0.979                           & 0.679                           & 0.977                           & 0.467                           & PCA                             & 0.739                           & 0.759                           & \textbf{0.797} & \textbf{0.814} & 0.802                           & 0.805                           \\
AWV        & 0.975                           & 0.598                           & 0.979                           & 0.656                           & 0.98                            & 0.66                            & AWV                             & 0.7                             & 0.699                           & 0.711                           & 0.736                           & 0.723                           & 0.735                           \\
MDS        & 0.975                           & \textbf{0.678} & \textbf{0.98}  & \textbf{0.706} & \textbf{0.982} & \textbf{0.72}  & D2V                             & \textbf{0.776} & \textbf{0.784} & 0.782                           & 0.801                           & \textbf{0.822} & \textbf{0.821} \\
D2V        & \textbf{0.977} & 0.583                           & 0.979                           & 0.664                           & 0.98                            & 0.632                           &                                 &                                 &                                 &                                 &                                 &                                 &                                 \\
Placetypes & D1                              &                                 & D2                              &                                 & D3                              &                                 & Movies                          & D1                              &                                 & D2                              &                                 & D3                              &                                 \\
OpenCYC    & ACC                             & F1                              & ACC                             & F1                              & ACC                             & F1                              & Genres                          & ACC                             & F1                              & ACC                             & F1                              & ACC                             & F1                              \\
PCA        & 0.632                           & 0.371                           & 0.704                           & 0.381                           & 0.735                           & 0.365                           & PCA                             & 0.824                           & 0.412                           & 0.82                            & 0.441                           & 0.913                           & 0.463                           \\
AWV        & \textbf{0.66}  & \textbf{0.404} & \textbf{0.734} & \textbf{0.428} & \textbf{0.755} & \textbf{0.39}  & AWV                             & 0.81                            & 0.421                           & 0.837                           & 0.436                           & 0.912                           & 0.457                           \\
MDS        & 0.658                           & 0.374                           & 0.711                           & 0.385                           & 0.746                           & 0.35                            & MDS                             & \textbf{0.849} & \textbf{0.446} & \textbf{0.839} & \textbf{0.463} & \textbf{0.918} & \textbf{0.495} \\
Foursquare & ACC                             & F1                              & ACC                             & F1                              & ACC                             & F1                              & Keywords                        & ACC                             & F1                              & ACC                             & F1                              & ACC                             & F1                              \\
PCA        & 0.785                           & 0.477                           & \textbf{0.907} & 0.474                           & 0.869                           & \textbf{0.531} & PCA                             & 0.737                           & 0.225                           & 0.727                           & 0.227                           & \textbf{0.709} & 0.22                            \\
AWV        & \textbf{0.918} & \textbf{0.541} & 0.881                           & \textbf{0.498} & 0.889                           & 0.466                           & AWV                             & 0.656                           & 0.201                           & 0.672                           & 0.203                           & 0.652                           & 0.2                             \\
MDS        & 0.82                            & 0.416                           & 0.879                           & 0.482                           & \textbf{0.897} & 0.485                           & MDS                             & \textbf{0.745} & \textbf{0.23}  & \textbf{0.74}  & \textbf{0.233} & 0.708                           & \textbf{0.224} \\
Geonames   & ACC                             & F1                              & ACC                             & F1                              & ACC                             & F1                              & Ratings                         & ACC                             & F1                              & ACC                             & F1                              & ACC                             & F1                              \\
PCA        & 0.665                           & 0.348                           & 0.754                           & 0.342                           & 0.743                           & 0.306                           & PCA                             & \textbf{0.647} & \textbf{0.466} & \textbf{0.721} & \textbf{0.499} & 0.681                           & 0.492                           \\
AWV        & \textbf{0.711} & \textbf{0.444} & \textbf{0.795} & \textbf{0.533} & \textbf{0.802} & \textbf{0.473} & AWV                             & 0.646                           & 0.463                           & 0.692                           & 0.474                           & 0.677                           & 0.483                           \\
MDS        & 0.591                           & 0.289                           & 0.772                           & 0.333                           & 0.764                           & 0.352                           & MDS                             & 0.62                            & 0.463                           & 0.692                           & 0.489                           & \textbf{0.686} & \textbf{0.498}
	\end{tabular}
	\caption{ all dirs}
\end{table}
\end{landscape}


\subsection{Clustered Directions}
\ref{ch3:clusterexample}
These results were obtained by taking the single directions that performed the best in the previous results and clustering them with a variety of hyper-parameters for the clusters. K-means mostly outperforms Derrac. It does not in the case of Keywords, where it performs better for every Decision Tree. Although the differences in absolute values are quite small in this case, it is still significant as it is quite difficult to achieve high performance on this task, making these relative changes important. This case can give us insight into how disentanglement affects performance on different classes and domains - and how our unsupervised method selects the best parameters. 

When looking into the how the individual classes fared, the 100-size Derrac clusters performed better at the keywords "shot-in-the-chest" and "machine-gun" and sacrificed performance in the "sequel" class. In Derrac, there was the following cluster ("soldiers combat fighting military battle ... weapons rambo gunfights spaghetti guns ...") while in the best performing k-means 200-size clusters these words were split into two separate clusters, one for guns ("gun explosions shoot shooting weapons ... rambo") and one for military ("war soldiers combat military ... platoon infantry"). It's possible that as the Derrac method combined these together into their own cluster they were able to better capture the classes for "shot-in-the-chest" and "machine-guns" because these things occurred in war films where people were shot or shooting. So in this case, the parameters chosen for Derrac supported the classification of the documents into keywords because they better captured particular class concepts through a lesser degree of disentanglement. This idea is supported when looking at the depth-three tree for this class, which uses this cluster as its first node as well as a node in the depth-two layer. This is an instance where having a heavily populated cluster average their direction performs better than strongly disentangling the concepts. 

Meanwhile, this same lack of disentanglement caused it to lose performance in the "sequel" class. In K-means, the cluster was found for ("franchise sequels sequel installments") while in Derrac the cluster was ("franchise sequels sequel instalments entry returns"). This cluster was also chosen in Derrac as the first node of its Decision Tree, but this caused it to perform worse than k-means. This is likely because although the words "entry" and "returns" were most similar to this cluster, they disrupted the direction too much. Indeed, when looking at the k-means clusters, the "returns" direction is clustered with "events situation conclusion spoiler ... protagonists exscapes break scenario ...", seemingly referring to a character or thing "returning" in a conclusive part of the movie, and the word "entry" is clustered with the words "effective genuine ... hits build surprisingly ... succeeds essentially finale entry ..." seemingly relating to a more sentiment related cluster about how a movie performed. So in this case k-means being able to find more disentangled clusters than Derrac gave it a performance advantage. 

This could be due to the best-performing Derrac clusters being 100-size (meaning the clusters would contain more terms) and the k-means being 200-size. However, in the 100-size K-means clusters, "gun" and "explosions" ended up being in a cluster with ("western outlaw heist shootout west"), making it a more western oriented cluster, and the idea of a war was even more disentangled with a single cluster corresponding to ("war soldiers military solider army sergeant sgt platoon infantry"). In conclusion, Derrac for the Keywords task captured certain concepts better than k-means, in particular by clustering together the idea of "war" and "guns" to achieve high performance on the keywords "shot-in-the-chest" and "machine-guns". K-means favoured a more disentangled approach to these ideas, which meant that although it captured the idea of "war" well, it was not able to capture the classes inbetween the idea of "war" and "guns".

In conclusion, the clustering method that performs the best for a task in this unsupervised context is the one that creates clusters that correspond closely with the task's classes, through clustering together words which average into a particular concept, or disentangling words into concepts so that they more precisely model it.




\begin{landscape}
\begin{table}[]

	\begin{tabular}{lll@{\hskip 0.15in}lllllllllllllll}
		Newsgroups  & D1                              &                     &                      &                   & D2                              &         &                  &                     & D3                                             \\
	& ACC                             & F1                              & Prec                             & Rec                              & ACC                             & F1                                                        & Prec                             & Rec                              & ACC                             & F1                              & Prec                             & Rec                                    \\
		\toprule
		K-means 200 & \textbf{0.852} & \textbf{0.394} & \textbf{0.261} & 0.795                           & \textbf{0.958} & \textbf{0.433} & \textbf{0.58} & 0.345                           & \textbf{0.963} & \textbf{0.513} & \textbf{0.704} & 0.403               \\
		K-means 100 & 0.842                           & 0.388                           & 0.257                           & 0.791                           & 0.958                           & 0.366                           & 0.516                          & 0.284                           & 0.962                           & 0.5                             & 0.635                           & \textbf{0.412}            \\
		K-means 50  & 0.834                           & 0.381                           & 0.248                           & \textbf{0.819} & 0.815                           & 0.336                           & 0.212                          & \textbf{0.81}  & 0.961                           & 0.485                           & 0.612                           & 0.402                     \\
		\midrule
		Derrac 200  & 0.803                           & 0.313                           & 0.202                           & 0.693                           & 0.797                           & 0.306                           & 0.191                          & 0.781                           & 0.958                           & 0.409                           & 0.605                           & 0.309                      \\
		Derrac 100  & 0.792                           & 0.305                           & 0.197                           & 0.667                           & 0.791                           & 0.287                           & 0.179                          & 0.721                           & 0.957                           & 0.374                           & 0.56                            & 0.281               \\
		Derrac 50   & 0.769                           & 0.26                            & 0.162                           & 0.661                           & 0.768                           & 0.237                           & 0.143                          & 0.693                           & 0.955                           & 0.315                           & 0.47                            & 0.237                                     \\
		&                                 &                                 &                                 &                                 &                                 &                                 &                                &                                 &                                 &                                 &                                 &                                                       \\
		
   
	\end{tabular}
\centering
	\caption{All clustering size results for the newsgroups}
\end{table}

	\begin{table}[]
		\footnotesize
		\begin{tabular}{llllllllllllll}
		Reuters     & D1                              &                                 & D2                              &                                 & D3                              &                                 & Sentiment                      & D1                              &                                 & D2                              &                                 & D3                              &                                             \\
& ACC                             & F1                              & ACC                             & F1                              & ACC                             & F1                              &                                & ACC                             & F1                              & ACC                             & F1                              & ACC                             & F1                                    \\
\toprule
K-means     & \textbf{0.875} & \textbf{0.338} & \textbf{0.975} & \textbf{0.54}  & 0.973                           & \textbf{0.58}  & K-means                        & 0.623                           & 0.674                           & \textbf{0.837} & \textbf{0.844} & 0.658                           & 0.707                                   \\
\midrule
Derrac      & 0.797                           & 0.291                           & 0.973                           & 0.402                           & \textbf{0.974} & 0.485                           & Derrac                         & \textbf{0.712} & \textbf{0.735} & 0.802                           & 0.82                            & \textbf{0.803} & \textbf{0.813}            \\
&&&&&&\\
Placetypes  & D1                              &                                 & D2                              &                                 & D3                              &                                 & Movies                         & D1                              &                                 & D2                              &                                 & D3                              &                          \\
OpenCYC     & ACC                             & F1                              & ACC                             & F1                              & ACC                             & F1                              & Genres                         & ACC                             & F1                              & ACC                             & F1                              & ACC                             & F1                                         \\
\toprule
K-means     & \textbf{0.641} & \textbf{0.413} & \textbf{0.735} & \textbf{0.405} & 0.75                            & \textbf{0.43}  & K-means                        & \textbf{0.813} & \textbf{0.431} & \textbf{0.913} & \textbf{0.513} & \textbf{0.913} & \textbf{0.506}    \\
\midrule
Derrac      & 0.605                           & 0.39                            & 0.672                           & 0.392                           & \textbf{0.755} & 0.391                           & Derrac                         & 0.759                           & 0.341                           & 0.789                           & 0.431                           & 0.911                           & 0.432                                    \\
&                                 &                                 &                                 &                                 &                                 &                                 &                                &                                 &                                 &                                 &                                 &                                 &                                         \\
Foursquare  & ACC                             & F1                              & ACC                             & F1                              & ACC                             & F1                              & Keywords                       & ACC                             & F1                              & ACC                             & F1                              & ACC                             & F1                               \\
\toprule
K-means     & \textbf{0.913} & \textbf{0.462} & \textbf{0.911} & \textbf{0.5}   & \textbf{0.891} & \textbf{0.511} & K-means                        & 0.667                           & 0.208                           & 0.648                           & 0.202                           & 0.678                           & 0.213                               \\
\midrule
Derrac      & 0.768                           & 0.392                           & 0.835                           & 0.445                           & 0.805                           & 0.425                           & Derrac                         & \textbf{0.726} & \textbf{0.215} & \textbf{0.745} & \textbf{0.22}  & \textbf{0.707} & \textbf{0.219}        \\
&                                 &                                 &                                 &                                 &                                 &                                 &                                &                                 &                                 &                                 &                                 &                                 &                            \\
Geonames    & ACC                             & F1                              & ACC                             & F1                              & ACC                             & F1                              & Ratings                        & ACC                             & F1                              & ACC                             & F1                              & ACC                             & F1                       \\
\toprule
K-means     & \textbf{0.772} & 0.43                            & \textbf{0.774} & 0.407                           & \textbf{0.819} & \textbf{0.472} & K-means                        & \textbf{0.671} & \textbf{0.504} & 0.638                           & \textbf{0.507} & \textbf{0.686} & \textbf{0.513}       \\

Derrac      & 0.678                           & \textbf{0.449} & 0.74                            & \textbf{0.411} & 0.807                           & 0.415                           & Derrac                         & 0.651                           & 0.445                           & \textbf{0.669} & 0.463                           & 0.627                           & 0.479                 
	\end{tabular}
\centering
\caption{The best clustering results for each domain and task}
\end{table}
\end{landscape}

%\subsection{What is the value of different score-types?}

%\subsection{Producing Vector Space Models}

%We use unsupervised representation learning methods, with the intention to obtain a representation that represents all salient features of the domain and can adapt to a variety of tasks. 

%For the Vector Space Model, we compute the Positive Pointwise Mutual Information (See \ref{bg:PPMI}) scores for the Bag-Of-Words, and use that as input to a variety of different off-the-shelf dimensionality reduction algorithms. We explain these in further detail in Section \ref{ch3:Spaces}. 


%\subsection{Quantitative Results}
%From a domain, e.g. movie reviews, where each document is a collection of reviews for a movie, we preprocess the text such that it is converted to lower-case, and non-alphanumeric characters are removed. From here, we remove standard English stop words using the NLTK library \cite{Bird}. We show an example of a review's original and converted formats in Figure \ref{ch3:OrigAndConverted}. From this preprocessed corpus, we obtain a Bag-Of-Words where we count the frequency of each term $BOW_wf$, see \ref{background:BOW}. 

%The difference between single directions and clusters is best highlighted when comparing their use in simple interpretable classifiers. In figure \ref{ch3:ComparedTrees} we demonstrate this.

%1. Negative directions (e.g. church for horror)
%2. Non-contextualized, non-direct ways of classifying, versus clustering which finds salient properties which almost directly correspond to these natural tasks.

%\begin{figure}[t]
%	\includegraphics[width=\textwidth]{images/ComparedTrees.png}
%	\centering
%	\caption{An example of a hyper-plane and its orthogonal direction in a toy domain of shapes. Green shapes are positive examples and red shapes are negative examples, but despite the problem being binary those closest to the hyper-plane are less defined than those further away, resulting in the orthogonal vector being a direction.}\label{ch3:ComparedTrees}
%\end{figure}




% Deeper explanation of our methodology and approach

\subsection{Conclusion}

In conclusion, we introduce a methodology to go from a Vector Space Model of Semantics and an associated bag-of-words to an interpretable representation and interpretable classifiers. We define an interpretable representation in this work as having two properties: disentanglement and labels, and an interpretable classifier as a simple linear classifier that has components corresponding to the interpretable representation that has these properties, e.g. nodes in a decision tree. In general, we give a simple methodology that can be used to achieve interpretable features and classifiers as an alternative to methods like Topic Models, and give insight into the parameters required and qualitative results that can be obtained. We extensively test the qualitative and quantitative results, finding that the highest-performing quantitative results also make good intuitive qualitative sense. We find that our method greatly outperforms the original representations on low-depth Decision Trees, giving good evidence that we have disentangled the representation. Additionally, we find that we are also competitive with standard interpretable representation baselines in most cases. We introduce variations to the original work that produced these kind of interpretable representations, in particular finding that scoring directions using NDCG performed better than Kappa in most cases, and that we could achieve much stronger results than the original clustering method using K-means. Further, we experimented using a variety of space-types and domains, verifying that the methodology can be applied more generally than shown in \cite{Derrac2015}. The main experiments that would be interesting to expand on for this chapter would be more state-of-the-art representations, specific investigations of how those representations are able to achieve such strong results, and interpretability experiments to see how our cluster labels fare in real-world situations.

\chapter{Directions in Neural Networks}\label{chapter5}

\section{Introduction}

% Neural networks are not interpretable

% However, they clearly contain a lot of useful information, as they perform well at tasks

% There is usually a theory that these networks contain more complex relations the deeper they go, or more 'abstract' dimensions.

% Three questions, first: Is it possible to obtain these more complex, accurate to domain knowledge relations and use them as features of an interpretable classifier using the methods in Chapter 3? 

% Second: Can we discover what kind of relationships neural networks are finding by investigating the directions that we will use as features in the interpretable represenation obtained from the network? 

% Finally: In networks which obtain successively more abstract representations, can we create a mapping of specific to abstract concepts?

% The contribution of this chapter is as follows:

% We look at feed-forward networks and auto-encoders. The idea with auto-encoders is that we will be able to map specific terms to more complex ones by reducing the size of the vector space. With feed-forward networks, we want to see how the neural network achieves strong performance on a text classification task, aka how it transforms the representation such that it can achieve strong results on a supervised task.

% The results for auto-encoders was this, feed-forward networks was that

% Summary/whats coming next

\section{Background}

\subsection{Explanations}

% Explanation is a huge field

% The different kinds of explanations

% Where our directions fit-into this paradigm


\subsection{Rules and Neural Networks}

% We plan to use rules in-order to create a kind of mapping of domain knowledge from specific to general using auto-encoders

% There is a history of using rules to understand neural networks, pedagogical etc

% Where our method fits into this


\subsection{Investigation methods}

% There are many visualization/investigation tools

% Where our directions fit-into this paradigm

\subsection{Feed-forward networks}

% Feed-forward networks 

% Basic layout, input/hidden/output, weights

% Activation functions (Relu, Tanh) refer to paper

% Loss (binary_crossentropy), trainer, Adagrad

% Dropout

% Thresholding


\subsection{Auto-encoders}

% Input/output is the same

% Hidden layer constrained in some way so that the representation is more general

% Denoising auto-encoders

% Variations of auto-encoders in recent years. We just use the basic one.

\section{Method}

% Generally, we treat the hidden layers of neural networks as vector spaces.

% We use the same methods as described in chapter 3, obtaining directions and rankings

% Using these directions and rankings, we obtain interpretable representations

\subsection{Auto-encoders}

% Using the interpretable representation, we create a mapping from layer-to-layer using a rule-based classifier

\subsection{Feed-forward networks}

These representations these networks build should differ from those obtained in the unsupervised representation in a few different ways. First, the neural network used is multi-label, meaning that all of the classification objectives are trained at the same time. Second, a non-linear activation function is used, sometimes in multiple layers. Finally, the supervised objective should shape the space more than an unsupervised objective, although only objectives that are relevant to the domain are considered in this case - as we are not interested in a network that ignores a large amount of information, rather we want the network to construct a representation that more accurately represents domain knowledge.

 Following results showing that simple neural networks can still perform well on text classification without needing a complex architecture  \cite{Lakhotia2018} \cite{Nam2014}, we use a neural network with the following changes: Cross-entropy loss, Adagrad trainer with default parameters, Dropout, Softmax activation on the output layer, ReLu or tanH activation function. 

We trained neural networks with non-linear activation functions so that they would  perform better than linear SVM's on  document representations. The assumption behind this is that if the network is able to achieve stronger results on the task than a baseline representation, it must be doing something more than representing the information. In other words, the representation must be better for the task than an unsupervised representation. The goal of this section is to discover why neural networks are able to perform so well, and explain it in terms of directions. 

To learn a baseline feedforward network, we take the highest performing representation found in Table \ref{bg:repsresults} for clusters on depth-3 decision trees and use them as input representations.   For the newsgroups, this is the size 100 Doc2Vec space. Then, we set the hidden-layer to be the same size as the input representation and use a non-linear activation function.  The reasoning behind this is twofold: First, we can assume that the representation that performs well on depth-three decision trees contain good interpretable concepts that can be meaningfully adjusted by the network, and second that this allows for easy comparison between the original representation and the representation taken from the hidden layer of the network. These feedforward networks are simple, but do achieve stronger results than the linear SVM.  

The  differences between the supervised representation and the unsupervised representation are examined here, with the main question being how exactly does a neural network achieve such strong results on the task? Some general ideas or explanations that are given are that through non-linear transformations, the internal representation of the neural network is able to find a more complex or accurate representation of domain knowledge relevant to the task.


We additionally learn a neural network starting from the bag-of-words  with two hidden-layers, the first of size 1000 and the second of size 100. In preliminary experiments, the network with two layers outperformed the single layer network consistently, and this neural network performs much more strongly than the one starting from the Doc2Vec space. 

% We simply take the hidden-layer representation and look at what's going on. 



\section{Parameters}

We use three different domains: Newsgroups, Placetypes and Movie reviews. We choose newsgroups and movie reviews as they have a relevant task that can be used to verify if the network has learned general domain concepts, e.g. in the case of newsgroups the natural categories of the documents and the movie reviews the genres of movies. Placetypes are chosen not because we expect them to perform well with neural networks, but rather because their entities are easy to understand and in-turn explain. In fact, as placetypes has such a low number of entities for its tasks we cannot expect good results with neural networks which typically perform well with a large number of entities.

When obtaining the directions, we set the frequency cut-off to the top 10,000 words and the direction cut-off when classifying to 2000 features. This is arbitrary, as both of these cut-offs need to be tuned for the specific space-type and task in-order to achieve strong results. The reason that these parameters are not tuned in this case is because we are interested in the qualitative nature of the directions, not the performance on the text classification task. 

We investigate the task using the newsgroups as it gives a good example of a representation that contains many different concepts and aspects of domain knowledge, as well as a well defined classification task that if achieved well on clearly demonstrates the representation contains a good amount of information in the domain. The place-types task does not contain too many entities, making it unreliable for neural network training and the reuters task has similar problems. One alternative would be the movies domain, but as identifying the genres of movies is functionally similar to the newsgroups task, we assume that any results found for the newsgroups can easily generalize to any classification task.

We tune the network for the following values: 

epoch = [100, 200, 300]
activation_function = ["relu", "tanh"]
dropout = [0.1, 0.25, 0.5, 0.75]
hidden_layer_size = [1,  2, 3, 4]

When using the bag-of-words as input to the neural network for the newsgroups, we found that it worked well with default parameters.  The parameters for this network are not tuned as the baseline network is, instead an initial hidden layer of size 1000 was chosen following recommendations in \cite{Nam2014a} and the secondary layer was chosen as it has a size similar to the Doc2Vec space and converged quickly. As the method described in chapter \ref{ch3} scales poorly to larger dimensional spaces, this secondary layer was necessary. The batch size is 100 for all experiments, excluding the place-types which used a batch-size of 10 as there were so few entities.

\subsection{Feed-forward Networks}

% We use hyper-parameter optimization to get the best performing neural network

\section{Quantitative Results}

Here, we show the results for the neural networks, the results for the directions obtained from the hidden layers of those networks on depth-3 decision trees, and the results of the clustering of those directions on depth-3 decision trees. The intention of these results is not to achieve state-of-the-art on the task, instead it is to create networks that can perform reasonably well on the task given a simple configuration. This is because the goal of this work is to use directions from neural networks as feature and investigatory tools, and to do that we need to know if the directions obtained from the network are still usable as features.

So, this section is not intending to show that supervised models outperform unsupervised ones, as that is not necessarily the case. Rather, it intends to provide insight into what the directions in the neural networks are using the quantitative results. The main focus is not on if it outperforms other results, but rather how it performs. If it does perform well, then that gives some insight into the neural network, in particular that it must be arranging the space in a better way than an unsupervised representation can to perform well at the associated task. Similarly if it performs poorly, it must be arranging the space poorly. 

These results are designed to give these kind of preliminary insights into what we expect the directions to be in the qualitative analysis in Section \ref{ch5:qual}.

\subsection{Feed-forward networks}

\subsubsection{Neural Network Results}

Compared to newsgroups, the neural network that used a vector space as input performed significantly better than the linear SVM on the unsupervised representation. It obtained an F1-score of 0.574, when the linear SVM's score was 0.532. The assumption follows that if it actually has found more meaningful concepts in the space, then the directions should also perform well on a depth-3 decision tree.

For the place-types domain, the neural network that used a vector space as input performed significantly worse than the SVM, as it scored 0.53 in F1 score compared to 0.622. However, this is not particularly meaningful as there are so few entities. Similarly for the network starting from a bag of words, it scored 0.58 at best.

In the movies domain, the network that used a vector space as input performed better than the original by a large margin. The original scored 0.53 and the network scored 0.57.

Obtaining the directions for spaces transformed by the relu activation function were a lot slower than those of the tanh activation function.


\subsubsection{Directions}


\subsubsection{Clusters}

\section{Qualitative Investigation}\label{ch5:qual}

\subsection{Feed-forward networks}

\subsubsection{Neural Network Results}

Compared to newsgroups, the neural network that used a vector space as input performed significantly better than the linear SVM on the unsupervised representation. It obtained an F1-score of 0.574, when the linear SVM's score was 0.532. The assumption follows that if it actually has found more meaningful concepts in the space, then the directions should also perform well on a depth-3 decision tree.

For the place-types domain, the neural network that used a vector space as input performed significantly worse than the SVM, as it scored 0.53 in F1 score compared to 0.622. However, this is not particularly meaningful as there are so few entities.

In the movies domain, the network that used a vector space as input performed better than the original by a large margin. The original scored 0.53 and the network scored 0.57.

 


\subsubsection{Directions}

Two groups of single word directions and clusters were obtained from three different domains, in each one is from the second layer of  a two-layer neural network starting from a bag-of-words and the other is from the first layer of a one-layer neural network starting from an unsupervised representation. The examples in this section for the clusters are from the clusters that performed the best in depth-3 decision-trees. When investigating single word directions, sometimes it is unclear what they mean. In this case, the most similar single word-directions are found and used to provide additional context.

% Compare the same directions but see how their rankings have changed



% Compare the top directions for NDCG, top for Kappa



\subsubsection{Starting from vector space}

\subsubsection{Starting from bag-of-words}

Obtaining directions from this space takes much longer than normal vector spaces.

\subsection{Auto-encoders}

\section{Conclusion}



 Neural network models that encode spatial relationships in their hidden layers have achieved state-of-the-art in Text Classification by using transfer learning from a pre-trained Language Model \cite{Gong2018}. There have also been neural network models that produce an interpretable representation, for example InfoGan.
 Most state-of-the-art results rely on Vector Space Models. Ideally the method would be able to achieve strong results for simple interpretable classifiers by transforming an existing representation that performs well at the task.
 

\subsection{Chapter 3 Space Types}
\begin{landscape}
	\begin{table}[]
		\begin{tabular}{llllllllll}
			& Genres     &         &         & Keywords  &         &         & Ratings  &         &         \\
			Movies            & D1         & D2      & D3      & D1        & D2      & D3      & D1       & D2      & D3      \\
			Space             & 50 PCA     & 50 MDS  & 100 MDS & 200 PCA   & 200 MDS & 200 MDS & 50 PCA   & 200 PCA & 50 PCA  \\
			Single directions & N/A        & N/A     & N/A     & N/A       & N/A     & N/A     & N/A      & N/A     & N/A     \\
			&            &         &         &           &         &         &          &         &         \\
			& Newsgroups &         &         & Sentiment &         &         & Reuters  &         &         \\
			Rep               & 200 PCA    & 200 PCA & 100 PCA & PCA 100   & PCA 50  & PCA 50  & 200 PCA  & 200 PCA & 100 PCA \\
			Single dir        & 200 MDS    & 100 D2V & 50 D2V  & D2V 100   & PCA 50  & D2V 100 & N/A      & N/A     & N/A     \\
			&            &         &         &           &         &         &          &         &         \\
			& Foursquare &         &         & OpenCYC   &         &         & Geonames &         &         \\
			Placetypes        & D1         & D2      & D3      & D1        & D2      & D3      & D1       & D2      & D3      \\
			Rep               & MDS 100    & AWV 50  & MDS 200 & AWV 50    & MDS 200 & AWV 50  & MDS 50   & MDS 50  & AWV 200 \\
			Single dir        & N/A        & N/A     & N/A     & N/A       & N/A     & N/A     & N/A      & N/A     & N/A    
		\end{tabular}\caption{Space-types, clusters have the same as single directions.}
	\end{table}
\end{landscape}




  %This elevates neural networks from simply being representation learners to models that can discover and create new ideas in their representations in-order to solve the task.

Theoretically if we are able to identify features from each layer, then we can also map how they interact with each other. 









%For each domain, we test two different network set-ups. The first is one where the hidden layer is the same size as the input layer, and the second is where the hidden layer is smaller than the hidden layer. The assumption is that by constraining the hidden layer size, we can force the network to generalize and find more abstract concepts in the hidden-layer, hopefully gaining some insight into the domain by comparing the input representation and the more general hidden representation.

There are no additional techniques applied to improve the performance of the neural networks.
\chapter{Fine-tuning Vector Spaces to Improve Their Directions}

\chapter{Conclusion}\label{ch6}


%Introduction+conclusion are shorter chapters

%Introduction
%Hypothesis
%research questions
%context
%Summary of the contributions
%Good to introduce gently but research questions can be fine
%Doesn't have to be accessible to the man on the street, need computer science background, vector spaces are included in that, so are decision trees and neural networks, if it's borderline reference background

%%Conclusion revisits research questions, provides the answer not just a summary

\section{Introduction}


%%%% WHAT IS THIS SECTION ABOUT? WHY AM I READING IT? WHAT IS IT?
This \hmark{c}hapter provides conclusions on how the work in this thesis contribute\hmark{s} to the hypothesis and research questions. 

\section{Thesis Summary and Contributions}

In Chapter \ref{ch1} the hypothesis was introduced. It was as follows:  

\begin{quote}
\hmark{\textit{Semantically meaningful features can be obtained from vector space representations of documents. These features are sufficiently predictive to be useful for simple interpretable classifiers, of which low-depth decision trees are a prototypical example, allowing for a performance that is close to the performance of an unbounded decision tree using the same features as input. Further,  these feature can be obtained from neural network hidden layers, and can provide valuable insights into  aspects of the domain that the neural network has learned. }}
\end{quote}

In Chapter \ref{ch3}, disentangled feature representations obtained using the methods in this work performed better than the original unsupervised representations when used as input to low-depth decision trees. These results showed their potential \hmark{as semantic features for interpretable classifiers}, \hmark{especially} as the classifiers in some cases performed better than linear \hmark{SVMs} trained on all features of the original unsupervised embedding.  The qualitative investigation found that the features are clear and meaningful, and these features were used to determine the difference between document embeddings and parameters of the method. The NDCG scoring metric introduced for use as a scoring metric for directions in this work was found to generally be a good choice, in particular because it had the most consistent performance when the resulting features were used as input to low-depth decision trees. \hmark{Varying the clustering method was found to be successful in some cases. }

In Chapter \ref{ch4} feed-forward and auto-encoder neural networks were \hmark{quantitatively and qualitatively} investigated. Characteristics of feed-forward networks that used BOW as input were identified and compared with unsupervised representations\hmark{. For} feed-forward networks, we found that higher-quality feature directions could be found, for those features that were relevant to the classification task on which the network was trained. Furthermore, we noticed that terms that were less relevant to the classification task scored less. Auto-encoders with increasingly low-dimensional embeddings were qualitatively and quantitatively investigated, and \hmark{despite not being effective at learning natural categories, they seemed to capture} increasingly abstract features. \hmark{This was validated quantitatively.} Finally,  a rule-based classifier that connected the features from layers together was used to further investigate what the neural network was doing. %Although the results were preliminary for the auto-encoder, they lay a methodological foundation for connecting together the layers of neural networks. 

In Chapter \ref{ch5} a methodology was introduced to fine-tune the original document embedding that the feature representation was obtained from. This was done in order to improve the directions, and in turn improve the associated rankings. It was found to improve the scores of the feature representations when used as input to low-depth decision trees. Additionally, a qualitative investigation showed that the entities were more intuitive and specific than before, and that noise had been eliminated.



% These low-depth decision trees when the disentangled feature representation was used as input, obtained from unsupervised representations were disentangled the, and a qualitative investigation was conducted to further verify that the dimensions correspond to clear semantic features. The disentangled feature-representation had strong performance on low-depth decision trees, sometimes matching the performance of the original representation when classified with a linear SVM. 

\section{Research Questions}

In the introduction, three research questions were asked. This section repeats those research questions and discusses how they were answered in the thesis.

\hmark{\textit{\textbf{Question 1:} Can meaningful semantic features be characterised as directions across a wide range of vector space encodings and domains, and do these semantic features allow us to learn effective low-depth decision trees?}}

In Chapter \ref{ch3} it was found that feature representations derived from unsupervised vector spaces contained clear semantic features, resulting in good disentangled representations. Chapter \ref{ch4} also provided some quantitative analysis of  disentangled representations obtained from the hidden layers of neural networks, finding that their performance when used as input to low-depth decision \hmark{trees} was close to, matched or exceeded the performance of the original network on the task. In Chapter \ref{ch5} the associated rankings of these features was made more clear and their performance on low-depth decision trees was improved, resulting in an overall better disentangled representation using an entirely unsupervised method.

\hmark{\textit{\textbf{Question 2:} Can semantic features be obtained from the hidden layers of neural networks, and to what extent can these features be used to  investigate  the characteristics of   different neural networks?}}

\hmark{From the qualitative analysis of the hidden layer representations in Chapter 5, it was clear that the} properties derived from these representations was related to how well they classified the task. \hmarkn{Quantitatively, properties derived from feed-forward networks when used as input to low-depth decision trees allowed them to achieve good performance classifying key domain tasks.} Some characteristics of feed-forward networks were identified, in particular that they are representing terms related to the task better than the unsupervised representations, and that \hmark{that they reduced how well terms unrelated to the task are captured in the space.} Additionally, increasingly low-dimensional auto-encoders were found to capture increasingly abstract properties.

\hmark{\textit{\textbf{Question 3:} Is it possible to obtain higher-quality semantic features in an unsupervised way by fine-tuning the initial vector space?}}

In Chapter \ref{ch5} an unsupervised methodology was introduced that fine-tuned the initial vector space, resulting in higher performance of low-depth decision trees  when the improved disentangled representations were used as input. Additionally, qualitative results showed that the entities had improved. 


\section{Future Work}

% Verifying the interpretability of the disentangled feature representations normal and fine-tuned
%		How will you verify the interpretability?thi
%		What can you compare it to?

This thesis has investigated the disentanglement of  vector spaces, but it has not experimentally validated how interpretable the  labels of the features  are. To do so, the intruder task from topic models could be used. Each group of words in a label (either the cluster words or the single-term word\hmark{s} and the words associated with its most similar directions) is given an intruder word from another cluster. The feature representation performs well if users are able to identify the intruder word. Another interesting test would be to use a domain, e.g.\ text in medicine, where some text classification task is required and verify that the decision trees make sense to expert users like doctors. 


% What form can explanations take

Although decision trees are interpretable, they may not make sense to end-users. One avenue for future work that will benefit many other applications of this method is converting low-depth decision trees that use disentangled features into human-readable explanations that can be understood by users in the domain. Although this might seem straightforward (e.g.\ the explanation could just be of the form IF Movie has "Horror", and it is "Romantic", then it is a Romantic Horror movie) it can be a bit more complicated when using features that do not have such a natural intuition or classifying things that cannot be put into such a simple format (e.g.\ Compare "IF Blood AND Gun THEN Mobsters" to IF there is "Blood" in the movie, and there are many "Gun" scenes, then the movie is likely about "Mobsters"). These kind of natural language explanations could result in better adoption of the method in real-world domains.

% Explaining neural networks with a proxy model low-depth decision tree
%		How will you ensure the model matches the predictions of the orignal neural network?
%		What form will the explanations take

This work also has potential applications  to eXplainable AI (XAI), methods that explain existing learned neural network models. Essentially, the approach outlined in \hmark{Section} \ref{ch4} would be used to create a "proxy model" that approximates the  behaviour that the neural network does in a simple way. Then, the quality of this proxy model is determined by how well it matches the predictions that the neural network made. Following the previous idea about transforming decision trees into natural language explanations, the method could be applied in a variety of domains to explain existing models to end-users.

% Creating alternative interpretable models that match or outperform state-of-the-art models
%	 	Which interpretable classifiers?
%	 	Which models? How will you do it?

Models learned from disentangled feature representations derived from neural network hidden layers could also act as interpretable alternatives to them. This brings up a primary question that follow-up work could address: can disentangled feature represenations be derived from neural networks that achieve state-of-the-art e.g.\ BERT \cite{Kenton1953}? BERT is a neural network that achieves strong results on a variety of tasks and has entered mainstream application usage,  but can the hidden layers of BERT be disentangled into interpretable features, and can an interpretable classifier be learned? This is but one of a variety of potential  state-of-the-art models that could be disentangled  to produce an alternative interpretable model.

% Using Rank SVM to learn the rankings rather than taking the orthogonal direction

In Chapter \ref{ch3} the linear SVMs were used to obtain hyperplanes that classify words on a binary occurrence task, i.e.\ if the word occurred or if it did not occur in the document. Following this, the orthogonal direction was taken and \hmark{documents} were ranked on it using the dot product. It would be interesting to instead try RankSVM \cite{Lee2012} to directly learn rankings of documents on words, where the "true ranking" is determined by the Positive Pointwise Mutual Information (PPMI) score. This would potentially result in a more accurate ranking derived from the spatial representation for the word.


% Improving the clusters by ensuring that they are not disrupted by noisy words w/ post-processing step?
%		What is the problem?
%		What is your idea for solving that problem?
%		How will it work?

In Chapter \ref{ch3} a potential problem with cluster features was identified, in particular that their  meaning may be disrupted due to terms unrelated to the task being clustered with them. It would be interesting for a qualitative investigation into this phenomenon, seeing if manual removal of words from clusters would indeed benefit their performance in low-depth decision trees. If this is the case, there is potential for a new clustering method that performs this process automatically. Theoretically this could be done by taking a bag of PPMI scores as in Chapter \ref{ch5, and if the similarity between the ranking of \hmark{documents} on the cluster direction and the bag of PPMI scores drops too far with addition of the new word then it could not be added. 


% Fine-tuning state-of-the-art neural networks such that their internal representations are disentangled
%		How will you do this? 
%		What will be different from preliminary tests?
%		What is the goal of this? How will it achieve that goal?


The fine-tuning process outlined in Chapter \ref{ch5} originally followed the results demonstrating  direction scores went down in when obtained from neural network hidden layers as they are  stacked. Given this, the idea was formed that it would be possible to fine-tune these spaces such that they can retain the semantic features in the first layer while also adjusting the representation according to the objective. However, in preliminary tests this did not yield such a result, instead of fine-tuning the representation it stopped learning information related to the task. This preliminary investigation was done by taking each resulting representation, fine-tuning it, and then using it as input to the next auto-encoder. There is potential future work in pursuing this idea in application to explainability and interpretability. In particular, work in investigating if neural networks can be forced to retain disentanglement during learning. If this is the case, it may be possible to make neural networks interpretable.




\section{Summary}

In summary, this thesis demonstrated that disentangled feature representations can be obtained from a variety of document embeddings, and a qualitative analysis was conducted on these representations. It was found that the features were clear and meaningful. Additionally, a method was introduced that fine-tunes vector spaces to improve these disentangled features. The methods seem promising in application to interpretability and explainability and offer much potential for future work.

%This thesis experimentally validates and improves on a methodology for obtaining disentangled feature representations for text documents. It is validated using document classification tasks, and disentangled features from neural networks are  qualitatively  analysed. The method is unsupervised, acts as a post-processing step on vector space representations, and  disentangles important semantic relationships in the space.  


\backmatter 

\bibliographystyle{plain}
\bibliography{bibliography}


\end{document}
