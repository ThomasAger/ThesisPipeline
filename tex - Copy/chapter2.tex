\chapter{Background}
% What text representations actually are
% Why text representations are useful, what they are used for, how do they relate to our work, how do they relate to interpretability
% Classification, what classification is, example with Decision Tree
% What semantic spaces actually are
% Examples of classification with SVM's 
% What interpretable representations actually are, desiradata/features of interpretable representations
% Examples with Decision trees, examples of classifiers that produce an interpretable result automatically



% What are text representations composed of? Why do text representations matter? Why does the other stuff matter?  How will it be used in this thesis? Why should I read this chapter?
%Labels, features, models, representations, regression, classification

\section{Introduction}

This Chapter begins by explaining the general process required to solving tasks with machine-learning starting from raw text data. The steps of this process are expanded on in the later sections.


Similar to how it would not be possible for a human to solve a problem without a good understanding of the subject area, the first step of solving a problem with  machine-learning  is to obtain a suitable  representation of the data. In this thesis a representation must be obtained from raw text data that is effective for the task of document classification. Document classification is the task of distinguishing between entities in a domain, where entities are e.g. movies in a domain of movie reviews, people in a domain of twitter posts, or reviews in a domain of product reviews. 

As the task is to separate entities, after data collection the first step of solving the task is to separate the corpus of text into a document for each entity. Then, the text is pre-processed so that noise is removed, noise in this thesis being information in the text that is not useful when solving associated document classification tasks. For example, metadata like the e-mail of a movie reviewer in movie review text, or unnessecary punctuation and grammar. It's important to remove this information at this stage as raw text data is easy to manipulate and the the result of any modifications can be clearly seen. If we tried to remove this kind-of information after obtaining a representation, it would be a much more complex process.

One popular and simple representation is the bag-of-words. The bag-of-words  represents an entity as a vector where each element corresponds to a unique word in the corpus. The values of these elements are usually some statistic related to the words importance in the document e.g. word frequency where if a word occurs five times in a document the value is five. One disadvantage of this representation is that it doesn't retain the context of words, another is that it is sparse, as each vector has an element for every word in the corpus and only some of them will have a frequency above zero. This means that the to store it and process it in memory efficiently, specialized data structures and machine-learning techniques must be used. However, it has the advantage of being easy to understand for humans as each element of the vector representation for each entity corresponds to a word. 

Ideally the number of dimensions would be reduced while retaining the information. One method of doing this would be hand-crafted feature selection, where words which are identified by experts as not meaningful are not included as an element. However, if it is done manually it would take a large amount of time and require expert knowledge, and  if this is automated a lot useful information can be lost to make the size of the data manageable. An alternative approach is to use the vector similarity between entities, e.g. the similarity between their frequency BOW vectors, to produce a low-dimensional vector space, where entities are encoded  such that their semantic similarity matches their spatial similarity.  In this space, the vectors that correspond to entities are just co-ordinates in a e.g. 2000 dimensional vector space. However, this results in vectors whose elements are no longer interpretable in the sense the bag-of-words is, instead information is stored as similarity relationships between entities in the vector space.



%For humans, our understanding of how things behave and what they are is a representation of reality,   produced from our experience with those things in the real world. Each person is constantly adjusting to some degree their representation of what different things are, and this construction and transformation of our personal representations is what we call learning. 

% Machines are computational and can process large amounts of information, but it must be in a suitable representation for them to use it.  For tasks in the domain, e.g. categorizing Amazon product reviews into positive sentiment or negative sentiment, it is usually required that the representation encodes fundamental domain knowledge. Put in terms of human intelligence, it is akin to how we must have some fundamental representation of the domain before we can appropriately act in it.

%In this thesis the data used is raw text, so a scalable computational representation of the meaning of text data is required.  There are a variety of methods to achieve representations of raw text data, and each one leaves out some information from the domain and prioritizes other parts. This makes each suited to solving a different problem. For example, the bag-of-words representation is called as such because word-context is ignored, but on a task of sentiment analysis word context is important, e.g. a review for a movie that contains the sentence "This was so good I want to rip my eyes out." although sarcastic, would still count as a compliment. To achieve better results on this task, it would be better to instead use a representation that can represent context somehow. In section \ref{ch2:representations} we cover how to obtain different kinds of representations and explain how they work.

%Determining  sentiment of e.g. an Amazon product review is one of many domain tasks that machine-learning can solve. This is specifically the task of categorization or "classification" that naturally occurs in many domains. For both humans and machines, the essential question when making these decisions is if the representation we have of the entities in the domain is accurate to the reality, i.e. if it represents Amazon product views well. As with the methods, there are many different ways to classify using machine-learning given a representation. However, there is no way a machine-learning classifier can perform well on a task without a good representation. Different classifiers that are used in this thesis are covered in section \ref{ch2:classifiers}.

%Growing up learning a language, people learn the complexities of that language intuitively and can recognize if something is wrong without necessarily being able to verbalize why. This is the problem of a representation that is difficult to express, and in-turn difficult to understand. In a similar way, representations for machines that can e.g. capture the context of words precisely can be difficult to understand for humans, as it is difficult to encode this complex information in a simple way. This is the problem of interpretability, taking representations and classifiers that are only intelligible to machines and making them understandable for humans as well. There are a variety of ways that more complex information can be represented interpretably, and this is covered in Section \ref{ch2:interpretability}. 






%There are a variety of different text representations, but the most common is the bag-of-words. A bag-of-words ignores word-context, instead using the frequency of terms in a document as its text representation. This representation is a matrix, where each column is a word in the vocabulary of the domain and each row is a document. 


%\subsection{Classification Problems}





%One example of an unsupervised machine-learning task is to transform data into a representation that can be learned from by the 



%\subsection{Interpretablility}

%Simple rules are a good basis for an interpretable classifier. However, this is only possible when the features (e.g. the word frequencies in a bag-of-words) are clearly defined. This forms the basis of what an interpretable representation is viewed as in this thesis. If the features correspond to some meaningful property in the domain, e.g. "good" or "thrilling", then that is treated as an interpretable representation. In this sense, a bag-of-words is an interpretable representation, despite the approach in this thesis being different from the methods a bag-of-words employs.


%\subsection{Vector Spaces}




%\subsection{Conclusion}

As mentioned in Section \ref{ch1:contributions} the main focus of this thesis is in transforming vector spaces into interpretable representations while retaining their information. This Chapter introduces the process of obtaining a representation from text data and using it to solve machine-learning problems, as well as giving a general introduction to related work. To outline the process, first as covered in Section \ref{ch2:data} the data is preprocessed so that unnessecary information is removed. Then some basic representations are obtained in Section \ref{bg:BOW} and this is followed up by more complex vector space representations \ref{ch2:vectorspaces}. To complete the process, Section \ref{ch2:classifiers} covers different machine-learning methods to solve problems using these representations. Finally, interpretable representations and classifiers are covered in Section \ref{ch2:interpetable} to give context in the literature for the work in the next three Chapters. 

%With the rise of services on the web that enable large-scale user-generation of text data,, the internet has become largely populated by text posts that are related to some specific, niche topic within a domain. For example, a review on Amazon for a product is specially tailored text for that product within the domain of Amazon reviews. Taken from a closer lens, we could even argue that each review-type has its own domain, e.g. Product reviews, Food reviews, Movie reviews. However, the text posts themselves are largely unstructured semantically. 

%Text data. Why text data? What useful applications does text data have? 

%The availability of text data has expanded as technology, in particular the web,  has taken a larger part of our day-to-day lives. The availability and volume of this text data has driven research into how to use it, solving problems like automatic language translation, predicted terms, or even detecting if a patient acquired an infection in a hospital from their recorded text data \cite{Ehrentraut2018}. The first question that needs to be answered is, 

%What is this section about? Why is it here? What will they get out of the end of this section? 

%In this Chapter, the fundamentals are covered that are required to understand how to go from text data to machine learning models that are useful in our day-to-day lives. The future Chapters introduce a new question following the previous paragraph, how can we make a representation that is intelligible to both humans and machines, and how can it be applied? In particular, Chapter 4 conducts a deep experimentation into simple interpretable models, Chapter 5 uses these models to gain insight into what other models have learned, and Chapter 6 refines them so that they perform better and are easier to understandw. 

%How do machines understand text data? How do machines use text data? 

%In the case of this work, we look at text split into documents, where e.g. in a collection of imdb movie reviews, each movie would be a document composed of all of its reviews. Exactly what composes a document depends on the dataset, but it is generally longer than a sentence and is an object in the domain, e.g. a news article in a domain of news websites

%Where does the background go from here? What do the remaining sections cover?

%This thesis is about learning interpretable features (See the discussion in Section \ref{ch1:interpret}). In this case, this means that each feature is labelled with words so that humans can understand what that feature means, e.g. in a domain where each document is a movie review, a bag-of-words is interpretable as it has the label "Scary" for a feature and its frequency value. However, this is only useful if we desired to classify

%This Chapter continues as follows: First we go-over the bag-of-words and improvements for it, then we move on to how to obtain representations that model more complex relationships between documents using a variety of methods. From there, we explain a few different types of classifiers and finally give more specific context for the work in the thesis, describing interpretable representations and other related work.



%These tasks can range from Document Classification where documents are organized into categories e.g. separating news articles into "World News" and "UK News" based on their text, to Sentiment Analysis where if a text document like a movie review is classified as positive or negative. Each of these tasks requires a different representation to be most effective, for example ideally when learning a Sentiment Analysis task the representation would model sarcasm and context e.g. representing that "Now, this wasn't bad" doesn't carry the same meaning as "Now, wasn't this bad". 


%Text documents, pre-processing


%Bag-of-words

%Vector spaces

%Word-vectors

%Linear SVM's

%Decision Trees

%Isotonic Regression

%Topic Models

%Clustering

%Interpretable representations



%Accuracy and F1

\section{Text Data}

This thesis is focused on producing interpretable representations from text data, and solving specific problems in text domains. In this section, the basics of what the text data is, terminology associated with it and how it is preprocessed is described.

\subsection{Text Domains}

"Text domain" refers to a subject area that is unique in its vocabulary and structure. One example is the newsgroups domain (See Section \ref{data:datasets}), which is composed of online news discussion groups from 1995. In this domain, there are subject areas, topics and posts. Each subject area has topics that users create, and each topic has posts that users respond with. Within each of these subject areas, specific jargon and a unique structure specific to that subject area and the overall domain has developed. Below an example post is provided from the newsgroups domain that contains unique jargon like "NOEMS", "EMM386" and a unique structure e.g. signing the post with the persons name and a personal tagline for contacting them.


\begin{quote}
Has anyone else experienced problems with windows hanging
after the installation of DOS 6?  I have narrowed the
problem down to EMM386.

If if remove (or disable) EMM386, windows is ok.  If EMM386
is active, with NOEMS, windows hangs.  If I use AUTO with
EMM386, the system hangs on bootup.

Dave.


-- 

-------------------------------------------------------------------

David Clarke   ...the well is deep...wish me well...

ac151@Freenet.carleton.ca  David\_Clarke@mtsa.ubc.ca  clarkec@sfu.ca
\end{quote}

These particularities to the domain are what makes the distinction between domain-specific text data to general text data. A machine-learning model will develop a representation of how to solve the task dependent on this data. If the model is given general text data examples to learn from, then it will miss out on domain-specific quirks that can help it solve the task e.g. when learning to identify if a newsgroups post belongs to the subject of windows, if the general examples do not use jargon like "AUTO" and "EMM386" then this important information will not be used. However, if the examples given are over-specialized then the model may place excess importance on domain-specific quirks that are not actually meaningful, e.g. if in the training examples of posts about windows most users signed off their text-posts with an email that includes ".ca", meaning they are from canada, then the model may identify all posts that include ".ca" emails as about windows despite this simply being a strange quirk in the data.



%One goal of machine-learning is to predict if a piece of text will be shared or liked by users. In this case, it is clear that in-order to determine if a movie review will be liked or shared, it is difficult to determine if that will be the case when using the same logic that you would for a facebook post. Facebook posts that trend or are well-liked are typically brief, easily consumed and focused on humour. Meanwhile, movie reviews that trend are due to cutting and intelligent analysis of a movie in a relatable way. 


 %text data available from a domain is unique to that domain in many ways, for example on the social media site Facebook text data can be formatted into posts and comments, and is posted by users. Although a movie review by a critic could on the surface have the same structure - a post with comments below, the rules governing what is contained in the post are clearly not the same. 



%The main point here is not that these domains are completely different, but rather that there are meaningful differences between between text from the domain and text outside of it, and although there are different rules for each domain there are still common trends between them. Typically in machine-learning the methods that perform the best use some-kind of large-scale data that is not directly related to the domain as well as a lesser amount of domain data.


% Text from different domains is formatted differently, e.g. posts and comments
% Corpus are usually split into documents because of its associated tasks, give examples?
% Text from different domains differ in language
% Tasks in different domains are different
% Text from different domains perform better using different representations and classifiers
% There are similarities between domains that is taken advantage of e.g. using word-vectors
















%EXAMpLES SOURCE ETC




\subsection{Pre-processing Text Data}\label{ch2:data}


% Corpus are split into documents

Text documents in a domain usually reflect task that needs to be solved. In a domain-specific classification task, e.g. identifying the genre of a movie in a domain of movie reviews from its review text, the subject of the task are entities in the domain, in this case movies. A natural way to arrange the data to create a document for each entity that contains all of its related data. For example, putting all the reviews for one movie in the same document. 

% We just use the raw text data without modifying it.

%There are a variety of ways to add additional structure to raw text, one-such way is to label parts-of-speech (known as POS tagging) like nouns, adjectives, or other grammatical constructs. This can be done automatically with reasonable accuracy, however there are not many datasets with this kind-of structured information available, and it is difficult to achieve reliable results without experts annotating the data, which is costly and time-consuming. This thesis focuses on how to use raw text without adding additional structure or information. Using raw text without additional structure enables the method will have broad applicability and allows easy comparison with other work.




% Building a vocabulary
% What parts of text data do we want to keep?, % What is noisy data?
% There are many ways to preprocess raw text data using expert knowledge, but we focus on an unsupervised and machine-learning approach
% Standard rules to reduce noise

To obtain a good representation of a corpus, the text data must first be processed so that the representation is built with as little noise as possible. What exactly noisy data is depends on the representation and the task, but for this thesis it can be seen as parts of the text data that are not meaningful when distinguishing between types of objects in the domain. Noisy text data can have a knock-on effect on the representations that are built from it, resulting in a much worse representation. If the email of a movie reviewer was retained in the review text, that will not be useful information for a task related to the movie. Additionally, you could also see  a word starting with an uppercase or lowercase as noise, as it is not information that will benefit the representation. %that if retained in the representation would harm it more than help it. Another example of noise that you would ideally like to remove is metadata,

% noise versus not noise - what does it mean?

The first stage of obtaining a bag-of-words is building a vocabulary $W_w$, composed of unique words $w \in W$ from the corpus. In this vocabulary, it is important that words which have the same meaning are not treated as different words e.g. if the word "Dog" was considered to be different to the word "dog." then the vocabulary would be too noisy. There are standard methods for removing noise in a dataset. We describe them in the following bullet-points: %In Table !!! we show some examples of noise in a domain.  %examles of noies!!!!

\begin{itemize}
	\item  Convert all text to lower-case, e.g. "The man and The Dog" converted to "the man and the dog"
	\item  Remove all punctuation including excess white-space, e.g. "the man, and the dog..." converted to "the man and the dog"
	\item Using a predefined list of "stop-words", listed in full in Table \ref{ch2:stopwords}, remove words that are not useful, e.g. "the man and the dog" converted to "man dog"
	\item Remove infrequent words, e.g. "man dog, dgo, dog man" converted to "man dog, dog man".
	\item Domain-specific pre-processing to remove metadata, e.g. removing emails from the end of movie reviews.
\end{itemize}






%When using a bag-of-words as a representation in a classification task, that threshold is usually set higher, as it is more important to remove noise that would not naturally be removed when using the bag-of-words to create another representation. 

%One way to remove noisy words is to apply a filter where words that are not frequent enough are removed. %Typically when creating a representation the threshold $T$ for this filter is low, e.g. two or one, so that only truly noisy words are removed. 

%%%%%%%%%%%%% Convert to a bullet point list with better organization

%However, if grammar and punctuation is removed then we can simply count that there are two occurrences of the word dog, resulting in a more robust way to count words. Some words are too common to be meaningful, to remove these words a list of "stop-words" is created - words like "the" and "and" resulting in a sentence that only contains meaningful words, e.g. "man dog dog man dgo". Finally, terms that do not occur more than a set threshold $T$ are usually pruned, with the lowest threshold being one. This is because terms which do not occur often are likely noise,   for example leaving us with a final representation of "man dog dog man", removing all remaining noise. 

%Despite removing some structure and making it less readable  at-scale this pre-processed sentence results in a better representation of the meaning of the text for machines. 

It can be assumed that for all of the methods that are described in this section, the  rules above are applied to the corpus beforehand. In terms of removing words that were not frequent enough, words that did not occur more than once are removed.  Although these rules are not universal, they are a good basis for computational methods of representing text data that do not rely on word-context and grammar. In the next section, we cover a variety of methods for text representation and explain their basic utilities.  %EXAMLES

\section{Text Representations}\label{ch2:representations}

% What is a text representation? Why is it usefl? What is a representation?



In the next Chapter of the thesis, as well as in section \ref{ch2:interpretability} how to make a representation that both humans and machines can understand is discussed, but this section  focuses on representations that are useful to machines when used for machine-learning, rather than being interpretable. In particular this section covers preprocessing data \ref{ch2:data}, sparse bag-of-words representations \ref{ch2:BOW} and obtaining vector spaces \ref{ch2:vectorspaces}.





%rom here, we can assume that 



%This thesis deals with text-document representations. Text documents are unstructured raw text data that have been separated according to their domain, e.g. Amazon product reviews separated such that each product is represented by all of its reviews, or news articles separated such that one document contains the data for one news article. One way to gain insight into this data is to count the frequencies of the words in each document. If a word is high-frequency for a document, then that word will likely be important to understand the meaning of that document. 

%Text data, for example forum posts, amazon product reviews, or news articles have become readily available as the digital infrastructure that supports our lives has grown. This data largely is unstructured, and cannot be readily processed by Artificial Intelligence tools without being reformatted. For example, if the text is separated into documents e.g. the raw text for one news article is put into a separate document from another.

%Need to write about the concept of salient features of a domain here.
\subsection{Bag-Of-Words}\label{bg:BOW}

Humans can have an intuitive understanding of the semantics that are present in unstructured text, but machines do not.  To obtain a machine intelligible representation of raw text it is not necessary to  annotate grammar and meaning symbolically, rather a simple representation that can scale to an extreme amount of data performs well. However when looking to achieve state-of-the-art results, representations that are more complex are usually used. However although a representation may be simple, with enough examples even a basic representation can have enough information to clearly distinguish between types of objects in a domain for a task. 

One of the most common representations is a bag-of-words (BOW) which ignores word context, instead taking the words that occur in each document and assigning a value to them in a matrix, e.g. where each word in a document is assigned a value for  its frequency in that document. For example, a short document of text like "there was a dog, and a man, and the man, and the dog" would be translated into word frequencies "there: 1, was: 1, a: 2, and: 3, the: 2, man: 2, dog: 2". This representation is simple, and ignores word context, grammar and punctuation but is highly effective when using machines to solve problems using a large amount of unstructured text documents. The bag-of-words is an important part of the work of this thesis, serving as the foundation of more complex and interpretable representations. 

As mentioned in the previous section, unnessecary parts of the data that are not  meaningful for the task should be removed. The bag-of-words is a representation that comes with the following assumption: the context of words is an unnessecary part of the data to perform well on the task. How correct this assumption is depends on the task, but despite this view being overly-simplistic the application and use of the bag-of-words (BOW) is broad. There are multiple ways to represent words in the BOW format, but the most common is by the frequency of the words in a document.

The natural structure for this kind of representation is that of a matrix, where rows are documents and columns are words in the domain as defined by their vocabulary. Specifically,  text documents in a domain $d \in D$ have an associated vocabulary of unique words across all documents $w \in W$. The bag-of-words $B_D$ is a matrix where each document is a row, and each column is a word, where the value of each word for a document is the word!!!s frequency in that document $d = ({wf}_1, {wf}_2, ..., {wf}_n)$ where ${wf}(d)$ is equal to the frequency of a word in a document and $n$ is equal to the number of unique words  in the vocabulary for all documents $w \in W$.

Given these vectors we can determine the similarity between two documents or two words by the similarity between their frequency vectors or document vectors. %!!!!!!!!!EXamles???

\subsubsection{Term Frequency Inverse Document Frequency (TF-IDF)}

There are two main problems of using frequency is that words which are frequent in the domain are given a higher value than words which are used frequently only in a single document. First, longer documents result in overall higher values than shorter ones. So for example if a Amazon product review was very long and repeated the word "good" 15 times, but the word "bad" 1 time, then compared to a short review that only used the word "good" one time the first product review is fifteen times as good as the second one. When building representations that use vector similarity (e.g. where the bag-of-words vectors are compared in similarity to each other) these kind of value adjustments are very meaningful, as the documents need to be normalized relative to each other.


 The second problem is that words that are frequent in many documents are given equal importantance to those that are frequent only in some documents. However, we are concerned with what distinguishes documents from each other so giving equal importantance  for example, in the domain of movie-reviews, to the word "movie" does not accurately represent how important it is for the meaning of the movie. Rather, we would be interested in terms that are frequent for only that movie review, as for example if the term "gore" was frequent in only five different movies out of 15,000 then it is clearly important for those movies. 


The idea that words which are infrequent overall but frequent for some documents are important can be applied to a bag-of-words using the Term Frequency Inverse Document Frequency (TF-IDF) formulae. The first part of TF-IDF is Term Frequency $TF_d,w$, which is a normalization of frequency that solves the first problem of larger documents being treated as more important than shorter ones. 

\begin{align*}
TF_{(w, d)} =  \frac{{wf}(d)}{\sum_{n} {wf}_n(d)} 
\end{align*}

Where ${wf}(d)$ is the number of occurrences of word $w$ in document $d$ and $n$ is the number of words overall in the vocabulary. Note that frequency is still important, its just that it is not important how frequent it is relative to other documents.  The next part of TF-IDF is Inverse Document Frequency, which is a measure that rewards terms that have a low Document Frequency. 

\begin{align*}
IDF_{w} =  \frac{d_n}{{df}(w)} 
\end{align*}

Where ${df}(w)$ is the amount of documents the word $w$ has occurred in and $d_n$ is the amount of documents in the corpus. Note that while Term Frequency measures the frequency of a term in a document relative to that documents length, Document Frequency measures the overall occurrences of the term across all documents, relative to the number of documents. Essentially, it measures how rare that term is for a document, rather than how rare it is for a word. Finally, the TF-IDF is just the Term Frequency multiplied by the Inverse Document Freqency.

\begin{align*}
{TF-IDF} = TF \times IDF
\end{align*}


\subsubsection{Positive Pointwise Mutual Information (PPMI)}



%If one movie review contains the word "scary" 100 times, "funny" 10 times and "romantic" 5 times, this can be represented as a vector for the movie review where each column is a word $[100, 10, 5]$. Representations like this can be used to find patterns that separate movies into genres, e.g. when comparing the previous vector to one for another movie that is more funny and less scary and romantic $[0, 100, 0]$, a simple pattern could be that \textit{IF $Scary_f$ > 50 THEN Movie is Horror} and \textit{IF $Funny_f$ > 50 THEN Movie is Comedy}.

%By extending this  so that each word in the vocabulary $w \in W$ has an associated frequency ${wf}(d)$ for each document  $d \in D$  the result is a vector for each document composed of word-frequencies  $d = ({wf}_1, {wf}_2, ..., {wf}_n)$, with ${wf}_1$ referring to the first word in the vocabulary, and so on until the final word $n$ in the vocabulary. By using these vectors as a representation of the text documents, the result is a matrix  with columns equal to the amount of words in the vocabulary $w_n$ and rows equal to the amount of documents $d_n$. 



%Using simple frequency has its problems. Even when grammar is removed, noisy words which are widely used can still be the most frequent for a document. For example, in a domain of movie reviews, the word "movie" would be the highest frequency for a variety of documents, which is not informative. This is solved by the following approaches:

Pointwise Mutual Information (PMI) comes from probability theory and information theory, and is a metric that measures how dependent  two variables are i.e. what is the difference between the chance that both variables occur at the same time independently. In this case, it can be used to measure how dependent a word is on a document. Obviously it is not possible to determine a precise probability that a word will occur, so in practice the frequency of the word is treated as an approximation of the chance it will occur. In application, we can understand that the word "the" is not independent from the document - it is a word that is just as likely to occur in one document than another because its occurrence is not dependent on the document. However, in a domain of movie reviews a word like "thrilling" would be more dependent on its associated text document, as it would only occur for movies which are thrilling. The pmi value for a word $w$ in a document  $d$ is given by:


\begin{align*}
\textit{pmi}(w, d) = \log\big(\frac{p_{{wf}(d)}}{p_{wf} \cdotp p_{d}}\big)
\end{align*}

where $P_{{wf}(d)}$ is equal to the chance of the word occurring in the document assuming they are dependent on each other 



\begin{align*}
P_{{wf}(d)} &= \frac{{wf}(d)}{\sum_{wf} \sum_{d} {wf}(d)}
\end{align*}

and ${wf}(d)$ is the frequency of a word for a document. To calculate the chance that a word will occur, we simply take the chance the word will occur in any document (estimated by its summed frequency) over all frequencies, and for the document we take the chance that the document  will occur (represented by the sum of the  frequencies of all words that occur in it) over all frequencies:
 
 \begin{align*}
 P_{wf} &= \frac{\sum_{d} {wf}(d)}{\sum_{wf} \sum_{d} {wf}(d)} &
 P_{d} &= \frac{\sum_{wf} {wf}(d)}{\sum_{wf} \sum_{d} {wf}(d)} &
 \end{align*}
 


As this value can sometimes be negative when words are less correlated than expected, we use Positive Pointwise Mutual Information (PPMI), as we are only interested in words which are positively correlated.

\begin{align*}
\textit{ppmi}_{{wf}(d)} = \max \big(0, pmi)
\end{align*}


\subsection{Dimensionality Reduction Techniques}\label{ch2:vectorspaces}

The most common representation is one based on frequency statistics: the Bag-Of-Words (BOW). This representation has the benefits of being easy to understand on a granular level, as each feature is a distinctly labelled word. However, it does not capture more complex relationships between words at a conceptual level. For example, it may be the case that in a text domain of movie reviews a horror movie frequently mentions the words "scary" "blood" and "gore", and through these terms occurring together we may infer that it is a Horror movie, or that other documents with similar words like "terrifying" or "blood" are similar documents. This is similarity information, which can be induced by taking the frequency vectors of a Bag-Of-Words and interpreting them in terms of relative cosine similarity. However, simply finding the similarity between each document would result in an \textit {N} by \textit {N} representation. Which is obtrusively large. In reality we do not need a dimension per document - it is more likely that a group of documenta e.g. Horror movie reviews from a domain of movies could be used as a "Horror" feature by averaging their dimensions together, the same with romantic movies or particularly cinematographally good ones. This is why typically these similarity representations are obtained with dimensionality reduction methods - with the general rule of thumb being that you want as many salient features as dimensions, where smaller dimensions represent more general concepts (with the most specific amount of dimensions being equivalent to the amount of words). 



In these vector spaces, regions form that describe properties in the domain. 



\subsection{Principal Component Analysis}

Principal Component Analysis is a dimensionality reduction method that results in dimensions ordered by importance. 



In application to text-processing, this matrix could be the document by word matrix of PPMI values. Essentially, we want to go from the extremely large and sparse PPMI matrix to a dense matrix of our own specified size. To do so, we can apply singular value decomposition (SVD) of a matrix that has been normalized. This method can only model linear relationships.


%Starting with a large data matrix, e.g. the PPMI bag-of-words, we first find the covariance matrix for these values. Then, from this covariance matrix we obtain the eigenvalues. We can then linearly transform the old data in-terms of this covariance matrix to obtain a new space of size equal to an arbitrary value smaller than our matrix.

%PPMI values capture the meaning of words in documents, but are extremely sparse, with one dimension of the vector for each word. This dimensionality makes it difficult to e.g. learn a shallow decision tree that is resistant to overfitting.
%Vector spaces are a popular way to represent unstructured text data, and have been broadly applied to and transformed by supervised %approaches. They vary in method, producing structure from Cosine Similarity, Matrix Factorization, Word-Vectors/Doc2Vec, etc. %More refs
%They also vary in how they linearly separate entities. %How?
%However, their commonality is that they are able to represent semantic relationships spatially. %ref
%See Section \ref{background:WhySpace}
%This brings up an essential point: When using a semantic space, are we taking advantage of relationships that are discriminative or incorrect? The danger of relying on these spaces and the models that use them has greatly affected their adoption in critical application areas like medicine, %Citation needed
%and has raised legal concerns about their application in e.g. determining if someone is suitable for a loan. 




\subsection{Multi-Dimensional Scaling}

The goal of multi-dimensional scaling is to create a space that spatially represents the dissimilarity between documents. So if two text documents are very dissimilar, they will be spatially distant from each other. The same as PCA, this is a dimensionality reduction where the amount of dimensions are specified, but it requires a $D_n x D_n$ dissimilarity matrix, which can consume a lot of memory when being applied. Instead of PCA, where dimensions are ordered by importance, the resulting dimensions of MDS are not as clearly meaningful. However, this method can model non-linear relationships.

\section{Text Document Classification}\label{ch2:classifiers}

Problems that machine-learning can solve can be split into two distinct categories, supervised and unsupervised. Supervised problems have some data that is labelled, and some that is not labelled. The goal of a supervised task is to assign labels to the data that is not labelled, by learning with the data that is labelled. Unsupervised problems do not have any labels, and instead try to solve a problem just from unlabelled data. The learning method to solve the problem is called a machine-learning model. In a supervised task this model  learns using the labelled examples, and then is used to label the unlabelled examples.

Representation are  used to learn how to separate different kinds of objects in a domain. This is called a classification problem which can be described as follows: given a set of possible labels $c \in C$, documents $D$ and document/label pairs assigned a binary truth value $(d, c) = {0, 1}$ find a function with a classifier $FUNCT$ that assigns unlabelled documents $d \in D$ labels $(d, p) $ approximates an unknown target function that can accurately label any document. For example, in a domain of movie reviews labelled with if that review is positive or negative, find a function that can determine if unlabelled movie reviews are positive or negative. In this case we use classifier to refer to the method to obtain the function.

If the classifier performs well, and can predict a variety of unlabelled documents  the representation must represent enough domain knowledge  to produce that function. This is why classification tasks can measure how good a representation is, if they can perform on key domain tasks like predicting the genre of a movie based on its movie reviews then they clearly represent fundamental semantic information about movies.  As an example,  the bag-of-words can be considered a good representation if the frequencies of sentiment-related words, like "good", "bad", and "thrilling" would be good enough to achieve reasonable performance, as a machine-learning classifier could determine rules based on the frequency of these relevant words, e.g. "IF good > 30, and thrilling > 20, THEN positive sentiment". The tasks that are solved in this thesis are all classification tasks. 


\subsection{Binary Classification}
\subsection{Multi-Label Classification}
\subsection{Training, Testing, Validation Sets}

When using a bag-of-words as a representation in a classification task, that threshold is usually set higher, as it is more important to remove noise that would not naturally be removed when using the bag-of-words to create another representation. 

Classification of text documents can be used for example to identify if social media posts, product reviews, etc are positive or negative \cite{Burel2018},  identify social media posts that happen during crises and automatically categorize them to be useful to responders \cite{Burel2018},  or detect infections acquired while patients are in a hospital . However, text documents like news articles, product reviews or social media posts cannot be classified without first being represented computationally.  Representations $r$ are composed of features $r = (x_1, x_2, ..., x_n)$, where ideally each feature $x$  is meaningful in the domain. For example, meaningful features when determining the  value of a house would be the amount of bedrooms $x_1$, and the amount of toilets $x_2$. An example vector from these examples would be $[5,2]$ for a house with 6 bedrooms and 3 toilets.

One way to obtain features for text documents is to use the frequencies of words in that document. As a vector $d = ({wf}_1, {wf}_2, ..., {wf}_n)$ with ${wf}_1$ referring to first word in the vocabulary $w \in W$, and ${wf}_n$  referring to the final word in the vocabulary. This is called a Bag-Of-Words (BOW), called as such because word-order is not retained. However, to have a consistent bag-of-words representation, the text must be normalized so that any word $w \approx w$ will  $w = w$, so where a word varies in format but not alphanumeric characters it is treated as the same word, e.g. "Wow, wow, WOW!!!" would be treated as  "wow wow wow". This is a common step taken when producing a representation from text, where it is simplified to make it easier to represent.

In-order to classify documents, a label $y$ is required. Labels can be understood as categories in the domain, e.g. in the domain of sentiment analysis on movie reviews, labels could be "very good", "good", "average", "bad", "very bad" represented as $[0, 1, 2, 3, 4]$. Given features $x$ and labels $y$, a model $m$ learns a way to predict the label of a document given its features, and this learned method can be applied to unlabelled documents. For example, given a bag-of-words representation, one way to automatically label a news article category would be with thresholds on the frequency of words in the text e.g. if the word "amazing" and the word "great" both occur more frequently than a threshold $T$ determined by a model, then the label is $0$,  a "very good" movie. 

\subsection{Evaluation Metrics}

To evaluate a model, the difference between the real labels of documents and the predicted features of documents is compared. However, the value of the model is in its ability to predict the labels of documents that are unlabelled. Typically, this problem is solved by splitting the documents into a training set and a test set. The training set is used when learning the model, and the test set is used to verify the model is working correctly. 

Here, we assume we are classifying a single binary class, where positive labels are 1 and negative labels are 0. The most simple way to evaluate a model is by its accuracy $a$, where ${t_n}$ is the number of correct predictions, and $P_n$ is the number of all predictions.

\begin{align*}
a &= t_n / P_n  &
\end{align*}

However, this can give a misleadingly high score if for example, the dataset is unbalanced with many more negative labels than positive ones, and the model predicts only negatives. An example of where this would be the case is when classifying out of all social media posts, which ones are important for emergency responders to investigate. Although there are very few positive instances of this class, identifying those is very important. In the case of a model predicting only negatives, the accuracy would be high as the number of correctly predicted negatives ${tn}$ is high, but the model has not actually learned anything, which we can tell by looking at the number of correctly predicted positives ${tp}$. For a metric that can take this into account, we must consider the number of incorrectly predicted positives (negatives classified as positive) ${fp}$ and the number of incorrectly predicted negatives ${fn}$.

In this situation, the metric we would want to optimize would be recall. Recall ${rec}$ is the proportion of true positives ${tp}$ identified correctly. 

${rec} = {tp} / {tp} + {fn}$

In the case of a model predicting only negatives, the ${rec}$ would be zero. Recall is useful in these situations where we are interested in how many false negatives ${fn}$ there are. However, if the model is instead prioritizing positive predictions too much rather than negative ones, we can use precision ${pre}$

${pre} = {tp} / {tp} + {fp}$



\subsection{Decision Trees}\label{bg:trees}

Decision Trees are a model that produce a tree of decisions, composed of nodes. Each node has its own feature and associated threshold value $T$.  If the value given in the feature is larger than the threshold $T$, then it traverses one direction, otherwise it traverses the other direction. Eventually, upon reaching the leaf node the decision made on the threshold is the classification of the document. Decision Trees work well as they are simple, and easy to understand. However, to model most complex relationships in e.g. an MDS space, there would need to be many nodes to achieve a strong classification result, as the features are not necessarily meaningful independently. Additionally with a vector space, it is not easy to understand, as the features used are not clearly meaningful. Ideally we would want clearly labelled features like in a bag-of-words, but to model complex domain tasks this would require many nodes, which greatly increases complexity of the tree. 



%\begin{itemize}%
%	\item Explanation of what decision trees are
	%\item Explanation that they may not perform well on sparse information
	%\item Max features
	%\item Criterion
	%\item CART decision trees versus others
%\end{itemize}





\section{Neural Networks}



\subsection{Feedforward Neural Networks}

\subsection{Word Vectors}\label{bg:WordVectors}

\subsection{Doc2Vec}




\section{Support Vector Machines}\label{bg:SVM}
\begin{itemize}
	\item Performance increase for support vector machines on sparse data, balancing, etc
	\item C parameters, gamma parameters
\end{itemize}



\section{Clustering}\label{bg:clustering}

\subsection{K-means}

\subsection{Derrac's K-means Variation}

\section{Interpretable Representations}

Although dimensionality reduction does achieve similar results on tasks as sparse representations, the dense vector space representations usually no longer have features which are meaningful to humans. This is a trade-off when going from a sparse representation to a dense representation, the features are no longer meaningful. Interpretable representations are those representations where the features are meaningful, which bag-of-words is a part of. However, this work instead focuses on obtaining dense fine-grained interpretable features that retain the dimensionality reduction quality. In this case, fine-grained means that the features are rich rather than sparse, detailing exact differences between documents according to features e.g. in a ranking. For this reason, we do not include representations in this section that claim interpretability but are sparse, instead focusing on those methods that are able to achieve a similarly dense representation that also has interpretable features.

\subsection{Topic Models}\label{bg:TopicModels}

%\section{Classification}
%Classification, particularly document classification is separating documents $d$ into labels $y$ using their features $x$, typically with a machine learning model $m$. Here, we explain some classifiers, using the bag-of-words PPMI representation introduced earlier as an example representation, and the datasets introduced in \ref{chapter3:datasets}




% What representations are semantic spaces? What is not a semantic space?
%\subsubsection{How do vector spaces represent semantics? Why do we use them to represent semantics?}\label{background:WhySpace}
%Distributional representations of semantics, known as 'semantic spaces' are well-recognized for their ability to represent semantic information spatially. These representations have been widely adopted for Natural Language Processing (NLP) tasks %Tasks here
%thanks to their ability to represent complex information in a dense representation. In particular, entity-embeddings have been applied  to represent items in recommender systems \cite{Vasile:2016:MPE:2959100.2959160,liang2016factorization,van2016learning}, to represent entities in semantic search engines \cite{DBLP:conf/sigir/JameelBS17,van2017structural}, or to represent examples in classification tasks \cite{DBLP:conf/iccv/DemirelCI17}. %Copied from CONLL paper. Shift this to talk more about applications and tasks rather than specific stuff related to our ideas.





\section{Interpretable Representations}\label{ch2:Interpretability}
a. NNSE
b. compositional
c. 2007 paper as wikipedia similarities
d. Topic models\label{bg:TopicModel}
%The interpretable representation that is obtained by this method is composed of in terms of salient features, where each of these features is described using a cluster of natural language terms. This is somewhat similar to Latent Dirichlet Allocation (LDA), which learns a representation of text documents as multinomial distributions over latent topics, where each of these topics corresponds to a multinomial distribution over words \cite{Blei03latentdirichlet}.  Topics tend to correspond to salient features, and are typically labelled with the most probable words according to the corresponding distribution. On the other hand, our work leverages clustering methods to obtain the feature labels. Many extensions of LDA have been proposed to incorporate additional information as well, e.g.\ aiming to avoid the need to manually specify the number of topics \cite{teh2005sharing}, modelling correlations between topics \cite{Blei2006}, or by incorporating meta-data such as authors or time stamps \cite{rosen2004author,wang2006topics}. Nonetheless, such techniques for extending LDA offer less flexibility than neural network models, e.g.\ for exploiting numerical attributes or visual features. For comparison, in our experiments the standard topic model algorithm Latent Dirchlet Allocation (LDA) is used as a baseline to  compare to the new methodology that transforms standard Vector Space Model representations. 
e. Infogan, etc

%There is much work on learning interpretable representations, with one popular way being to introduce sparsity or non-negativity constraints while learning, for example, sparse PCA learned using the l1-norm, \cite{H.Zou2006} \cite{Zhang2012},  or Non-Negative Sparse Embeddings (NNSE)  \cite{Murphy} which are sparse interpretable word-vectors obtained using sparse-matrix factorization and non-negativity constraints. A similar technique can also be applied to distributional word-embeddings by integrating this method with the Skip-Gram model \cite{Luo2015}. However, our approach is not intended to transform the learning processes, but rather be a post-processing step on an existing representation.

%Similar to the approach in this chapter, \cite{Faruqui2015} introduce a post-processing method to convert any distributional word-vector into sparse word vectors, which additionally satisfy our idea of disentangled interpretability. However, the representation produced by the method in this work differs from sparse representations in that it is dense, where each feature is salient and interpretable. Another method is to describe a representation, e.g. sense word-embeddings that are linked to synsets \cite{Panchenko2016} in-order to make them interpretable. Although this is a post-processing step similar to our method, this is a linking rather than a transformation of the representation.  

%Another method is to integrate grammatical structure into the learning of the representation, for example \cite{Liu2017} obtained a representation learned with attention mechanisms on the dependency structures of sentences, but this differs from the intention of our work, which is not to introduce new structures to the representation to make it more interpretable but instead use the already existing structure to obtain an interpretable representation. For short interpretable documents, \cite{Martinc} introduced tax2vec, which produced interpretable features from word taxonomies, useful for low data models. In \cite{Code} word-vectors were clustered and then used as a bag-of-clusters, where if a word occurs in those word-vector clusters it contributes to the Bag-Of-Words frequency. Although clustering is used in the method, it is not used to create a Bag-Of-Words, instead relying on the spatial relationships in the space as our representation. 
\cite{Zhang2012} Sparse PCA (Why not compare lol)

Vector space models typically use a form of matrix factorization to obtain low-dimensional document representations. By far the most common approach is to use Singular Value Decomposition \cite{ASI:ASI1}, although other approaches have been advocated as well. 
Instead of matrix factorization, another possible strategy is to use a neural network or least squares optimization approach. This is commonly used for generating word embeddings \cite{DBLP:conf/nips/MikolovSCCD13,glove2014}, but can similarly be used to learn representations of (entities that are described using) text documents \cite{DBLP:journals/corr/DaiOL15,van2016learning,DBLP:conf/sigir/JameelBS17}. Compared to topic models, such approaches have the advantage that various forms of domain-specific structured knowledge can easily be taken into account. Some authors have also proposed hybrid models, which combine topic models and vector space models. For example, the Gaussian LDA model represents topics as multivariate Gaussian distributions over a word embedding \cite{DBLP:conf/acl/DasZD15}. Beyond document representation, topic models have also been used to improve word embedding models, by learning a different vector for each topic-word combination \cite{DBLP:conf/aaai/LiuLCS15}. %copy pasted

The most commonly used representations for text classification are bag-of-words representations, topic models, and vector space models. Bag-of-words representations are interpretable in principle, but because the considered vocabularies typically contain tens (or hundreds) of thousands of words, the resulting learned models are nonetheless difficult to inspect and understand. Topic models and vector space models are two alternative approaches for generating low-dimensional document representations. %copy pasted

\subsection{Word Vectors}
